%%%%%%%%%%%%%%%%%%%%%%%%%%%% -*- Mode: latex -*- %%%%%%%%%%%%%%%%%%%%%%%%%%
%%
%% Copyright (c) 1991-2003 Francois Felix Ingrand.
%% All rights reserved.
%%
%% Redistribution and use in source and binary forms, with or without
%% modification, are permitted provided that the following conditions
%% are met:
%%
%%    - Redistributions of source code must retain the above copyright
%%      notice, this list of conditions and the following disclaimer.
%%    - Redistributions in binary form must reproduce the above
%%      copyright notice, this list of conditions and the following
%%      disclaimer in the documentation and/or other materials provided
%%      with the distribution.
%%
%% THIS SOFTWARE IS PROVIDED BY THE COPYRIGHT HOLDERS AND CONTRIBUTORS
%% "AS IS" AND ANY EXPRESS OR IMPLIED WARRANTIES, INCLUDING, BUT NOT
%% LIMITED TO, THE IMPLIED WARRANTIES OF MERCHANTABILITY AND FITNESS
%% FOR A PARTICULAR PURPOSE ARE DISCLAIMED. IN NO EVENT SHALL THE
%% COPYRIGHT HOLDERS OR CONTRIBUTORS BE LIABLE FOR ANY DIRECT, INDIRECT,
%% INCIDENTAL, SPECIAL, EXEMPLARY, OR CONSEQUENTIAL DAMAGES (INCLUDING,
%% BUT NOT LIMITED TO, PROCUREMENT OF SUBSTITUTE GOODS OR SERVICES;
%% LOSS OF USE, DATA, OR PROFITS; OR BUSINESS INTERRUPTION) HOWEVER
%% CAUSED AND ON ANY THEORY OF LIABILITY, WHETHER IN CONTRACT, STRICT
%% LIABILITY, OR TORT (INCLUDING NEGLIGENCE OR OTHERWISE) ARISING IN
%% ANY WAY OUT OF THE USE OF THIS SOFTWARE, EVEN IF ADVISED OF THE
%% POSSIBILITY OF SUCH DAMAGE.
%%
%% $Id$
%%
%%%%%%%%%%%%%%%%%%%%%%%%%%%%%%%%%%%%%%%%%%%%%%%%%%%%%%%%%%%%%%%%%%%%%%%%%%%%%%%

\part{\OPC{}}
\node{OP Compiler}

\chapter*{Overview of the \OPC{}}
\node{Overview of the OP Compiler}
\cpindex{Overview of the \OPC{}}
\cpindex{Introduction to the \OPC{}}


The \OPC{}  is the program which compiles OP files (textual or graphical) in 
\OPRS{} internal code. There are a number of good reasons to use such
program:

\begin{itemize}
  
\item it is nuch faster to reload compiled OPs than non compiled one (as the
  kernel must compile them at loading time).

\item the compiled code is portable, and can thus be reloaded on any
  compatible \CPK{}.

\item one needs to use compiled OPs if one has a \COPRSAE{}. Indeed,
  \COPRSAE{} are runtime environment only and can only load OPs produced and
  compiled in a \COPRSDE{}.

\end{itemize}

The \OPC{} is in fact a part of the \CPK{}, and most of the code thus shared 
with the \CPK{}. 

\chapter{How to Use the \OPC{}}
\node{How to Use the OP Compiler}
\cpindex{How to Use the \OPC{}}

The \OPC{} is only part of the \COPRSDE{}. Moreover, if one produce \COPRS{}
code for a \COPRSAE{}, one need to compile the OP before passing them to
the\COPRSAE{}.

\section{Argument of the \OPC{}}
\node{Argument of the OP Compiler}
\cpindex{Argument of the \OPC{}}

Usage: \begin{verbatim}
opc [-v] [-t] [-X] [-e]* [-o output-dopf]*
    [-i command-file]* [-l input-opf]*
    [-d imputdopf]* [-p oprs-data-path]* op-file*
\end{verbatim}


Most arguments are optional and can be used more than once. The order of the 
argument is very important as they get executed in the ordre they are given.

\begin{description}

\item[\code{-v}] will produce verbose output.

\item[\code{-t}] will compile ops in text mode.

\item[\code{-X}] will compile ops in graphic mode, the default.

\item[\code{-e}] will empty the internal op database.

\item[\code{-o output-dopf}] will  the compiled op present in the op db in
  the specified file.

\item[\code{-i command-file}] will include an include .inc or symbol .sym file.

\item[\code{-l input-opf}] will load a .opf file in the op db

\item[\code{-d imputdopf}] will load a .dopf file in the op db.

\item[\code{-p oprs-data-path}] will set the oprs-data-path (overiding any
  previous value).

\item[\code{op-file}] will empty the op db, load op-file and dump the compiled version.

\end{description}

\section{\OPC{} Environment Variables}
\node{OP Compiler Environment Variables}
\cpindex{OP Compiler Kernel Environment Variables}

There is one environment variable which can be used to customize the \OPC{}.
However, the argument passed using the command line has precedence on the one
acquired from the environment variables.

\begin{description}


\cpindex{OPRS\_DATA\_PATH}
\item[\code{OPRS\_DATA\_PATH}] is used  to specify a data path, i.e.\ a colon separated
list of directories where the kernel will look for data files (\file{.inc},
\file{.opf} and \file{.db}). It is used by the \CPK{} and the \XPK{}. It is
equivalent to the \code{-p} command line argument. Note that the use of
\code{-p} will overide any value previously set (with \code{OPRS\_DATA\_PATH} 
or wih a previous \code{-p}).\*
Example:
\begin{verbatim}
setenv OPRS_DATA_PATH ./data:/usr/local/oprs/data:${HOME}/data
\end{verbatim}

\end{description}

\section{Using the \OPC{}}
\node{Using the OP Compiler}
\cpindex{Using the \OPC{}}

\section{Errors Reported by the \OPC{}}
\node{Errors Reported by the OP Compiler}
\cpindex{Errors Reported by the \OPC{}}

A certain number of errors can be reported by the \OPC{}.

%%% Local Variables: 
%%% mode: latex
%%% TeX-master: "oprs"
%%% End: 

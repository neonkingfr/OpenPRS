%%%%%%%%%%%%%%%%%%%%%%%%%%%% -*- Mode: latex -*- %%%%%%%%%%%%%%%%%%%%%%%%%%
%%
%% Copyright (c) 1991-2011 Francois Felix Ingrand.
%% All rights reserved.
%%
%% Redistribution and use in source and binary forms, with or without
%% modification, are permitted provided that the following conditions
%% are met:
%%
%%    - Redistributions of source code must retain the above copyright
%%      notice, this list of conditions and the following disclaimer.
%%    - Redistributions in binary form must reproduce the above
%%      copyright notice, this list of conditions and the following
%%      disclaimer in the documentation and/or other materials provided
%%      with the distribution.
%%
%% THIS SOFTWARE IS PROVIDED BY THE COPYRIGHT HOLDERS AND CONTRIBUTORS
%% "AS IS" AND ANY EXPRESS OR IMPLIED WARRANTIES, INCLUDING, BUT NOT
%% LIMITED TO, THE IMPLIED WARRANTIES OF MERCHANTABILITY AND FITNESS
%% FOR A PARTICULAR PURPOSE ARE DISCLAIMED. IN NO EVENT SHALL THE
%% COPYRIGHT HOLDERS OR CONTRIBUTORS BE LIABLE FOR ANY DIRECT, INDIRECT,
%% INCIDENTAL, SPECIAL, EXEMPLARY, OR CONSEQUENTIAL DAMAGES (INCLUDING,
%% BUT NOT LIMITED TO, PROCUREMENT OF SUBSTITUTE GOODS OR SERVICES;
%% LOSS OF USE, DATA, OR PROFITS; OR BUSINESS INTERRUPTION) HOWEVER
%% CAUSED AND ON ANY THEORY OF LIABILITY, WHETHER IN CONTRACT, STRICT
%% LIABILITY, OR TORT (INCLUDING NEGLIGENCE OR OTHERWISE) ARISING IN
%% ANY WAY OUT OF THE USE OF THIS SOFTWARE, EVEN IF ADVISED OF THE
%% POSSIBILITY OF SUCH DAMAGE.
%%
%% $Id$
%%
%%%%%%%%%%%%%%%%%%%%%%%%%%%%%%%%%%%%%%%%%%%%%%%%%%%%%%%%%%%%%%%%%%%%%%%%%%%%%%%

\part{Using \COPRS{}}
\node{Using OPRS}
\cpindex{Using \COPRS{}}



\chapter{Introduction to Using \COPRS{}}
\node{Introduction to Using OPRS}
\cpindex{Introduction to Using \COPRS{}}

Now that you have read through the whole documentation (you still have the
Appendices\dots), you may start to wonder how the different parts fit in.  You
have got this real-time, control and supervision application to solve, and you
get the strange but growing feeling that \COPRS{} is the tool you need.  Good.
This part of the manual go through the various process you have to do to set up
your own application.

We are first going to go through a simple \OPE{} and \XPK{} session. We
will then  examine the various questions you have to ask
yourself before getting started. How many kernels do I need? what do I
put in the database? how many OPs? do I need meta level reasoning? how
can I optimize my application? etc...

Then we will examine various applications and justify the answer to the
aforementioned questions.

\chapter{Setting Up your Environment}
\node{Setting Up your Environment}
\cpindex{Setting Up your Environment}

To be able to use the \COPRSDE{} you may need to set up a number of environment
variables.

\begin{verbatim}
setenv OPRS_DIR /usr/local/oprs
setenv PATH "${PATH}:${OPRS_DIR}/bin"
if ( $?XFILESEARCHPATH ) then
         setenv XFILESEARCHPATH "${XFILESEARCHPATH}:${OPRS_DIR}/\%T/%N"
else
         setenv XFILESEARCHPATH "${OPRS_DIR}/%T/%N"
endif
setenv OPRS_DOC_DIR ${OPRS_DIR}/doc
setenv OPRS_DATA_PATH  ${OPRS_DIR}/data:./data:.
\end{verbatim}

\chapter{Getting Started}
\node{Getting Started}
\cpindex{Getting Started}
\cpindex{How to Start}
\cpindex{Where to Start}

To understand this chapter, the reader is supposed to have some common
knowledge about the following subject: Unix, X11, X server, Motif, Xt. 

Assuming \COPRSDE{} has been properly installed at your site, and assuming
you have the \COPRSDE{} in ``your path'' (i.e. Unix will find the command
if you invoke them), then you can now start to use \COPRS{}.



\section{Getting Started with the \OPE{}}
\node{Getting Started with the OP Editor}
\cpindex{Getting Started with the \OPE{}}

To get started, you can probably just call the \OPE{}. This tool described in
detailed (\pxref{How to Use the OP Editor}), is used to display/edit/create
procedures as they are represented in used in the \COPRSDE{}.

The \OPE{} is a graphical tool and can only be used if and only if you are
running an X server. To call the \OPE{}, just type (at the Unix prompt):

\code{\% op-editor}

This should bring the \OPE{} window. Then, you can either load an already
existing OP file, or you can start to create your own procedure.

To load an existing OP File, you must select the Load OP File menu item
from the File menu, \nxref{Load OP File}. This will bring a File Selection
Dialog Box from which you can select a file to load. Usually, data files
are available in the \file{/usr/local/oprs/data/} directory. If your version
of the \OPE{} is properly installed, the File Selection Dialog Box should
point to this directory.

After the OP file is loaded, you will be asked (with a List Selection
Dialog Box) to select which procedure to edit from this file (keep in mind
that a OP File can contain more than one procedure). If you select a
procedure, then it should appear on the screen. In case, you do not select
any procedure, you can still select a procedure later by using the Select
OP menu item from the OP Menu \nxref{Select OP}.

You can now browse through the different menus available in the menu bar,
\nxref{Menubar of the OP Editor}. You will see that there are many
commands available to select/load/save/unload OP File, as well as command
to
deal with procedures and OPs.

If you decide to create your own procedure, you must select the Create OP
menu item from the OP menu (\pxref{Create OP} and \pxref{Creating a OP}).

When you have a OP selected on the screen, you need to choose from the
Working Menu (the menu on the left of the drawing area) which operation
you want to perform. Here also, there are a number of operations
available. Each operation is more or less self explanatory, when
selected, it stay selected until you select another operation. Note that
at any time you can consult the Help Footer (\pxref{Footer and Dialog Box Help}) for information on how to proceed.

\section{Getting Started with the \XPK{}}
\node{Getting Started with the X-OPRS Kernel}
\cpindex{Getting Started with the \XPK{}}

Before starting a \XPK{}, you must start a \OPRSS{} (this operation
will also start a \MPA{}). If you attempt to run a \XPK{} or a \CPK{}
without any \OPRSS{} running, the program will exit with an error
message stating that it cannot connect to the \MPA{} (which is mandatory
for \XPK{} and \CPK{}).

So, before doing anything, you need to call a \OPRSS{} by issuing
the following command at the Unix prompt:

\code{\% oprs-server}

From then, you can either start a \XPK{} from the \OPRSS{} by issuing
the command at the \OPRSS{} prompt:

\code{OPRS-Server> make\_x foo}

This will create a \XPK{} named FOO. Another way to obtain a \XPK{} is to
issue (at the Unix prompt) the command:

\code{\% xoprs foo}

and then to issue (at the \OPRSS{} prompt) the command:

\code{OPRS-Server> accept}

In both case, you should get the \XPK{} window on your screen.

At this stage, this \XPK{} is fully functional and can execute any
procedure you will load in it. To try a well known example, you can load
the file \file{/usr/local/oprs/data/fact-meta.inc}. This include file will
load various OP files (\file{/usr/local/oprs/data/fact-meta.opf} and
\file{/usr/local/oprs/data/new-default.opf}). To load the include file,
you have to select the Include File command from the File Menu
(\pxref{Include}).

You should see in the text pane various information about the procedures
which are loaded.

Now, your kernel is running, it has some procedures loaded, you need to
post a new goal or a new fact. For this particular example you just loaded,
you need to post the goal: \code{(! (print-factorial 3))}, or any
particular integer value. If you try with an integer > 30, you will most
likely reach the maximum integer value, and will get an erroneous result.
You should then use float (\code{(! (print-factorial 150.0))}). In any
case, to post a new goal or a new fact, you need to select the Add Fact or
Goal menu item from the OPRS menu (\pxref{Add Fact or Goal}), type the goal,
and click on OK. You will see the result printed on the screen, in the text
pane. During the run, you may  noticed some activity in the intention
graphic trace pane. It is the evolution of the different intentions
which appear and then disappear.

You can make as many runs as you want and play with the different trace
and execution options available to the user. For example you can select
the OP Graphic Trace menu item in the Trace Menu (\pxref{OP Trace/Step}). This will pop up a List Selection dialog box where you can
select the OP for which you want a graphic trace of their execution. For
this particular application, it is a good idea to trace the OP:
\code{Print Factorial, Iterative Factorial, Recursive Factorial and Meta
Factorial}.

Now that you are becoming more and more familiar with the interface, you
can start to play with the \xref{Control and Status Panel}. For example,
you can halt the kernel, then add a new goal, and then click on Step of
Next to see a step by step execution (see \nxref{Control and Status Panel}, for an explanation of the differences between Next and Step.

\chapter{Setting Up a \COPRS{} Application}
\node{Setting Up a OPRS Application}
\cpindex{Setting Up a \COPRS{} Application}

Setting up a \COPRS{} application is an easy task. However, there are a number
of questions you have to ask yourself, or ask the final users or the experts
who will bring the knowledge in the system. These questions mainly relates to
the structure of the problem and to the way you want to solve it using \COPRS{}.
Keep in mind that \COPRS{} is a shell, a tool, and if its functionalities makes
it particularly well suited for a certain type of application, you still need
to do some work to organize your application.

In this chapter we will examine all these questions you have to ask
yourself and put forward some elements to help you to answer them.



\section{How Many \CPK{}s Does it Takes to Screw a Light Bulb?}
\node{How Many OPRS Kernels Does it Takes to Screw a Light Bulb?}
\cpindex{How Many \CPK{}s Does it Takes to Screw a Light Bulb?}

The answer is 42.

In fact, this is indeed a question you have to ask yourself before
starting to implement your \COPRS{} application. The answer is in your
application itself, you have to analyze it and figure out the answer. To
help you we can go through a number of criteria which participate to the
decision.

% **************************************
% ********** To be completed ***********
% **************************************

\paragraph{How many agents are involved in your application?} Indeed,
the number of agents are currently used to solve the problem is a good
indication of the number of kernels you may need to solve it with
\COPRS{}. For example, if your problem involved two agents working on
their own part of the problem, exchanging information from time to time,
then using two kernels seems a wise choice. Another possible situation is
when you have a large number of agents doing ``more or less'' the same
task, but either in a distributed way, or for their own objectives.

\paragraph{Can all the computation be done by one kernel?} This is an
important consideration. You must evaluate the amount of computation which
will be done for the application and consider its distribution if it
appear to be to large for a single kernel/platform. Keep in mind that
the kernel will not only execute the procedure you loaded but also the
evaluable functions, predicates, and actions you have linked in the
kernel. Some of these operations can be rather time consuming and
required to be sliced or distributed.

\paragraph{Is the application composed of different independent modules
or steps?} When the resolution of the problem is well defined in
different modules, or can be broken down in a number of well separated
steps, then one can consider implementing these different modules or
step in separate kernels.

\paragraph{Is the shared information between modules small?} This is an
important consideration to take into account. You have to make sure that
the amount of information between the different kernels is not too large.
Otherwise, you will loose in communication time, the time you have spare
in distributing your application.

\section{\CPK{}s or \XPK{}s}
\node{OPRS Kernels or X-OPRS Kernels?}
\cpindex{OPRS Kernels or X-OPRS Kernels}

This question is easy to answer. First from a functional point of view, the
two kernels are identical. However, due to some obvious overhead induced by
X and Xt the \XPK{} is slower than the \CPK{}. Keep in mind though that the
\XPK{} provides some interesting tracing and debugging capabilities which
are missing in the \CPK{} (the \CPK{} still provide some trace, but they
are not as visual as the one provided in the \XPK{}).

It is usually a good practice to develop the various \COPRS{} agents with
the \XPK{}, just to be able to trace and debug the OPs. Then, when a set
of procedure has been debugged and run flawlessly, one can load them in a
\CPK{}, you should not have any problem, except with the increased
performance.

Note however, that for some applications, the final user or the operator
may need to ``see'' the procedures executing. Therefore, it may be
necessary to run these procedures in an \XPK{}.

Similarly, there are some applications for which there are no need to
follow the graphic execution of the procedure, or to follow the graphic
evolution of the intentions graph. In this situation, one may run these
procedure in a \CPK{}.

Of course, one can mix \XPK{}s and \CPK{}s in the same application.

An important criterion to evaluate, regarding the use of \CPK{} or \XPK{},
is the respective size of these two kernels. The size of each kernel
depends of the availability of shared libraries on your system, the debug
flag with which the kernels were compiled and more generally the size of
binaries on your machine. Nevertheless you will noticed a
great difference in size between the \XPK{} and \CPK{} (the ratio can be of
one to ten...). Therefore, depending on the size of the swap space and the
size of memory on your machines, and the number of kernels involved, you
may have to limit the number of \XPK{}.

\section{The Database: Facts, Only the Facts}
\node{The Database: Facts Only the Facts}
\cpindex{The Database: Facts, Only the Facts}

The database has a very important and critical role in a \COPRS{}
application. It is the memory of the ``system'', and it is heavily used
by the kernel to retrieve, check, conclude information. There are many
questions to be asked before you start building your database. This
section will go through these questions and will bring some elements of
answer.



\subsection{The Representation of Facts}
\node{The Representation of Facts}
\cpindex{The Representation of Facts in the Database}

A very common question is what is exactly the proper way to represent
facts or beliefs in the database? Which format should be used? What is
the proper way to represent such and such information and what is the
wrong way to do it?

The format used to represent facts in the database is simple: \\*
\code{(predicate-name <arguments>*)}.

The semantic associated with the predicate name and the arguments (and
the order of the arguments) is yours, and should be consistent through
the whole database and application.

For example, if you consider the predicate \code{position}, you probably
want to associate it to the idea of devices or objects being in
particular position. Such as in: \code{(position valve open)}, or
\code{(position switch on)}, or even \code{(position robot unknown)}.

Note that for this particular predicate, we put first the argument to which
this information apply, and then the argument which represent the ``value''
of this predicate for the first argument... However, this is not always
the case. For example, a predicate such as \code{connected} which would
represent that two entities are somehow ``connected'' in your application
may not bear any particular meaning on the order of the argument.
Therefore, \code{(connected a b)} would be semantically equivalent to
\code{(connected b a)}. However, from a strict database point of view,
these two facts are different and it is the user task to make these 2
equivalent when needed. To summarize, the order of arguments may be
meaningful or not, but in any case, it is up to the user to ensure that
this order is properly respected through the whole application when it is
meaningful, and to ensure that it is properly used when it is meaningless.

An important consideration, is to keep in mind that the predicate name is
always first, and cannot be replaced with a variable in any context. For
example, it is forbidden to write in a OP something like: \code{(=> (\$x a
b))}... Even if the variable \code{\$x} is bound to the symbol \code{connected}.
The reasons for this limitation take their roots in the ``First Order Predicate
Calculus''.

The last consideration to take into account relate to the goal
representation in \COPRS{}. Do not forget that goals are built upon facts.
According to the temporal operator used, they modify the semantic of the
fact they are qualifying. As a consequence, it is important that the
facts underlying the goal representation be consistent with the goal
representation.

If one consider the fact \code{(position door open)}, then we can easily
build on this fact to get the following goal: \code{(! (position door
open))} to close open the door if it is not already open, \code{(?
(position door open))} to test if it is currently open , \code{(\^{}
(position door open))} to wait until somebody open the door (if it is not
already open), etc... In all these examples, we can see that the fact and
the goals are really carrying the wanted information. Moreover, when any
of these goal is attempted, the corresponding facts is checked in the
database...

\subsection{Which Predicate?}
\node{Which Predicate?}
\cpindex{Which Predicate?}

Now that the notion of predicate is clear. Remains the problem of
deciding which predicate should be used and for what.

First, the semantic of the predicates is the one the user wants to give
to the predicate. If you want to use the word \code{position} to
represent things we have nothing to do with position, you can... This will
just make your OP unreadable...

Similarly, you can overload the semantic of a predicate. The predicate
\code{position} can be used in an application to represent the position
of various objects, and even different types of objects: switches,
valves, lights, etc. An another alternative is to use predicate such as
\code{switch-position} or \code{valve-position}. Here also, there is no
rules, it is question of readability of the OPs and the database.

To help organize a database, very often, one use a \code{type} predicate
to specify the type of the various object. This is very useful for
example when one looks for all the open valve of the system: \\*
\code{(\& (type valve \$v) (position \$v open))}.

\subsection{Which Predicate Should be Declared as Closed World Predicate?}
\node{Closed World Predicate?}
\cpindex{Which Predicate Should be Declared as Closed World Predicate?}

This is an easy question to answer... In general, most if not all
predicates should be declared as closed world predicates. Why? Because
this is usually the way we tend to think, i.e. if something is not know
to be true, then its contrary is usually true. To use the example given
in section \nxref{Closed World Predicates}, if we do not have any
information about a direct flight between Toulouse an San Francisco, most
likely, no such flight exist, and therefore, its negation is true.
However, keep in mind that this information will be looked at (the fact
that this specific fact is not in the database, and the fact that it is a
CWP) if and only if you consult the negation of the fact. In other words,
there are no need to declare CWP predicate for which you never retrieve
the negation.

Last, there may exist some predicates which are not closed world
predicate, i.e.
for which a request of the negation of something which is not known as
true is false...

\subsection{Which Predicate Should be Declared as Functional Facts?}
\node{Functional Facts?}
\cpindex{Which Predicate Should be Declared as Functional Facts?}

\xref{Functional Facts} explains in great details the logic
behind functional facts (in short FF).

The important things to remember are:

\begin{itemize}

\item FF are expensive.

\item Only the predicates for which there exist a ``functional''
representation can be considered as FF.

\item Declare predicate as functional fact, if and only if you want to
insure an automatic clean up of previous value.

\item Remember to order the arguments in such a way that the predicate can
be declare as FF.

\item Sometime, Basic Events (\pxref{Basic Events}) and (\pxref{Basic Events?}) are enough to represent facts you want to get deleted
automatically.

\end{itemize}

Here are example of predicates which are often declared as FF:\\*
\code{position}, \code{status}, \code{connected}. For each of them one
usually want to remove the previous value when a new value for a specific
object is concluded.

\subsection{Which Predicate Should be Declared as Basic Events?}
\node{Basic Events?}
\cpindex{Which Predicate Should be Declared as Basic Events?}

This is another easy question to answer. You want to declare as Basic
Events (\pxref{Basic Events}), all the predicate you want to trigger OP
applicability or intentions/threads awaking, but that you do not want to
keep in the database. This is to represent events which are ephemeral,
still need to be considered by the kernel main loop, but do not need to
be remembered in the database for future use.

\subsection{Forbidden Things and Things to Avoid with the Database}
\node{Forbidden Things and Things to Avoid with the Database}
\cpindex{Forbidden Things and Things to Avoid with the Database}

There are a number of things one should not do or avoid while using the
database mechanism. Here is a non exhaustive list of forbidden things:

\begin{itemize}

\item Do not conclude facts containing unbound variables.

\item Do not consult evaluable predicate with unbound variable (unless
you predicate is able to handle the situation, which is unlikely...)

\end{itemize}

Here is a non exhaustive list of things to avoid as much as possible:

\begin{itemize}

\item Avoid consulting big disjunction, which can lead to a huge number
of possible results. If you must use disjunction, try to put the most
discriminating facts in first.

\item Avoid using long argument list for facts.

\item Remember to tune the size of the hashtable according to the number
of facts you are concluding in the database.

\item Do not conclude evaluable predicates.

\item Do not conclude facts containing variables bound to objects which
are, by their nature, pointing to internal objects, themselves pointing
to other internal objects. This will not break the system, but this will
result in ever growing kernel (as the Garbage Collector will be unable to
collect the internal data concluded in the database).

\item Do not consult negation of Closed World predicate with unbound
variable, as you will not get binding even if the negation is true.

\item Do not forget to clean up your database. Many application grows
indefinitely because of a bad database clean up.

\end{itemize}

\section{Which OP for Which Task?}
\node{Which OP for Which Task?}
\cpindex{Which OP for Which Task?}



\subsection{Fact Invoked OPs}
\node{Fact Invoked OPs}
\cpindex{Fact Invoked OPs}

Fact invoked OPs are usually written to respond or react to events. They do not
have any explicit goal or objective.

These OPs usually correspond to events you have to monitor or you need to
check, such as alarm, or change of values, etc.

\subsection{Goal Invoked OPs}
\node{Goal Invoked OPs}
\cpindex{Goal Invoked OPs}

Goal invoked OPs are written to achieve a particular goal or objective.
Their success is equivalent to achieving the goal which invoked them.

These OPs are usually implementing a goal directed behavior, i.e. the execution
of these OP is supposed to achieve the goal for which they are applicable.

\subsection{Predefined OPs}
\node{Predefined OPs}
\cpindex{Predefined OPs}

A number of predefined OPs can be loaded in a specific application. The user
can pick up OPs in the different OP files provided with the distribution
(\pxref{Default OPs}). For example, the OPs in the file \file{new-default.opf}
(\pxref{new-default.opf}) provide a number of interesting functionalities. In
fact, most application load this OP file.  Even if it is not required, it is
strongly recommended, as OP as basic as \code{=} are defined in this file.

Most of the default OPs defined in  \file{new-default.opf} are actions OPs. The
actions they called are usually documented and can be directly used by the
user.

\section{User Defined Evaluable Functions}
\node{User Defined Evaluable Functions}
\cpindex{User Defined Evaluable Functions}

Do you need to defined your own evaluable functions? This depends of your
application. Do not look after them, the need will come in its time.

You may need evaluable functions for one the following reasons:

\begin{itemize}

\item to perform some specific computation on some Terms. For example,
you may have to define the \code{max} function which taking to integers
return the biggest of the two. Similarly, you may want to defined a
function which concatenate two stings, etc... That's up to you.

\item to perform some specific computation on some user defined objects
(\code{U\_POINTER} Term). For example, you may have defined your own user type, a
fancy C structure which hold various information required by your
application, and you want now to write a function which return the string
contained in a specific field of this object.

\end{itemize}

In any case, the need for evaluable function is easy to identify, and
easy to solve. Keep in mind though, that evaluable function are called
whenever they are encountered while posting a goal containing a composed
term which function name is an evaluable function. As a consequence, the
time taken by these function should be as short as possible. Contrary to
Actions, the execution of evaluable functions cannot be time sliced.

Last, there are a number of things to consider when you write evaluable
functions:

\begin{itemize}

\item Check the number of arguments which is passed to your function.

\item Check the type of each argument you are getting. (do not assume
that the user will always pass the argument of the right type at the
right position).

\item Check that all the arguments are bound. This relates to the previous
comment as you will get a Term of type \code{VARIABLE} if a variable is
unbound...

\end{itemize}

\xref{How to Define  your Own Evaluable Functions}, for more information
on this subject.

\section{User Defined Evaluable Predicates}
\node{User Defined Evaluable Predicates}
\cpindex{User Defined Evaluable Predicates}

As evaluable functions, evaluable predicates may be required in some
application:

\begin{itemize}

\item to perform some specific computation on some Terms. For example,
you may have to define the \code{ordered} predicate which taking a list
of integers return \code{TRUE} if the list is ordered, \code{FALSE}
otherwise.

\item to perform some specific computation on some user defined objects
(\code{U\_POINTER} Term). For example, you may have defined your own user type, a
fancy C structure which hold various information required by your
application, and you need a predicate which evaluate to \code{TRUE} if a
this structure satisfy a particular condition.

\end{itemize}

Last, there are a number of things to consider when you write evaluable
predicates:

\begin{itemize}

\item Remember that you just need to return a \code{PBoolean}, i.e.
\code{TRUE} or \code{FALSE}, not a Term.

\item Check the number of arguments which is passed to your function.

\item Check the type of each argument you are getting. (do not assume
that the user will always pass the argument of the right type at the
right position).

\item Check that all the arguments are bound. This relates to the previous
comment as you will get a Term of type \code{VARIABLE} if a variable is
unbound...

\end{itemize}

\xref{How to Define  your Own Evaluable Predicates}, for more information
on this subject.

\section{User Defined Actions}
\node{User Defined Actions}
\cpindex{User Defined Actions}

Do you need to defined some actions? Here also, this depends of your
application. It usually depends of what kind of interaction or
``actions'' (thus their name) you need to perform with the kernels.
If your application only prints some statements (such as advices to an
operator), then you may not need to define any actions. On the opposite,
if your application need to directly interact or act with subsystems,
then you will need to define some action.

Keep in mind that if your application is organized in such a way that
there is module which perform the real action on the world, and that this
module is connected to the \MPA{}, then most if not all the actions are
done using the \MPA{} (i.e. by sending a message to the module in charge
of performing the action (a simulator or an interface to the real system).

Nevertheless, you may need actions for one the following reasons:

\begin{itemize}

\item to associate a specific functions to a OP. Thus enabling the action to
succeed or fail and make the corresponding OP returning the corresponding
result. Many users tend to forget that evaluable functions (as presented
above) are evaluated at posting time, i.e. when the goal in which their
form is contained is posted. But their evaluation is always meaningful and
does not bear any success or failure. For example, in \code{(! (foo (+ 3
4)))}, the   \code{(+ 3 4)} will be evaluated right away (unless you have
unselected the \code{eval\_on\_post} option). However, this evaluation is
always feasible and always gives a result. For an action (in fact a
standard action), the result can  be either \code{T} or \code{NIL}, and
upon this result depend the success or failure of the OP which calls it.

\item to perform some actions using code linked in your kernel. Keep in
mind that when you build your application, you will, or may link some
code in it. For example, if your application is linked (and by linked we
mean the operation of linking in one program a number of functions calling
each others) to an application dependent library and that you want to
call some of these functions which may fail, then you need to define
actions to call these functions.

\item to time slice the execution of functions which take to long to
execute. Keep in mind that the reactivity of your system depend of the
longest action/evaluable functions of your application. Therefore, to
increase the reactivity of your kernel, you may have to time slice the
execution of these actions.

\item to write functions which return a list of terms instead of one
term. \xref{Multi Variable Special Action} for more details.

\end{itemize}

Last, there are a number of things to consider when you write functions:

\begin{itemize}

\item Check the number of arguments which is passed to your function.

\item Check the type of each argument you are getting. (do not assume
that the user will always pass the argument of the right type at the
right position).

\item Check that all the arguments are bound. This relates to the previous
comment as you will get a Term of type \code{VARIABLE} if a variable is
unbound...

\item Make sure you return a \strong{new} Term, and this recursively...
Do not return for example one of the argument passed to the functions
``as is'', or a \code{LISP\_LIST} containing object which have not been created
by you.

\item Make sure your function return the Term containing either:

\begin{itemize}

\item The symbol \code{T}, the symbol \code{NIL} or the symbol
\code{:wait}, if it is a Standard Action. \xref{Standard Action} for more
details.

\item Any kind of term or the symbol \code{:wait}, if it is a Special
Action. \xref{Special Action} for more details.

\item A list (\code{OPRS\_LIST}) of any kind of term or the symbol \code{:wait}, if it
is a Multi Variable Special Action. The list should have exactly the same
number of element as the list of variable it is unify against.
\xref{Multi Variable Special Action} for more details.

\end{itemize}
\end{itemize}

\xref{How to Define  your Own Actions}, for more information
on this subject.

\section{Do You Need Meta Level?}
\node{Do You Need Meta Level?}
\cpindex{Do You Need Meta Level?}

This question is somewhat difficult to answer. The following points have to
be considered before making a decision.

\begin{itemize}

\item Do you need a particular type of control in the main loop?

\item Do you have to choose among various applicable OPs with a specific
heuristic?

\item Do you mind ``loosing'' or forgetting external events?

\end{itemize}

In any case, always use the right level of options in the kernel to suit
your meta level needs. For example, if your kernel does not use any meta
level reasoning, you should disable it in the kernel (\pxref{Meta Level Reasoning}. However, if it does, for example use the \code{SOAK} meta
fact, you should enable it and enable the conclusion of this particular
fact (and disable all the other one), see \nxref{OPRS Kernel Meta Level Option Commands}.

\section{Intention Graph Manipulation}
\node{Intention Graph Manipulation}
\cpindex{Intention Graph Manipulation}

% **************************************
% ********** To be completed ***********
% **************************************

\section{Data and Commands}
\node{Data and Commands}
\cpindex{Data and Commands}

If your \OPRS{} application is embedded in a ``real'' or simulated
environment, you will probably get some data from this environment and
you may send commands to it.

In most case, these data and these commands will transit through the
\MPA{}.

Example are provided in the \file{demo} directory of connection between
applications and \CPK{} through the \MPA{}.

However, you can also design your own communication interface, which will
interact with the external application through evaluable predicate, functions
or actions.

\section{Linking C Code in the Kernels}
\node{Linking C Code in the Kernels}
\cpindex{Linking C Code in the Kernels}

To be able to execute your own code in a \CPK{} or \XPK{}, you need to link it
to relocatable kernels.

Technically, a relocatable is an ``almost executable'' in which are missing a
number of functions. The \COPRSDE{} contains two relocatables,
\file{oprs-relocatable} and \file{xoprs-relocatable} (and their C++ counterpart,
\file{c++-oprs-relocatable} and \file{c++-xoprs-relocatable}) . These files are
almost like \file{oprs} and \file{xoprs} except that a number of functions are
missing.

Apart from some system libraries needed to build the final executable,  the
missing functions are:
\begin{verbatim}
declare_user_eval_funct 
declare_user_eval_pred 
declare_user_action 
start_kernel_user_hook 
end_kernel_user_hook 
\end{verbatim}

For the C++ version, the main is missing (so the user can have a C++ main) and
is called oprs\_main (and takes the same argument as main).

Somehow, you need to define these functions. Example of these functions are
given in the \file{pub\_src} directory, in the files: \file{user-action.c}, 
\file{user-action.h}, \file{user-ev-function.c}, \file{user-ev-function.h},
\file{user-ev-predicate.c}, \file{user-ev-predicate.h}, \file{user-external.c},
\file{user-external.h} and \file{user-external\_f.h}.

When  these files are defined, you need to write a Makefile more or less like
the following one:
\begin{verbatim}
OPRS_DIR =

include ${OPRS_DIR}/site.make

OPRS_INCL_DIR = ${OPRS_DIR}/include

USER_SRC = user-evaluable-predicate.c user-action.c user-evaluable-function.c user-external.c

USER_OBJ = user-evaluable-predicate.o user-action.o user-evaluable-function.o user-external.o

SRCS = $(USER_SRC)

INCLUDE =  -I$(OPRS_INCL_DIR)

CFLAGS = $(ANSI) $(DEBUG_FLAG) $(OPTIMIZE) $(WARNINGS) $(MACHINE) 
        $(CFLAGS1) $(INCLUDE) $(X_INCLUDE) $(PROFILE)

LDFLAGS = $(ANSI) $(STATIC) $(DEBUG_FLAG) $(OPTIMIZE) $(PROFILE)

SYS_LIB = -lm

all: oprs xoprs

oprs: $(OPRS_DIR)/oprs-relocatable $(USER_OBJ)
        $(CC) -o oprs $(USER_OBJ) oprs-relocatable 

xoprs:  $(OPRS_DIR)/xoprs-relocatable $(USER_OBJ) $(LIB_UTIL)
        $(CC) -o xoprs xoprs-relocatable $(USER_OBJ) $(SYS_LIB) $(X_LIB)
\end{verbatim}

\section{Miscellaneous Questions}
\node{Miscellaneous Questions}
\cpindex{Miscellaneous Questions}

% **************************************
% ********** To be completed ***********
% **************************************

\section{Common Mistakes}
\node{Common Mistakes}
\cpindex{Common Mistakes}

There are a number of mistakes which are commonly made. \xref{Known Problems and Things to Avoid} for a list of potential pitfalls. Nevertheless, here is
a list of common mistakes.

\begin{itemize}

\item Do not write OPs with evaluable predicate in their invocation part.
They will not become applicable, because of this predicate becoming true.

\item If you write a goal such as \code{(! (\& (foo a \$x) (boo b \$y)))},
you may achieve it if this conjunction is true in the database, but if it
is not, there are absolutely no way this specific goal may be achieved
through a OP. Because only texpressions are allowed in Invocation Part as
element of their gmexpression.

\item When you declare a functional fact, do not forget the integer argument of
the declaration.

\item Before launching a \XPK{} or \CPK{}, make sure a \OPRSS{} and a \MPA{}
are running.

\end{itemize}

\chapter{Simple \COPRS{} Applications}
\node{Simple OPRS Applications}
\cpindex{Simple \COPRS{} Applications}



\section{Factorial Example}
\node{Factorial Example}
\cpindex{Factorial Example}

This fairly simple example introduces many different interesting concepts
of \COPRS. It uses logical and program variables, it introduces meta level
reasoning and also presents some advanced features such as parallel
goals intentions. This example can be found in \file{data/fact-meta.inc}
and related files.



\subsection{Factorial Example OPs}
\node{Factorial Example OPs}
\cpindex{Factorial Example OPs}

Here is the list of the procedures provided for this demo. They can be found in
the file \file{fact-meta.opf}.

\input{fact-meta}

\subsection{Other Factorial Example OPs}
\node{Other Factorial Example OPs}
\cpindex{Other Factorial Example OPs}

In this example, we use the new ``if-then-else'' node as well as the parallel
thread execution mechanism. These OPs can be found in the file
\file{fact-meta-if-par.opf}.

\input{fact-meta-if-par}

\chapter{Complex \COPRS{} Applications}
\node{Complex OPRS Applications}
\cpindex{Complex \COPRS{} Applications}




\section{Truck Loading Example}
\node{Truck Loading Example}
\cpindex{Truck Loading Example}

This example is a real application, with a simulator and a user interface.



\subsection{Truck Loading Example Presentation}
\node{Truck Loading Example Presentation}
\cpindex{Truck Loading Example Presentation}

\figuregif[scale=0.5]{truck-demo-window}{Truck Loading Demo}{Truck Loading Demo}

This example presents a supervision and control problem with various complex
temporal constructions. As described on figure \ref{Truck Loading Demo}, an
operator is in charge of a refilling station. The process to supervise is
basically the following.  From time to time, tank trucks come to the station.
They are queued until a traffic light (Queue Light) turns green. As soon as
they are in place, a ``truck in place'' signal is sent to the operator who has
to turn the Queue Light to red again and waits until the truck is ready to be
filled (Filling Talkback shows Empty). He then opens the valve controlling the
product flow.  This valve is opened and closed using a two position switch on
the control panel (Switch Valve). To monitor the current position of the valve,
two talkbacks with different sensors are used (Talkback Sensor1 and Talkback
Sensor2). These talkbacks can display the following information: closed, open,
and barber pole.  Barber pole is displayed by a talkback whenever the valve is
neither open nor closed but in between (most likely it is moving in position).
Whenever the operator is filling up a truck and a ``truck full'' signal is
received (the Filling Talkback shows Full), he immediately closes the valve.
When the valve is considered as closed, he gives an ``OK to go'' to the truck
driver, by turning the Filling Light to Green. This process goes on for ever
during the whole day.

In the plan executed by the operator, different things can go wrong. The most
important failure is the valve not closing properly; a slightly less important
problem is the valve not opening properly. If the valve reportedly fails to
open or close and, if there is a serious doubt about its real position, the
operator must activate the emergency shutdown (the Stop Button), which is
considered as a ``last resort'' solution.

The inputs (and their respective values) of this control and supervision
process are displayed in the Control Board (the blue panel on the top).

The actions the operator can take are available from the Command board (the
bottom blue panel).

The operator does not see the bottom panel of the demo window which displays
the real world (trucks showing up, filling up, etc). He can only rely on the
Control board to analyze the current situation, and on the Command board to
control the operations.

The three text windows in the upper part of the truck demo window display
information from various sources. Message: normal operation of the
demonstration. Warning: abnormal states in the simulation. OPRS: message sent
and received from the controlling \CPK{}.

The six buttons below the text panes are used to (from left to right) quit the
demo, reset the demo, clear the text panes, disconnect from the \CPK{}, connect
to the \MPA{}, and connect to \CPK{}.

The clock panel displays the current simulation time (from the simulator when
alone, or from \CPK{} when connected). It is also used to start or stop the
simulation time. There are also buttons to speed up or slow down the simulation
in non connected mode.

The Error board on the right can be used to modify a number of internal
parameters of the demo. It is mainly used to create dysfunction.

In general, the operator should monitor the following conditions and take
various actions in response to these conditions:

\begin{itemize}

\item Talkback 1 and Talkback 2 disagree (one says OPEN, the other says
CLOSE). If this condition arises, the talkback which disagrees with the switch
is considered as failed.

\item Talkback 1 and Talkback 2 show a barber pole for more than x seconds
(i.e., there are at least x seconds during which both talkbacks show
barber pole). The valve is jammed, emergency stop.

\item Talkback 1 or Talkback 2 show a barber pole for more than y seconds.
If this condition arises, then the faulty Talkback is concluded as failed.

\item z seconds after flipping the switch in one position or the other,
the valve is not reported in the right position by both talkbacks. This
interval is decreased down to w seconds if one of the talkback is failed. In
this case, an emergency shutdown is issued.

\end{itemize}

\subsection{How to Install the Truck Loading Demo}
\node{How to Install the Truck Loading Demo}
\cpindex{How to Install the Truck Loading Demo}

The \file{demo/truck-demo} directory should be placed or extracted in the \file{oprs}
directory. Then just type \\*
\code{\% make truck-demo}.

If everything goes well, you should end up with a \code{truck-demo} executable
in the \file{demo/truck-demo/bin} sub directory, as well as a \code{oprs} and
\code{xoprs} kernels in the \file{demo/truck-demo/oprs} sub directory. Note that
you must use these kernels as they define their own evaluable functions and
predicates (which are different from the one defined by default in the kernel).

This application is a good example of how to define your own actions, evaluable
functions, predicates, etc...

\subsection{How to Run the Truck Loading Demo}
\node{How to Run the Truck Loading Demo}
\cpindex{How to Run the Truck Loading Demo}

You first need to start the truck-demo. To do so, just execute the
\code{truck-demo} program produced by the make command.

At this point, you can play with the simulator as if you were the operator. You
are in ``Alone'' mode.

If you want to get \COPRS{} to manage the ``filling station'', start a oprs-server,
and then start a \CPK{} or \XPK{} with the following options (you must use the
kernels from the \file{demo/truck-demo/oprs} directory as it defines some
specific evaluable functions or predicates): \\*
\code{\% xoprs -n truck -x data/truck-demo.inc}

You then need to issue an \code{accept} command in the \OPRSS{}.

Go back in the truck-demo window, and click on "Connect to the MP" button, wait
until the connection completes (the button is highlighted). Click on the
"Connect to OPRS" button. If you have been playing with the demo before
connecting to the \CPK{}, it is necessary to reset the simulation (by clicking
on the Reset button) before connecting to OPRS.

From this point, the \CPK{} will control and supervise the filling truck
process. You will see the lights and the valve switching upon \OPRS{} request,
as well as \OPRS{} reacting to transients and to real problems.

Feel free to select the \XPK{} window and trace the OP used for this demo.

A priori, \CPK{} takes care of the filling station ``ad vitam eternam''. If you
want to produce an error (to check if \COPRS will find it), you can play with
the various settings of the Error board.

You can also improve the demo by adding OPs which will check other faulty or
error conditions. For instance, the \file{demo/truck-demo/oprs/data/truck-demo-plus.opf}
contains a OP which checks that you are not switching a traffic light on (to
green) while the valve is open. (Note that this OP is not loaded by default but
can be added with the Load OP command from the \XPK{}).

\subsection{Truck Loading Example OPs}
\node{Truck Loading Example OPs}
\cpindex{Truck Loading Example OPs}

Here is a list of the procedures provided for this demo. They can be found in
the file \file{truck-demo.opf}.

\input{truck-demo}

One can also find an extra OP, in the file \file{truck-demo-plus.opf}.

\input{truck-demo-plus}

\chapter{Applications of \COPRS{}}
\node{Applications of OPRS}
\cpindex{Applications of \COPRS{}}

If there exist a priori no exclusive domain of application to Procedural
Reasoning, some seem more adapted than other. They can be identified by
the following characteristics, which are illustrated with different
applications where Procedural Reasoning was applied:

\begin{description}

\item [Control \& Supervision of Complex Systems:] In this kind of
application, one or several operators are in charge of executing and
following, under some well defined conditions, the procedures the system
designers  established. These procedures cover the nominal mode of the
system, as well as emergency and critical situation like alarms, or
threshold overshooting. (Example: Telecommunication Network
Supervision).

\item [Operator Assistance:] For various, reasons, replacing operators of
complex systems with autonomous systems is not always recommended.
However, operators tend to make mistakes, with potentially dramatic
consequences, whenever they are overflowed with storms of alarms and
data. For this very reason, it is very important to assist them in the
most tedious tasks, or in those requiring quick  responses, like in power
plant supervision, or in threat assessment systems.

\item [Automation of predefined procedures execution:] In many domains,
complete folders of procedures have been written to describe all the
reachable states of the system (planes, space shuttle). Here also,
operators assure the triggering, retrieval, and execution of these
procedures.

\item [On-line planning and execution control:] These activities combine
the planning of task and actions, and the control of their execution. By
planning, we mean that the system chooses at run-time a solution among a
set of possibilities, and this according to some general criteria.
Execution control checks that the chosen plan or procedure is executing
properly (example: execution control of a mobile robot, pilot
associate).

\item [Diagnosis \& Troubleshooting:] The decision trees used in diagnosis
activities show a procedural structure. Thus, troubleshooting is usually
done by following the procedures which, test after test, determine the
system faulty component (examples: troubleshooting of radar system). Note
that these tests can be arbitrary complex and require the execution of
other procedures.

\item [Operator Training:] Given its graphical capabilities, procedural
reasoning can also be used for operator training. As a matter of fact,
problems can be simulated for the operator to follow the procedures the
system is currently running (example: operator training in the space
shuttle mission control room).

\end{description}

In every domain presenting these characteristics or properties,
procedural reasoning is the solution that improves system safety, speed
up response time and reduces operating costs.

% **************************************
% ********** To be completed ***********
% **************************************

\chapter{Optimizing a \COPRS{} Applications}
\node{Optimizing a OPRS Applications}
\cpindex{Optimizing a \COPRS{} Applications}



\section{Optimizing Hashtables}
\node{Optimizing Hashtables}
\cpindex{Optimizing Hashtables}

A number of commands (\pxref{OPRS Kernel Miscellaneous Commands}) or menus
(\pxref{Stat All Hashtables}, and others) are available to obtain statistics on
the use of the various hashtables of the kernel.

When you select these commands, you get the following printout (this one
was obtained with a \code{stat all} command):

\begin{verbatim}
 The id hashtable contains:
     442 element(s)
     in  355 buckets ( 1024 );
     with a maximal number of 4 element(s) in one bucket.
The database hashtable contains:
     48 element(s)
     in  48 buckets ( 1024 );
     with a maximal number of 1 element(s) in one bucket.
The predicate hashtable contains:
     107 element(s)
     in  51 buckets ( 64 );
     with a maximal number of 5 element(s) in one bucket.
The function hashtable contains:
     146 element(s)
     in  57 buckets ( 64 );
     with a maximal number of 6 element(s) in one bucket.
\end{verbatim}

You can then analyze the result and decide if a specific hashtables is
overloaded or not. You may then decide to change its size (reduce it or
increase it) using the argument of the \code{oprs} or \code{xoprs} command
(which are the same for this matter) (\pxref{Arguments to the oprs Command}).

Note that these statistic are ``static'' in the sense that they give you
snapshots of the use of your hashtables. However, most of them (except the
database and id hashtable) are loaded upon starting the kernel and do not
evolve much during the execution. Nevertheless, for the database hashtable, you
should stat it at different time of the execution to figure out its maximum
size and conclude then which size is the most appropriate. As for the id
hashtable, or the symbol hashtable, keep in mind that this hashtable cannot be
cleared by any mean. As a result, if you keep adding new symbols in the kernel,
then this will be an ever growing hashtable.

Last, you may consider reducing the size of the hashtables (after all ,
the default values may far exceed your need). This will have the
advantage of reducing the size of the kernel, by the amount of memory
saved.

Keep in mind that since release 1.1, all these hashtable are mostly used at
compile time. The id, the function and the predicate hashtables are use at
runtime just to parse coming external events, and user commands (which explain
why they are kept).

\section{Just the Right Level of Meta Level}
\node{Just the Right Level of Meta Level}
\cpindex{Just the Right Level of Meta Level}

This is probably one of the easiest optimization one can do... and if it
has not been done, it can be one of the most profitable optimization.

You should always use the right level of options in the kernel to suit
your meta level needs. For example, if your kernel does not use any meta
level reasoning, you should disable it in the kernel (\pxref{Meta Level Reasoning}) with a \code{set meta off} command or with the option menu of
the \XPK{}. However, if it does, for example use the \code{SOAK} meta
fact, you should enable it and enable the conclusion of this particular
fact (and disable all the other one).

\section{Database Organization}
\node{Database Organization}
\cpindex{Database Organization}

Database optimization can be of different type. First you should optimize
the size of its hashtable (not too big, not too small), see
\nxref{Optimizing Hashtables}.

\section{Slicing your Action}
\node{Slicing your Action}
\cpindex{Slicing your Action}

One of the important point to consider when optimizing an application is
the time taken by actions. If their execution takes too long time, you should
consider recoding them to slice their execution in smaller time chunk. This can
be done using the action slicing mechanism (\pxref{Action Slicing}).

See \code{action\_first\_call} and \code{action\_number\_called},
\nxref{Intention Manipulation Functions}. 


%%% Local Variables: 
%%% mode: latex
%%% TeX-master: "oprs"
%%% End: 

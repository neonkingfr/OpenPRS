%%%%%%%%%%%%%%%%%%%%%%%%%%%% -*- Mode: latex -*- %%%%%%%%%%%%%%%%%%%%%%%%%%
%%
%% Copyright (c) 1991-2003 Francois Felix Ingrand.
%% All rights reserved.
%%
%% Redistribution and use in source and binary forms, with or without
%% modification, are permitted provided that the following conditions
%% are met:
%%
%%    - Redistributions of source code must retain the above copyright
%%      notice, this list of conditions and the following disclaimer.
%%    - Redistributions in binary form must reproduce the above
%%      copyright notice, this list of conditions and the following
%%      disclaimer in the documentation and/or other materials provided
%%      with the distribution.
%%
%% THIS SOFTWARE IS PROVIDED BY THE COPYRIGHT HOLDERS AND CONTRIBUTORS
%% "AS IS" AND ANY EXPRESS OR IMPLIED WARRANTIES, INCLUDING, BUT NOT
%% LIMITED TO, THE IMPLIED WARRANTIES OF MERCHANTABILITY AND FITNESS
%% FOR A PARTICULAR PURPOSE ARE DISCLAIMED. IN NO EVENT SHALL THE
%% COPYRIGHT HOLDERS OR CONTRIBUTORS BE LIABLE FOR ANY DIRECT, INDIRECT,
%% INCIDENTAL, SPECIAL, EXEMPLARY, OR CONSEQUENTIAL DAMAGES (INCLUDING,
%% BUT NOT LIMITED TO, PROCUREMENT OF SUBSTITUTE GOODS OR SERVICES;
%% LOSS OF USE, DATA, OR PROFITS; OR BUSINESS INTERRUPTION) HOWEVER
%% CAUSED AND ON ANY THEORY OF LIABILITY, WHETHER IN CONTRACT, STRICT
%% LIABILITY, OR TORT (INCLUDING NEGLIGENCE OR OTHERWISE) ARISING IN
%% ANY WAY OUT OF THE USE OF THIS SOFTWARE, EVEN IF ADVISED OF THE
%% POSSIBILITY OF SUCH DAMAGE.
%%
%% $Id$
%%
%%%%%%%%%%%%%%%%%%%%%%%%%%%%%%%%%%%%%%%%%%%%%%%%%%%%%%%%%%%%%%%%%%%%%%%%%%%%%%%

\part{\OPE{}}
\node{OP Editor}



\chapter*{Overview of the \OPE{}}
\node{Overview of the OP Editor}
\cpindex{Overview of the \OPE{}}

The \OPE{} is the graphical editing tool of the \COPRS{} development
environment. It is not included in the \COPRS{} application environment,
because this product does not provide any tool to create OPs and
procedures. This tool uses X11/Motif as a graphical user interface.

The OP Editor is the graphical editor for  procedures/OPs. It enables the
user to create, edit and modify the procedures of a \COPRS{} application.

\begin{itemize}

\item It runs under the X11 window system with the Motif widgets toolkit.

\item Its embedded lexical and syntax checker ensure that the procedure
the user write  will be loaded in the \CPK{} or \XPK{}.

\item It is upward compatible with the  GRASPER II Graph format , and
Sun Graph format.

\item It allows the user to edit more than one procedure and more than
one procedure file at the same time.

\item It provides an on-line help and documentation using the your prefered
  HTML browser (Netscape).

\item It has a friendly user interface.

\end{itemize}

The editor manipulates OPs and procedures which are stored in files. A
file is called a OP File and has different formats (\pxref{OP File Format}).

Each OP file can contain a certain number of OPs and procedures. When you
edit a
OP file, you can select the specific procedure you want to edit.

When selected, OPs and procedures are visible on the screen and can be
modified as desired by the user.

The OP files produced by the \OPE{} can then be loaded in a \CPK{} or an
\XPK{} to be executed.

\chapter{How to Use the \OPE{}}
\node{How to Use the OP Editor}
\cpindex{How to Use the \OPE{}}
\pgindex{op-editor command}

\figuregif[height=0.5\textheight]{ope-window}{\OPE{} Window}{OP Editor Window 2}

The \OPE{} is invoked from the Unix shell with the command:

\code{\% op-editor}



\section{Arguments of the \OPE{}}
\node{Arguments of the OP Editor}
\cpindex{Arguments of the \OPE{}}

Usage: \begin{verbatim}
op-editor [-D file-directory] [-F ACS-op-file] [-l upper|lower|none]
          [-m message-passer-hostname] [-j message-passer-port-number]
          [-c op-file*] [-pwt] [-peo] [-L en|fr] [op-file]*
\end{verbatim}

The \code{op-editor} command also accept all the standard Xt arguments
(\pxref{Xt Command Line Arguments}). 

All the arguments are optional.

\begin{description}

\item[\code{-D}] to specify a directory from which you want the file to
be looked for. If you say \code{-D data}, it tries to load
subsequent files specified in the command line from the \code{data}
sub directory.

\item[\code{-F}] to specify a file to load in OPF format. This is
equivalent to the Load OP file command (\pxref{Load OP File}). The
\file{.opf} extension must be omitted as it will be added by the
argument parser.

\item[\code{-m}] to specify the hostname on which the \MP{} runs or will be
started. If the \OPE{} cannot connect to this hostname on the specified port
(even after trying to start the \MP{}), then the program exits with an error
message. This option only applies if your license agreement require connection
of the \OPE{} to the \MP{}.

\item[\code{-j}] to specify the port on which the \MP{} is expecting a
connection (or will be started if necessary). This option only applies if your
license agreement require connection of the \OPE{} to the \MP{}.

\item[\code{-c}] to convert OP file in the newest OP File format. A backup copy
of the file is done with the \file{.bak} suffix.

\item[\code{-pwt}] is used to print the \OPE{} widget tree (\pxref{OP Editor Motif Widgets Hierarchy}) . This can be useful if you do not have the
documentation at hand and still want to now the name or type of a particular
widget.

\item[\code{-peo}] (which stands for Print English Operator) can be used to
print the temporal operator in english instead of the single letter. It will
print \code{achieve} instead of \samp{!}, and \code{wait} instead of \samp{\^{}}
and so on.

\item[\code{-l upper|lower|none}] can be used to print and parse all the symbol
and id in upper case, lower case or in no particular case.

\item[\code{-L en|fr}] can be used to select the language of the interface
(French or English). Note that by default your kernel is in english. Note also
that for the applications with an X interface (i.e. \XPK{} and the \OPE{} the
choice of the \file{app-defaults} file will select the language
(\pxref{Xt/Motif Widgets Hierarchy and Resources}). In this case, selecting a
different value with the option will lead to a warning and to a mix of language
in the interface.

\item[\code{op-file}] to specify files to load in OPF format. This is
equivalent to the Load OP file command (\pxref{Load OP File}). The
complete name must be given.

\end{description}

If you specify \code{-G}, \code{-S} or \code{-F}, the respective file
extensions must be omitted.


\section{\OPE{} Environment Variables}
\node{OP Editor Environment Variables}
\cpindex{OP Editor Kernel Environment Variables}

There are a number of environment variables which can be used to customize the
\OPE{} or to define default arguments. Arguments passed using the command line
have precedence on those acquired from environment variables.

\begin{description}

\cpindex{OPRS\_DOC\_DIR}
\item[\code{OPRS\_DOC\_DIR}] is used  to specify the location of the online \COPRSDE{}
documentation. It is used by the \XPK{} and the \OPE{}.
Example:
\begin{verbatim}
setenv OPRS_DOC_DIR /usr/local/oprs/doc
\end{verbatim}

\cpindex{OPRS\_ID\_CASE}
\item[\code{OPRS\_ID\_CASE}] is used to specify if the program should upper case,
lower case or should not change the case of the parsed Id. This is equivalent
to the \code{-l} option. The possible values
are \code{lower}, \code{upper} or \code{none}:\*
Example:
\begin{verbatim}
setenv OPRS_ID_CASE none
\end{verbatim}

\end{description}

\section{Creating a OP}
\node{Creating a OP}
\cpindex{Creating a OP}

Creating a OP is a fairly easy task. In the \OPE{}, select the Create OP menu
item (\pxref{Create OP}) in the OP pull down menu. This pops up a large dialog
box (shown on figure \ref{cre-gra-op}) which can be filled with some of the
information needed to create a OP.  Do not worry if you mistype or forget
something. The \OPE{} checks the syntax, and you can change or edit the
different fields afterwards. Note that you can select the type of OP to create:
a Graph OP or an Action OP (this is not something you can change afterwards, so
you must choose now). According to your choice, you will get a OP with a start
node or (exclusive or) an action field. Click OK to create the OP. If any
syntax error is found, the \OPE{} tells you in which field it was located.

As soon as the create dialog box disappears, you see your OP on the screen,
or more precisely its squeleton, as it only contains the fields you
have filled up and the start node (or the action field). By selecting
the various operations of the working menu (\pxref{Working Menu Items}) and by following the instructions in the Footer help window
(\pxref{Footer and Dialog Box Help}), you can then create nodes or
edges, edit them, and so on.

\section{Editing an Existing OP}
\node{Editing an Existing OP}
\cpindex{Editing an Existing OP}

Editing an existing OP is also a fairly easy operation. In the \OPE{},
select the Select OP menu item (\pxref{Select OP}) in the OP pull down
menu. This pops up a selection list dialog box containing all the OPs
present in this OP file. You can then select the one you want to see in
the editing area.

\section{Scroll Bars}
\node{Scroll Bars}
\cpindex{Scroll Bars}

Scroll bars can be used at any time to change the view port of the drawing
area. The drawing area is virtually ``as big as needed'', so do not be afraid
to draw OPs as big as you want. When the \OPE{} starts, it ``creates'' a window
big enough to contain the biggest OP you loaded.  However, you can increase the
maximum size of the window by selecting the Change Drawing Size (\pxref{Change Drawing Size}) menu item in the Misc Menu.

\section{Selection Pane}
\node{Selection Pane}
\cpindex{Selection Pane}

\figuregif[scale=0.5]{ope-sel-pan}{Selection Pane}{selection-pane}

This pane (Figure \ref{selection-pane}) is located just below the menu bar and
gives the following information to the user: the file name which is currently
edited, the OP name which is currently selected, and a marker indicating
whether the file has been modified or not.

\section{Footer and Dialog Box Help}
\node{Footer and Dialog Box Help}
\cpindex{Footer and Dialog Box Help}

\figuregif[scale=0.5]{ope-help-pan}{Footer Help Pane}{help-pane}

At any time, the user can get information and valuable help on what he is
expected to do by looking at the footer window of the \OPE{} (Figure
\ref{help-pane}). Moreover, most dialog boxes have ``HELP'' button pointing at
the proper section of the on line manual (the present manual), which can be
browsed with your HTML browser.

\section{Pretty Printing}
\node{Pretty Printing}
\cpindex{Pretty Printing}

All text objects displayed by the \OPE{} (but also the \XPK{}) are pretty
printed. Pretty printing is usually appreciated when one has complex
expressions to edit, it makes the editing process easier and faster.
Moreover, the user can specify on which width the pretty printer
should try to print the object, as well as, if it should (when possible)
fill up the lines (\pxref{Edit Object}). This information (the width and
the line filling) is kept in the OPF format for each fields of each OP.

\chapter{\OPE{} Commands}
\node{OP Editor Commands}
\cpindex{OP Editor Commands}

The commands of the OP-editor are grouped into two different sets of menu.  The
first set is a menu bar with cascade menus. The second set is a group of push
buttons on the left side of the drawing pane. It is called the working menu, as
it contains the commands which are most often used to edit OPs.



\section{Menubar of the \OPE{}}
\node{Menubar of the OP Editor}
\cpindex{Menubar of the \OPE{}}

\figuregif[scale=0.5]{ope-menu-bar}{OP Editor Menu Bar}{ope-menu-bar}

The Menu Bar (Figure \ref{ope-menu-bar}) contains different buttons from
which cascade menus pop when selected.



\subsection{File Menu of the \OPE{}}
\node{File Menu of the OP Editor}
\cpindex{File Menu of the \OPE{}}

\figuregif[scale=0.5]{ope-file-menu}{OP Editor File Menu}{ope-file-menu}

The file menu (Figure \ref{ope-file-menu}) contains all the operations dealing
with files, i.e. operations to load OP files in different formats, to save or
write them, etc.



\subsubsection{Load OP File}
\node{Load OP File}
\cpindex{Load OP File}

\figuregif[scale=0.5]{load-op}{Load OP File Selection Box}{load-op-file}

This command is used to load a OP File, in OPF format (\pxref{OPF Format}).
The default (and recommended) extension for this file is \file{.opf} .  When
this command is selected, a file selection dialog box appears to enable the
user to select the file to load (see Figure \ref{load-op-file}) .

If the load command is successful, the Select OP dialog box pops up,
and the user is asked to select a OP to display.

\subsubsection{Append OP File}
\node{Append OP File}
\cpindex{Append OP File}

This command is used to append a OP File, in OPF format (\pxref{OPF Format}),
to the currently selected OP file. When this command is selected, a file
selection dialog box appears to enable the user to select the file to append.

If the append command is successful, the Select OP dialog box pops up,
and the user is asked to select a OP to display.

\subsubsection{New OP File}
\node{New OP File}
\cpindex{New OP File}

This command can be used to create a new OP file. By default, the name of
this new file is \file{Untitled} with an increasing suffix number. The
name of the file is chosen and defined the first time you save
it with the save or write command.

\subsubsection{Load Grasper OP File}
\node{Load Grasper OP File}
\cpindex{Load Grasper OP File}

This command is provided for upward compatibility with SRI Lisp OPRS.

This command is used to load a OP File in Grasper Graph format
(\pxref{GGRAPH Format}). The default (and recommended) extension for these
files is \file{.ggraph} . When this command is selected, a File Selection
dialog box appears to allow the user to select the file to load.

\subsubsection{Load Sun OP File}
\node{Load Sun OP File}
\cpindex{Load Sun OP File}

This command is provided for upward compatibility with SRI Lisp OPRS.

This command is used to load a OP File in Sun Graph format
(\pxref{SGRAPH Format}). The default (and recommended) extension for this
file is \file{.sgraph}. When this command is selected, a File Selection
dialog box appears to allow the user to select the file to load. Note that
nothing appears on the screen as the result of this command. You first need
to save the file (in OPF format), and then reload the saved file.

\subsubsection{Save OP File}
\node{Save OP File}
\cpindex{Save OP File}

The OP Editor can only save in OPF File format (\pxref{OPF Format}). By
default, the save command saves the currently selected OP File. If this
OP file has no real filename (if it has been created with the New OP File
command), then a Selection File dialog box pops up for you to choose the
filename. The default and recommended extension is \file{.opf}.

If the currently selected OP file is not in the OPF File Format, an error
dialog box pops up to advise you to use the write command instead.

\subsubsection{Write OP File}
\node{Write OP File}
\cpindex{Write OP File}

The write command can be used to write the currently selected OP File in OPF
format in the filename specified in the Selection File dialog box.

\subsubsection{Write TeX Doc File}
\node{Write TeX Doc File}
\cpindex{Write TeX Doc File}

The write tex doc file command can be used to write a \TeX{}/\LaTeX{} format documentation
of the selected file in the filename specified in the Selection File dialog
box. This documentation can the be included in \TeX{}/\LaTeX{} document.

\subsubsection{Quit}
\node{Quit}
\cpindex{Quit}

This command is used when you want to quit the \OPEditor. If some files
have not been saved, you are asked to confirm you want to quit
without saving them.

\subsection{Edit Menu of the \OPE{}}
\node{Edit Menu of the OP Editor}
\cpindex{Edit Menu of the \OPE{}}

\figuregif[scale=0.5]{ope-edit-menu}{OP Editor Edit Menu}{ope-edit-menu}

This menu (Figure \ref{ope-edit-menu}) contains the commands to deal with
loaded OP Files. 



\subsubsection{Select OP File}
\node{Select OP File}
\cpindex{Select OP File}

This command allows you to change the currently selected file. A list of
currently loaded or known OP Files is presented in a Selection Dialog Box. If
the selection is successful, a Select OP dialog box pops up for you to choose
the OP to display.

\subsubsection{Print OP File}
\node{Print OP File}
\cpindex{Print OP File}

This command applies the ``print'' command (\pxref{OP Editor Resources}, and
\pxref{Change Print Command}) to all the OPs of the current OP
file. Beware, this can be a rather long process.

\subsubsection{Unload OP File}
\node{Unload OP File}
\cpindex{Unload OP File}

This command allows the user to unload a OP file. As a result, the file
will not appear in the Select OP file menu any longer.

\subsection{OP Menu}
\node{OP Menu}
\cpindex{OP Menu}

\figuregif[scale=0.5]{ope-op-menu}{OP Editor Op Menu}{ope-op-menu}

This menu (Figure \ref{ope-op-menu}) contains all the commands which deal with OPs and, most often,
with the Selected OP.



\subsubsection{Select OP}
\node{Select OP}
\cpindex{Select OP}

This command allows the user to change the Selected OP. Keep in mind that
the name of the Selected OP is visible at any time in the information
pane just below the Menu Bar.

\subsubsection{Create OP}
\node{Create OP}
\cpindex{Create OP}

\figuregif[scale=0.5]{ope-cre-op-gra}{Create OP Dialog Box (Graph OP)}{cre-gra-op}

\figuregif[scale=0.5]{ope-op-gra}{Resulting Graph OP}{gra-op}

This command allows the user to create a new OP. A big dialog box pops up on
the screen (see figure \ref{cre-gra-op}), and you are asked to fill up the
various fields of the OP you want to create. Note that only two fields are
really required: the name and the invocation part. The resulting OP is shown
on Figure \ref{gra-op}. 

\figuregif[scale=0.5]{ope-cre-op-act}{Create OP Dialog Box (Action OP)}{cre-act-op}

\figuregif[scale=0.5]{ope-op-act}{Resulting Action OP}{act-op}

If you click on the Action toggle button, the Dialog Box will change as shown
on Figure \ref{cre-act-op}). An Action field is added, and you can enter the
action part of the OP. The resulting OP is shown on Figure \ref{act-op}.

\figuregif[scale=0.5]{ope-cre-op-txt}{Create OP Dialog Box (Text OP)}{cre-txt-op}

\figuregif[scale=0.5]{ope-op-txt}{Resulting Text OP}{txt-op}

If you click on the Text toggle button, the Dialog Box will change as shown
on Figure \ref{cre-txt-op}). A Body field is added, and you can enter the
body part of the OP. The resulting OP is shown on Figure \ref{txt-op}.

\subsubsection{Duplicate OP}
\node{Duplicate OP}
\cpindex{Duplicate OP}

This command allows the user to duplicate the currently Selected OP. A
prompt dialog box pops up for you to give the name of the duplicate
OP.

\subsubsection{Rename OP}
\node{Rename OP}
\cpindex{Rename OP}

This command allows the user to rename the currently Selected OP. A
prompt dialog box pops up for you to give the new name of the selected
OP. The text field is initialized with the previous name.

\subsubsection{Print OP}
\node{Print OP}
\cpindex{Print OP}

Will print the current OP using the print command (\pxref{Change Print Command}). This is done by dumping a bitmap of the OP in a temporary file, and
then by processing the resulting file with the print command. The temporary
file is then deleted. This OP is dumped in \code{xwd} format. It used to be in
\code{xpm} format, but the \code{xwd} format is much smaller on disk and
much faster
to produce.

\subsubsection{Dump OP Pixmap to File}
\node{Dump OP Pixmap to File}
\cpindex{Dump OP Pixmap to File}

This command allows the user to dump a bitmap or pixmap of the current OP.  A
file selection dialog box pops up to allow the user to choose in which file the
pixmap should be dumped. Note that the resulting file is \strong{HUGE}...

\subsubsection{Destroy Current OP}
\node{Destroy Current OP}
\cpindex{Destroy Current OP}

This command allows the user to destroy the current  OP.

\strong{Caution:} The current version of the OP Editor does not have an
Undo facility\dots{} So extreme care should be exercised when using this
command.

\subsubsection{Destroy A OP}
\node{Destroy A OP}
\cpindex{Destroy A OP}

This command allows the user to destroy a OP selected with a OP Selection
dialog box.

\strong{Caution:} The current version of the OP Editor does not have an
Undo facility\dots{} So extreme care should be exercised when using this
command.

\subsubsection{Previous OP}
\node{Previous OP}
\cpindex{Previous OP}

This command allows the user to quickly select the previous OP in the same
OPFile.

\subsubsection{Next OP}
\node{Next OP}
\cpindex{Next OP}

This command allows the user to quickly select the next OP in the same
OPFile.

\subsubsection{Toggle selected OPs}
\node{Toggle selected OPs}
\cpindex{Toggle selected OPs}

This command allows the user to quickly select the previously selected OP,
even if this OP is in another OPFile.

\subsubsection{Last Selected OPs}
\node{Last Selected OPs}
\cpindex{Last Selected OPs}

This command allows the user to quickly select a OP in the list of the
previously selected OPs, even if this OP is in another OP File.

\subsubsection{Copy OP in Buffer}
\node{Copy OP in Buffer}
\cpindex{Copy OP in Buffer}

This command allows you to copy a OP in an internal Copy/Paste buffer.
This OP can then be
retrieved or pasted in another OP File, using the Paste OP from Buffer
command.

\subsubsection{Paste OP from Buffer}
\node{Paste OP from Buffer}
\cpindex{Paste OP from Buffer}

This command enables the user to retrieve one or more OPs from the
Copy/Paste
internal buffer and paste it in the current OP file.

\subsubsection{Clear OP Buffer}
\node{Clear OP Buffer}
\cpindex{Clear OP Buffer}

Clear all the OPs in the Copy/Paste buffer.

\subsection{Misc Menu}
\node{Misc Menu}
\cpindex{Misc Menu}

\figuregif[scale=0.5]{ope-misc-menu}{OP Editor Misc Menu}{ope-misc-menu}

This menu (Figure \ref{ope-misc-menu}) contains commands to customize the \OPE{} or the appearance of
the OP on the screen.



\subsubsection{Change Print Command}
\node{Change Print Command}
\cpindex{Change Print Command}

Allow the user to change the print command. The default value is: \*
\code{xwdtopnm < \%s | pnmtops -r | lpr}.
The default value is the one defined in the \file{Op-editor.ad} resources
file (\pxref{OP Editor Resources}). It can be modified by the user (by
setting his own resource).

In any case, you must specify a command which contains a \code{\%s} to specify
the argument to which it will apply (the temporary file where the OP has been
dumped). Moreover, this command must be able to send a \code{xwd} (it used to
be in \code{xpm} format) to the printer.  We strongly advise the user to get
the pbmplus pacopge by Jef Posopnzer to print the result \cite{pbm-manual}.

The default command converts from xwd to pnm, then
transforms it in encapsulated postscript with
run length encoding and finally pipes it in the printer...

Note that you can use this command to dump OPs in Encapsulated Postscript
format (to include them in user manuals for example) by redirecting the last
command in a file instead of piping it in \code{lpr}.

\subsubsection{Change Drawing Size}
\node{Change Drawing Size}
\cpindex{Change Drawing Size}

\figuregif[scale=0.5]{ope-draw-size}{Change Drawing Size Dialog Box}{drawing-size}

This command pops up a small window (see Figure \ref{drawing-size}) containing
the current size of the drawing area in pixel. You can then change it. It is
not recommended to reduce it as you could render some parts of a OP invisible
or undisplayable (the invisible information is still present though, you just
cannot access it).

\subsubsection{Symbols List}
\node{Symbols List}
\cpindex{Symbols List}

\figuregif[scale=0.5]{symbol-list}{Symbol List Dialog Box}{symbol-list}

This command pops up a small window (see Figure \ref{symbol-list}) containing
the current list  of symbols declared in this OP file. You can then change the
content of this list which is saved within the OP file.

\subsubsection{Change Visible Fields}
\node{Change Visible Fields}
\cpindex{Change Visible Fields}

\figuregif[scale=0.5]{sel-field-dialog}{Selected Fields Dialog Box}{selected-fields}

When selected, this menu item pops up a dialog box (see Figure
\ref{selected-fields}) which enables you to tune the way a OP is displayed
on the screen, dealing with the visibility of unused or unnecessary fields.
The dialog box offers you a list of push buttons corresponding to the
different text fields of a OP (documentation part, context, and so on).
These push buttons are \code{on} when the corresponding field is visible.
They are \code{off} when it is invisible, and they are insensitive when the
corresponding field is not empty (in which case they cannot be made
invisible). You also have a \code{all} button and a \code{none} button
which allow you to make all the fields (to which the operation applies)
visible or invisible.

Note that you cannot hide a non empty field, and you cannot hide the
Invocation Part (which is always required). Similarly, to edit a field,
you need to make it visible first.

Note that you can hide the name of the OP if you want.

\subsection{Mode Menu}
\node{Mode Menu}
\cpindex{Mode Menu}

\figuregif[scale=0.5]{ope-mode-menu}{OP Editor Mode Menu}{ope-mode-menu}

This Mode menu (Figure \ref{ope-mode-menu}) contains commands equivalent to the
one in the working menu (\pxref{Working Menu Items}). Its purpose is mainly to
enable the use of accelerators on the working menu (as Motif only allowes the
use of accelerator on button in menu).

\subsection{\OPE{} Help Menu}
\node{OP Editor Help Menu}
\cpindex{OP Editor Help Menu}

\figuregif[scale=0.5]{ope-help-menu}{OP Editor Help Menu}{ope-help-menu}

The help menu (Figure \ref{ope-help-menu}) contains a number of items which,
when selected, pop your HTML browser with the proper documentation. The
first item is Help and shows the top level directory of the documentation.
The other item pops up the browser but at the selected documentation
section.

\section{Working Menu Items}
\node{Working Menu Items}
\cpindex{Working Menu Items}

\figuregif[scale=0.5]{ope-work-menu}{OP Editor Working Menu}{ope-work-menu}

The Working Menu (Figure \ref{ope-work-menu}) contains the
commands which are most often used to edit a OP.  Most of them deal with the
components of a OP. They remain selected until you select another mode. For
example, if you have selected \code{Destroy Object}, you stay in this mode
until you select another mode (this can be rather dangerous). By default, the
OP Editor starts in the \code{Move Objects} mode. Independently of any working
selected menu, middle click is always equivalent to Move Objects and right
click (whenever it is available) to Edit Object.



\subsection{Move Objects}
\node{Move Objects}
\cpindex{Move Objects}

This command is used to move objects on the screen. The movable objects
are: nodes, knots, text fields, edge texts. Just click on them and
a bounding rectangle appears. You can then drag the object to its new
position.

Note that if you click on the background window (i.e. not on an object),
the whole window moves. This can be faster than using the
scrolling bar.

\subsection{Create Node}
\node{Create Node}
\cpindex{Create Node}

Create a new node wherever you click.

\subsection{Open Node}
\node{Open Node}
\cpindex{Open Node}

A new node will be created just under the node on which you click. The outgoing
edge from the first node will be transferred on the second one. In other word,
you just opened the node to insert an edge on it.

\subsection{Create If Node}
\node{Create If Node}
\cpindex{Create If Node}

This command can be used to create IF-THEN-ELSE nodes.

\subsection{Flip Conj/Disj Out}
\node{Flip Conj/Disj Out}
\cpindex{Flip Conj/Disj Out}

When selected, this command allow the user to flip the status
(disjunctive/conjunctive) of the outgoing edges of a node. Conjunctive
corresponds to creation of new threads.

\subsection{Flip Conj/Disj In}
\node{Flip Conj/Disj In}
\cpindex{Flip Conj/Disj In}

When selected, this command allow the user to flip the status
(disjunctive/conjunctive) of the ingoing edges of a node. Conjunctive
correspond to merge of multiple threads.

\subsection{Create Edge}
\node{Create Edge}
\cpindex{Create Edge}

\figuregif[scale=0.5]{ope-cre-edge}{Create Edge Dialog Box}{create-edge}

Create an edge between the first selected node and the second one.
You can create intermediate knots by clicking on their desired position.
If, after selecting the first node, you change your mind and want to
unselect it, just click on the right mouse button. This will unselect the
first selected node.

If you have selected a second node, a prompt dialog box pops up (see Figure
\ref{create-edge}) and asks you to enter the goal to put on the edge.

\subsection{Create Knot}
\node{Create Knot}
\cpindex{Create Knot}

You first need to select an edge to which you want to add a knot by
clicking on the text part of the edge. Then, you can add as many knots as
you want. Click right to unselect the edge.

\subsection{Duplicate Objects}
\node{Duplicate Objects}
\cpindex{Duplicate Objects}

Duplicate a node or an edge. For the node, it will duplicate all its ingoing
and outgoing edges.

\subsection{Merge Node}
\node{Merge Node}
\cpindex{Merge Node}

Merge two nodes and all their edges.

\subsection{Edit Object}
\node{Edit Object}
\cpindex{Edit Object}

\figuregif[scale=0.5]{ope-edit-obj}{Edit Object Dialog Box}{edit-obj}

When this command is selected, the user can edit any editable object (i.e.
text fields, edge texts and node names). A modal dialog box appear (Figure
\ref{edit-obj}) and the user can change the object edited. This dialog box
has two particularities, there is a text field where the user can enter a
pretty print width, and a toggle button to specify if the pretty printer
should
fill up lines or not (\pxref{Pretty Printing}). The OP Editor is quite
rigorous for the syntax allowed in the text field. If anything wrong
is entered, you get an Error Dialog Box.

\subsection{Convert End}
\node{Convert End}
\cpindex{Convert End}

This enables the user to convert (or unconvert) node to end node. The
system checks that the node you are converting is a leaf of the
graph or that the node is not the Start node.

\subsection{Convert Start}
\node{Convert Start}
\cpindex{Convert Start}

This enables the user to convert a node to a start node. Considering that
only one start node is allowed, converting a node to start node
unconverts the current start node which will therefore automatically become
a standard node. The \OPE{} checks that the nodes you are converting do not
have ingoing edges or are not the start node.

\subsection{Align Object}
\node{Align Object}
\cpindex{Align Object}

This command enables the user to align nodes and text fields. You first
select an object (the anchor) on which you want to align other objects.
Then you click on the objects to align. It will align the other objects on the
closest of the vertical or horizontal line going through the anchor. If you
want to unselect the anchor, you need to press the right mouse button.

\subsection{Align Object Vert}
\node{Align Object Vert}
\cpindex{Align Object Vert}

This command enables the user to align nodes and text fields. You first select
an object (the anchor) on which you want to align vertically other objects.
Then you click on the objects to align. If you want to unselect the anchor, you
need to press the right mouse button.

\subsection{Align Object Hor}
\node{Align Object Hor}
\cpindex{Align Object Hor}

This command enables the user to align nodes and text fields. You first select
an object (the anchor) on which you want to align horizontally other objects.
Then you click on the objects to align. If you want to unselect the anchor, you
need to press the right mouse button.

\subsection{Destroy Object}
\node{Destroy Object}
\cpindex{Destroy Object}

This command is used when the user wants to destroy some objects on the screen.
Note that the destroy operation is done on the mouse up event\dots{} On the
mouse down event, the OP Editor shows you (by selecting it) the object which is
going to be destroyed. However, if (while the button is still pressed) you move
the mouse away from the bounding box of the object, it is not destroyed. A
recent version of the \OPE{} allows the user to destroy the knots of an edge.

\strong{Caution:} The current version of the OP Editor does not have an
Undo facility\dots{} So extreme care should be exercised when using this
command.

\subsection{Relevant OP}
\node{Relevant OP}
\cpindex{Relevant OP}

This command can be used to find out what are the OP relevant to a particular
goal. When this mode is selected, click on an edge (actually, the text of the
goal), and the \OPE{} will propose a list of relevant OP (among the OP
currently loaded in the \OPE{}). 

\chapter{OP File Format}
\node{OP File Format}
\cpindex{OP File Format}
% OP Format used in the OP Editor.

Version of the \OPE{} prior to 1.3 recognizes three different OP file
formats. The OPF File format is the official format used by the various
components of the \COPRS{} Development Environment. However, the two other
formats are used for compatibility with SRI Lisp \OPRS{} version.

Versions of the \OPE{} greater or equal to 1.3  only recognize the OPF
format. If you still have old files, use a version of the \OPE{} prior to 1.3
to convert them.




\section{OPF Format}
\node{OPF Format}
\cpindex{OPF Format}

The OPF format is the default file format used by ACS \OPEditor. It has
been designed for \COPRS{} Development Environment. It is the one and
only one format recognized by the \CPK{}. Moreover, the \OPE{}
knows how to write OP files only in this format (but can read other
formats). There exist different version of this format, but this is an
internal flag which allow the modules of the \COPRSDE{} to distinguish
between them.

\section{GGRAPH Format}
\node{GGRAPH Format}
\cpindex{GGRAPH Format}

The GGRAPH (Grasper Graph format) is provided for compatibility. The
\OPE{} is able to read this format and understand the various graphical
information it contains. However, several details
(usually unnecessary to the \CPK{}) are not properly parsed in
this format (such as the fonts used on a particular edge).

For various reasons, one cannot guarantee that the \OPE{} will
properly parse all the Grasper Graph files. If you have any problem with
a particular file, check the list of known problems with this format and
report your problem to \code{oprs-bug@ingenia.fr}.



\subsection{How to Get Grasper Graph on your Lisp Machine}
\node{How to Get Grasper Graph on your Lisp Machine}
\cpindex{How to Get Grasper Graph on your Lisp Machine}

In Grasper, load your OP graph, and select it. Make sure you are in the
proper pacopge (most likely OPRS), and execute the following code in the
Grasper Listener pane:

\begin{verbatim}
(let ((*print-length* nil)
      (*print-level* nil))
     (with-open-file (foo-graph "foo.ggraph" :direction :output)
                          (format foo-graph "~s" (grasper:describe-graph)))
\end{verbatim}

This saves your graph in the file \file{foo.ggraph}.

\strong{Warning:} These text-graphs are different from those you get when
you
execute the command Output-SUN under Grasper (which delivers a SGRAPH
format)\dots{} The former contains positions and graphical information, the
latter does not.

It has been reported that a nickname given to the pacopge \code{GRASPER}
causes Grasper to save the file with the nickname instead of the original
\code{GRASPER} symbol. The \OPE{} expects the \code{GRASPER} symbol\dots{}
So at worst, you need to remove this nickname before saving, or
replace the nickname with \code{GRASPER} under a text editor.

\subsection{Grasper Graph Incompatibilities}
\node{Grasper Graph Incompatibilities}
\cpindex{Grasper Graph Incompatibilities}

While loading a Grasper Graph, few things are not properly recognized by
the \OPEditor:



\subsubsection{Unrecognized Symbol}
\node{Unrecognized Symbol}
\cpindex{Unrecognized Symbol}

Replace \code{|\#:|} with \code{|\#} in node names like: \code{|\#:|e7756|}
These node names come from the time of ZetaLisp.

In the following forms, the \& and ! are not parsable.

\code{(*FACT (EQUAL \& (CAADR (GOAL-STRUCT-STATEMENT (OP-INSTANCE-GOAL
(FIRST \$X))))))}

\code{(*FACT (EQUAL ! (CAR (GOAL-STRUCT-STATEMENT (OP-INSTANCE-GOAL (FIRST
\$X))))))))}

The reason is quite obvious: these symbols should not be used explicitly in OPs
and these tests should be replaced with evaluable predicate doing the same
thing.

\subsubsection{Negation as Failure}
\node{Negation as Failure}
\cpindex{Negation as Failure}

|Meta ~! (negation as failure)| (which is a OP in default-processes.graph)
cannot be parsed because of the absence of negation as failure in \COPRS{}.  As
a consequence, the expression \code{(! (~ \$X))} cannot be parsed as a Temporal
Expression (according to \COPRS{} Grammar).

\subsubsection{Badly Formed Goals}
\node{Badly Formed Goals}
\cpindex{Badly Formed Goals}

The following goal is not allowed: \*
\code{(! (SEND-COMMAND-ACTION (DISPLAY-FOO (((CURRENT \$O) (CURRENT
\$B))))))} \*
because \code{(((CURRENT \$O) (CURRENT \$B)))} is not a valid \dfn{Composed
Term}.

This goal: \*
\code{( ! (\& (~> (FOOBAR \$P \$A)) (=> (FOOBAR \$P \$A)))))))} \*
is not allowed anymore\dots{}, replace it with: \*
\code{(\& (~> (FOOBAR \$P \$A)) (=> (FOOBAR \$P \$A))))))} \*

\section{SGRAPH Format}
\node{SGRAPH Format}
\cpindex{SGRAPH Format}

The SGRAPH or SUN Graph format is provided for compatibility with a OP
Format which has been used on SRI Sun version of the Lisp version of \OPRS.

Note that this format does not have any graphical information.
Consequently, it should be used in the last resort.

If you load a SGRAPH format file, nothing shows on the screen (this format does
not include any graphical information\dots{}). You have to save it first (in
OPF format) and read it again. The various OP fields are positioned at their
default position\dots{} But all nodes and edges are grouped in one position (20
20). You can then rearrange them as you wish. A future version of the \OPE{}
may attempt to display these graphs with a better organization, however, the
current one does not (Note that if you had the possibility to generate OP File
in SGRAPH format, you could also generate it in GGRAPH format, which is
preferred).


%%% Local Variables: 
%%% mode: latex
%%% TeX-master: "oprs"
%%% End: 

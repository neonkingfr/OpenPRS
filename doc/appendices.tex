%%%%%%%%%%%%%%%%%%%%%%%%%%%% -*- Mode: latex -*- %%%%%%%%%%%%%%%%%%%%%%%%%%
%%
%% Copyright (c) 1991-2004 Francois Felix Ingrand.
%% All rights reserved.
%%
%% Redistribution and use in source and binary forms, with or without
%% modification, are permitted provided that the following conditions
%% are met:
%%
%%    - Redistributions of source code must retain the above copyright
%%      notice, this list of conditions and the following disclaimer.
%%    - Redistributions in binary form must reproduce the above
%%      copyright notice, this list of conditions and the following
%%      disclaimer in the documentation and/or other materials provided
%%      with the distribution.
%%
%% THIS SOFTWARE IS PROVIDED BY THE COPYRIGHT HOLDERS AND CONTRIBUTORS
%% "AS IS" AND ANY EXPRESS OR IMPLIED WARRANTIES, INCLUDING, BUT NOT
%% LIMITED TO, THE IMPLIED WARRANTIES OF MERCHANTABILITY AND FITNESS
%% FOR A PARTICULAR PURPOSE ARE DISCLAIMED. IN NO EVENT SHALL THE
%% COPYRIGHT HOLDERS OR CONTRIBUTORS BE LIABLE FOR ANY DIRECT, INDIRECT,
%% INCIDENTAL, SPECIAL, EXEMPLARY, OR CONSEQUENTIAL DAMAGES (INCLUDING,
%% BUT NOT LIMITED TO, PROCUREMENT OF SUBSTITUTE GOODS OR SERVICES;
%% LOSS OF USE, DATA, OR PROFITS; OR BUSINESS INTERRUPTION) HOWEVER
%% CAUSED AND ON ANY THEORY OF LIABILITY, WHETHER IN CONTRACT, STRICT
%% LIABILITY, OR TORT (INCLUDING NEGLIGENCE OR OTHERWISE) ARISING IN
%% ANY WAY OUT OF THE USE OF THIS SOFTWARE, EVEN IF ADVISED OF THE
%% POSSIBILITY OF SUCH DAMAGE.
%%
%% $Id$
%%
%%%%%%%%%%%%%%%%%%%%%%%%%%%%%%%%%%%%%%%%%%%%%%%%%%%%%%%%%%%%%%%%%%%%%%%%%%%%%%%

\part{Appendices}
\node{Appendices}
\cpindex{Appendices}

\appendix


\chapter{Principal Differences Between C-PRS and \OPRS{}}
\node{Principal Differences Between C-PRS and \OPRS{}}
\cpindex{Principal Differences Between C-PRS and OPRS}


\chapter{Principal Differences with SRI PRS}
\node{Principal Differences with SRI PRS}
\cpindex{Principal Differences with SRI PRS}

One of the principal concern while writing C-PRS and \COPRS{} was to
improve some of the known problems in SRI Lisp PRS. As a consequence, there
are minor differences between the two versions.

\begin{enumerate}

\item The syntax is far more rigorous in \COPRS{}. As a result, a
number of expressions perfectly correct in the Lisp PRS are not
recognized by \COPRS{}. For example, the following goal is not allowed: \*
\code{(! (SEND-COMMAND-ACTION (DISPLAY-FOO (((CURRENT \$O) (CURRENT
\$B))))))} \*
because \*
\code{(((CURRENT \$O) (CURRENT \$B)))} \*
is not a valid Composed Term. Similarly, this goal: \*
\code{( ! (\& (~> (FOOBAR \$P \$A)) (=> (FOOBAR \$P \$A)))))))} \*
is not allowed any longer\dots{} It should be replaced with: \*
\code{(\& (~> (FOOBAR \$P \$A)) (=> (FOOBAR \$P \$A))))))}

\item The following syntax is still allowed: \*
\code{( ! (=> (INITIAL-VALUE \$P \$O \$D \$V)))} \*
but should be replaced with the \code{=>} temporal operator (a similar
remark holds for \code{~>}) .

\item The \code{@} variables have the expected behavior. In SRI PRS, the
only way to rebind a variable was to use a goal such as \code{(! (= @x
...))}. This limitation has been removed in \COPRS{}, and the \code{@}
variables can be rebound in any context.

\item The database uses a term indexing mechanism which has been extended
recursively (SRI PRS version did not handle embedded composed terms
properly).

\item The wait temporal operator is cleaner in \COPRS{} as it succeeds only
when it has been achieved\dots{} In SRI PRS, it always succeeded (but it
sleep until the condition was satisfied).

\item The database file has not quite the same format. You need to give a
list of facts, not just the facts in a file. To convert your already
existing files, just add an open parenthesis at the beginning, and a
closing one at the end, and remove all the declarations of functional
facts, basic events, and so on.

\item \COPRS{} does not support the negation as a failure, nobody ever used
it anyway.

\item Closed world predicates, evaluable functions and evaluable predicates
are ``local'' to a OPRS agent which is much better.

\item SOAK and other meta facts are now basic event (\pxref{Basic Events}).

\item \COPRS{} has no support for bignum\dots{} As a general rule, extreme
caution should be exercised regarding the availability of Lisp specific
functions or features. We do not plan to rewrite a complete Lisp
interpreter in \COPRS{}\dots{} (this would defeat the main argument for
writing \COPRS{}).

\item \COPRS{} checks predicates, functions and symbols when you compile
OPs.

\item Basic events are just declared with their predicates, not the
predicates and the number of arguments, since we considered this feature
useless.

\item The achiever field in OPs is not supported anymore. If it exists, it
does not even shows up in the \OPE{} or in the \XPK, but a warning is
issued. Moreover, its value is not taken into account by the kernel (here
also, this feature has hardly been used\dots{} so it was not worth putting
in the C version.)

\item Lisp lists are not ``standard'' objects: you need to specify them
with a different reader syntax (. .) to distinguish them from composed
terms.

\item The default OPs are not loaded by default in the \CPK{}. In
fact the notion of default OPs is a little weak, and we consider that it
is up to the end-user to decide which OPs are required in the kernel. Of
course, we still provide a number of OP the user can load in its
kernel(\pxref{Default OPs}).

\item The \code{APPLICABLE-OPS-FACT} and the \code{APPLICABLE-OPS-GOAL}
facts have different arguments in \CPK{}. Their syntax is: \*
\code{(APPLICABLE-OPS-FACT fact list-of-op-instances)} \*
and \*
\code{(APPLICABLE-OPS-GOAL goal list-of-op-instances)}. \*
In SRI PRS it was: \*
\code{(APPLICABLE-OPS-FACT list-of-op-instances)} \*
and \*
\code{(APPLICABLE-OPS-GOAL list-of-op-instances)}.

\item By default, the \CPK{} always evaluates what is evaluable in a goal
at posting time, unless the evaluable function is quoted (\pxref{Current and Quote}). This greatly simplify the syntax of the OPs as one do not need
to put \code{current} all over the place... Note that there is a flag in
case you prefer the old form, see \nxref{OPRS Kernel Run Option Commands} for
more details.

\item The \code{*FACT} and \code{*GOAL} marker in the general meta expression
are not recognized in recent version of \COPRS{}. Version of the OP-Editor prior
to 1.3 will recognize them and will thus convert your old OP to the new format.

\end{enumerate}

\chapter{Principal Differences Between Subsequent Versions of C-PRS}
\node{Principal Differences Between Subsequent Versions of C-PRS}
\cpindex{Principal Differences Between Subsequent Versions of C-PRS}




\section{Changes Between Version 1.0 and Version 1.1}
\node{Changes Between Version 1.0 and Version 1.1}
\cpindex{Changes Between Version 1.0 and Version 1.1}

There are a number of small changes which can be of interest to the user. For
more information, consult the \file{NEWS} file int the C-PRS distribution.

\begin{itemize}

\item Change the string pattern to allow \samp{"} in it. We are now using the C
string syntax... But for now, just the \code{\\"}, such as in \code{"foo bar
\\"asd asd\\""}. Internally, strings now do not have double quote around as they
used to. Now they are just C strings.

\item Added a printf function, more like the C one. Example:\*
\code{(printf (format "The \%d of \%s is \%f." \$x \$y \$z))} 

\item Added a list of symbols in the OP Editor and in the OPF format.

\item The kernel is now able to read pointer values.

\item Added a new way of scheduling intention. Instead of giving a
sorting predicate, it is given the list of runnable intentions, and it can be
ordered as the user want. This make it easier to make time sharing stuff for
example.

\item Added an extern \code{x\_oprs\_top\_level\_widget} for the user
to create its own widget tree in \XPK{}.

\item Defined the \code{OPRS\_DATA\_PATH} environment variable and the
corresponding \code{-d} argument. This is a list of directories where C-PRS
will look for data files.

\item Defined \code{OPRS\_MP\_PORT}, \code{OPRS\_SERVER\_PORT},
\code{OPRS\_SERVER\_HOST} and \code{OPRS\_MP\_HOST} environment variables, and
changed the various programs accordingly to take them into account. (command
arguments have a higher priority than environment variables).

\item Defined a \code{OPRS\_DOC\_DIR} environment variable to point to the
documentation directory.

\item  Implementation of \code{INT\_ARRAY} and \code{FLOAT\_ARRAY}. Added \samp{[} and \samp{]}
as reader characters to recognize arrays. The type of the array is determined
from the first element.  Subsequent elements are casted appropriately.

\item  Implementation of \code{U\_POINTER} user pointer to user defined objects.

\end{itemize}

\section{Changes Between Version 1.1 and Version 1.2}
\node{Changes Between Version 1.1 and Version 1.2}
\cpindex{Changes Between Version 1.1 and Version 1.2}

There are a number of important changes between version 1.1 and version 1.2.
Only the important one are listed in the following sections.



\subsection{Changes in the Commands Syntax of the \CPK{}}
\node{Changes in the Commands Syntax of the OPRS Kernel}
\cpindex{Changes in the Commands Syntax of the \CPK{}}

To clarify it and to make it more easy to use, the command set of the
\CPK{} has been greatly improved and unified.

Here is a table of correspondence between the 1.1 command and their new
syntax in 1.2 and above. The commands which are not referenced have not changed.

\begin{table}
\begin{center}
\begin{tabular}{||l|l|l||}
\hline
1.0&1.1.1  & Section \\
\hline
 show\_db & show db & \ref{OPRS Kernel Database Commands} \\
 save\_db & save db & \ref{OPRS Kernel Database Commands} \\
 empty\_db & empty fact db & \ref{OPRS Kernel Database Commands} \\
 declare\_cwp & declare cwp & \ref{OPRS Kernel Declaration Commands} \\
 declare\_be & declare be & \ref{OPRS Kernel Declaration Commands} \\
 undeclare\_be & undeclare be & \ref{OPRS Kernel Database Commands} \\
 declare\_ff & declare ff & \ref{OPRS Kernel Declaration Commands} \\
 load\_db & load db & \ref{OPRS Kernel Database Commands} \\
 delete\_op & delete op & \ref{OPRS Kernel OP Library Commands} \\
 unload\_opf & delete opf & \ref{OPRS Kernel OP Library Commands} \\
 print\_op & show op & \ref{OPRS Kernel OP Library Commands} \\
 traceg\_op & trace graphic op & \ref{OPRS Kernel OP Library Commands} \\
 tracet\_op & trace text op & \ref{OPRS Kernel OP Library Commands} \\
 traceg\_opf & trace graphic opf & \ref{OPRS Kernel OP Library Commands} \\
 tracet\_opf & trace text opf & \ref{OPRS Kernel OP Library Commands} \\
 list\_ops & list op & \ref{OPRS Kernel OP Library Commands} \\
 list\_opfs & list opf & \ref{OPRS Kernel OP Library Commands} \\
 compile\_ops & load opf & \ref{OPRS Kernel OP Library Commands} \\
 empty\_op & empty op db & \ref{OPRS Kernel Loading Commands} \\
 trace rop & trace relevant op & \ref{OPRS Kernel Trace Commands} \\
 trace opc & trace load op & \ref{OPRS Kernel Trace Commands} \\
 trace soak & trace applicable op & \ref{OPRS Kernel Trace Commands} \\
 trace fact & trace fact & \ref{OPRS Kernel Trace Commands} \\
 trace goal & trace goal & \ref{OPRS Kernel Trace Commands} \\
 trace db\_frame & trace db frame & \ref{OPRS Kernel Trace Commands} \\
 set fact\_inv & set meta fact & \ref{OPRS Kernel Meta Level Option Commands} \\
 set goal\_inv & set meta goal & \ref{OPRS Kernel Meta Level Option Commands} \\
 set app\_ops\_fact & set meta fact op & \ref{OPRS Kernel Meta Level Option Commands} \\
 set app\_ops\_goal & set meta goal op & \ref{OPRS Kernel Meta Level Option Commands} \\
 set eval\_on\_post & set eval post & \ref{OPRS Kernel Run Option Commands} \\
 set par\_post & set parallel post & \ref{OPRS Kernel Run Option Commands} \\
 set par\_intend & set parallel intend & \ref{OPRS Kernel Run Option Commands} \\
 consult\_rop & consult relevant op & \ref{OPRS Kernel Miscellaneous Commands} \\
 consult\_aop & consult applicable op & \ref{OPRS Kernel Miscellaneous Commands} \\
 show\_sleep\_int & show intention & \ref{OPRS Kernel Miscellaneous Commands} \\
 show\_mem & show memory & \ref{OPRS Kernel Miscellaneous Commands} \\
 declare & declare id & \ref{OPRS Kernel Declaration Commands} \\
 unifie & unify & \ref{OPRS Kernel Miscellaneous Commands} \\
 stat\_db & stat db & \ref{OPRS Kernel Miscellaneous Commands} \\
 stat\_op & stat op & \ref{OPRS Kernel Miscellaneous Commands} \\
 stat\_id & stat id & \ref{OPRS Kernel Miscellaneous Commands} \\
 stat\_all & stat all & \ref{OPRS Kernel Miscellaneous Commands} \\
 reset\_kernel & reset kernel & \ref{OPRS Kernel Miscellaneous Commands} \\
\hline
\end{tabular}
\end{center}
\caption{Commands Equivalence Between Version 1.1 and 1.2}
\end{table}

A shell script \file{update-inc-file} in the \file{util} directory of the
distribution, is provided to allow the user to automatically translate its
\file{.inc} and \file{.sym} files from versions previous to 1.1 to version 1.2.

\subsection{Miscellaneous Changes Between Version 1.1 and Version 1.2}
\node{Miscellaneous Changes Between Version 1.1 and Version 1.2}
\cpindex{Miscellaneous Changes Between Version 1.1 and Version 1.2}

There are a number of small changes which can be of interest to the user. For
more information, consult the \file{NEWS} file in the C-PRS distribution.

\begin{itemize}

\item Added a new option to the \CPK{} and to the \XPK{} to start
without registering to the \OPRSS{} (\code{-a}).

\item Added a new command in the \OPE{} (relevant OP) to find the
OP relevant for a particular edge.

\item Added an option to the \OPE{} to convert op files to the
newest format \code{op-editor -c <file(s)>}.

\item Time stamping is now controlled by a flag (it is too
expensive to keep it all the time).

\item Wrote a default op to broadcast a message. Wrote a broadcast-message action.
Added a broadcast mode to the message passer. I separated the modules which can
receive and the one which can send (the \OPE{} and the \OPRSS{} do not
receive). Therefore, the \MPA{} never sends anything to the \OPRSS{} and the
\OPE{}.

\item Added a \code{val} evaluable function which returns the value on which
variable points... In fact it is the identity function, but it can be very
useful to force getting the value of program variables.

\item Added a number of new commands and associated tokens:
\code{list cwp}, \code{list be}, \code{list ff}, \code{list ep}, \code{list
ef}, \code{list action} and \code{list all}.

\item Wrote a \code{show intention} command (also available as a menu item)
which gives an extensive status of the intentions and their component (thread,
status, waiting condition, joining, etc).

\item Added code to execute all the root of the intention graph in //. This
is controlled with a flag (\code{set parallel intention on/off}) and in the
option menu. Default value of the flag is on.

\item Added op predicate, i.e. predicate which can only be satisfied by
OP not in the database.

\item Now the \MPA{} is started automatically by the application which want to
connect to it. In other words, if a connection failed because nobody is
listening on the host/port, then a \MPA{} is started on this host/port.  We use
\code{rsh} to start the mp on a remote host, and fork to start it on the same host.

\item Added a new program \code{kill-mp} to kill the \MPA{}. It registers to the
\MPA{} and kill it.

\item To prevent conflict with types defined in VxWorks system includes... the
type \code{LIST} and \code{NODE} have been renamed \code{OPRS\_LIST} and
\code{OPRS\_NODE}. 

\item Change the registration to the \MPA{} mechanism to pass the registration
protocol as argument... and to return a status to deny/allow the registration.
This change makes the registration mechanism incompatible with the previous
version.

\end{itemize}

\section{Changes Between Version 1.2 and Version 1.3}
\node{Changes Between Version 1.2 and Version 1.3}
\cpindex{Changes Between Version 1.2 and Version 1.3}

There are a number of important changes between version 1.2 and version 1.3.
The most important change between version 1.2 and 1.3 is the text OPs and is
described in the \xref{Text OPs} section.



\subsection{Miscellaneous Changes Between Version 1.2 and Version 1.3}
\node{Miscellaneous Changes Between Version 1.2 and Version 1.3}
\cpindex{Miscellaneous Changes Between Version 1.2 and Version 1.3}

There are a number of small changes which can be of interest to the user. For
more information, consult the \file{NEWS} file in the C-PRS distribution.

\begin{itemize}

\item Added CONS-TAIL and LAST evaluable function.

\item ope-graphic.c (scroll\_bars\_moved): Wrote a function to get the
horizontal and vertical motion at the same time (to avoid this
the awful steppy look of the redrawing).

\item mp-oprs.c: added a -l "file\_name" arg to specify an log file
(in which mp will log all the message which went thru it).

\item xp-op-graphic.c xp-rop.c: Extend the relevant OP mechanism, to
the invocation and context parts in the \XPK{} and in the \OPE{}.

\item Added a "stepped" and "all" button in the Trace OP dialog box.

\item Added the next, step, halt and run commands in the oprs kernel.

\item Added a "show run status" command in the \CPK{}.

\item The step by step execution of OP is now controled by a separate
flag (not the graphic traced one anymore). The commands "trace
step op ... on/off" et "trace step opf ... on/off" have been
added.

\item Added == as evaluable predicate to allow fancy construction such
as (? (V (== (..) (..)) (== (...) (...)))).

\item Added a scheduler for parallel intententions.

\item Added evaluable functions to return
user, system and user+system clock tick:
USER-SYS-CLOCK-TICK, USER-CLOCK-TICK, SYS-CLOCK-TICK.
These do not work under VxWorks.
 
\item Added a "transmit\_all" command in the oprs-server to transmit the same
command to all connected kernel. 

\item Added an option to have symbol in lower case or any case.
The default is still upper case.
The authorized options are "upper", "lower" or "none".  
You may specify this option by setting the OPRS\_ID\_CASE
environment variable:\*
     setenv OPRS\_ID\_CASE none \*
or in the command line with the '-l' option:\*
     xoprs -n Foo -l none -x ...\*
The command line has precedence on the envionment variable.
This option is avaible for the \OPRSS{}, the PK{}, and the \OPE{}.

\item Change the command \code{list op predicate} in \code{list op\_predicate}
and \code{declare op predicate} in \code{declare op\_predicate}.

\item Report the expr being concluded when an attempt is made to conclude such
expr containing a variable.

\item Added a LISP\_LIST intersection and a LISP\_LIST union functions. 

\item \code{mp\_socket} has default value of -1 (to detect
unitialization for example).

\item Added a memq\_ep evaluable predicate MEMQ.

\item \code{start\_kernel\_user\_hook} is now executed before the
initial\_command (i.e. load with the -x).

\item The name of the opfile appears in all op lists
presented to the user.

\item xp-dialog.c: Added the Conclude from Parser trace in \XPK{}.

\item Added a \code{find\_atom} function which does like \code{make\_atom}
but warn you if the symbol had not been declared previously.

\item Many prompt dialog box have been changed to provide an history of the
previously entered commands.

\item Added a \code{build\_string} function to build a Term from
a string, making a copy of the string. This is the function external users
should use.

\item Added a \code{show oprs\_data\_path} and a \code{set oprs\_data\_path <string>}
command. 

\item Added a \code{trace conclude} command to trace conclude operation
performed by the user.

\item The \XPK{} menus have been reorganized.

\item Added a \code{Display Next OP} and \code{Display Previous OP} in \XPK{}.

\item The \MPA{} registering functions have been changed. This change make them
incompatible with previous versions of the \MPA{}.

\item In the various parsers, we have added some code to check we are not
opening a directory instead of a file.

\item Added extern "C" declaration in all the public include files
which may be used by a c++ compiler.

\item Added a verbose argument \code{-v} to the \MPA{} to trace all messages.

\item An error handler for external code errors is now provided. 

\item One can log in a file all the output made in the \XOPRS{} text window
using the \code{-log} argument to \XOPRS{}. 

\item Now we can have the text trace associated to a particular intention in
its own window... To create this particular window, right click on the
intention in the intention graphic pane. Middle click gives an overview of the
intention current status.

\item The database is sorted before being printed.

\item Some trace messages have been shortened to make the trace less
verbose.

\item Added \code{build\_nil} and \code{build\_t} which return a Term with t or
nil as atom. Useful for evaluable functions and actions.

\item Reorganised the option menu in \XOPRS{}. We now have three different
option dialog boxes (Run, Compiler/Parser and Meta Level).

\item  Implementation of a = ev-predicate... It works... it is faster... and it
allows setting variable in invocation part parsing. (requested by
rachid@laas.fr).

\item Added a Show Global Variable menu and a "show variable" command to print
the global variables list. 

\item The \MPA{} kills itself after an hour with no new connection and when
nobody is registered. 

\item Added a show memory menu.

\item Added a \code{delete fact} and \code{conclude fact} command in the \XPK{}.

\item Memory consumption has been improved.

\item The list of variables do not appear anymore in the frame bindings.

\item Added a \code{require} command which does like include except that it
check this file has not been already loaded with another \code{require} command
(\pxref{OPRS Kernel Loading Commands}).

\item When broadcasting, do not send the message to the sender.

\item Added two new flags to control predicate and function
declaration. They can be changed with \code{set predicate|function on|off}
(\pxref{OPRS Kernel Compiler/Parser Option Commands}).

\item Added two new commands: \code{declare predicate} and \code{declare
function} commands (\pxref{OPRS Kernel Declaration Commands}). 

\item Added a delete fact (\pxref{Delete Fact Database}) and a conclude fact
(\pxref{Conclude Fact Database}) command in \XPK{}.

\item There is a semaphore OP library available (\pxref{semaphore.opf}).

\item the Write TeX Doc File command is now public (\pxref{Write TeX Doc File}).

\item ope/Parser.l: Dropped the *GOAL and *FACT syntax... If you still
have OP using this syntax, use the 1.2 OP editor to convert your
OP file. 

\item Modified the parser to count lines in files and in strings so we
can report where are the errors more accurately. 

\item Kernels and op-editor can now print the temporal operators in
english... This is controlled by a flag which can be set at startup time with
the "-peo" (PrintEnglishOperator) option for xoprs and op-editor and "-p"
option for oprs. This can also be specified as a ressource named
*printEnglishOperator (True or False). Default value is always false... and the
OP written in files are written with the short version. Clean up all the TEST
symbol in the various test file... (remember that TEST is now a reserved word).

\item XOprs.ad: Added a resource setting to enable/disable word
wraparound in the textWindow.

\item All grammar share the same basic files. So the
lexical grammar is now consistent other the 4 modules using it
(oprs, xoprs,op-editor and oprs-server). 

\item Removed support for grasper graph and sun graph. 

\item Created accelators in the op editor to quickly access all the edit modes.
This is user customizable with the default file. By default, they are
accelerated with one letter (which is present as a mnemonic on the text of the
button). No modifier are used (meta,control, nor shift).

\item The id cannot start with a : character. (reserved for keywords
in text op).

\item data/new-default.opf: Removed test and test-and-set which are
now obsolete and conflict with the test symbol used as a
temporal operator.

\item ev-predicate.c: defined some new predicates: NUMBERP, CONSP,
STRINGP, and ATOMP.

\item mp-register.c and send-message.c are using the standard
malloc... 

\item Grammars allows temporal operator in clear (achieve, test,
preserve, maintain, wait, conclude and retract).

\item The function \code{make\_and\_declare\_action}, 
\code{make\_and\_declare\_eval\_func} and \code{make\_and\_declare\_eval\_pred} do not
take a hash table argument anymore. Please, modify your code accordingly.

\item macro-pub.h and macro.h (MALLOC): is now defined in macro-pub
for the final user to allocate object for the kernel.

\item Added OPRS\_DATA\_PATH support under VxWorks.

\item ope-save.c (write\_opfile\_header): Added the version status in
the OP file header.

\item XOprs.ad. The linotype helvetica looks ugly... I had to force the
adobe one in the font set.

\end{itemize}

\section{Changes Between Version 1.3 and Version 1.4}
\node{Changes Between Version 1.3 and Version 1.4}
\cpindex{Changes Between Version 1.3 and Version 1.4}



\subsection{Main Changes Between Version 1.3 and Version 1.4}
\node{Main Changes Between Version 1.3 and Version 1.4}
\cpindex{Main Changes Between Version 1.3 and Version 1.4}

There are a number of important changes between version 1.3 and version 1.4.
The most important changes between version 1.3 and 1.4 are:

\begin{itemize}

\item The possibility to dump various internal data (OPs and database for now),
and the various dump commands. Note that the dump format is architecture
independant. Note also, that the garphical information may be dumped or not,
according to the kernel from which you dump the datas. While loading OP, if a
OP with a name and a filename is already loaded, the newly loaded OP is
ignored. \xref{OPRS Kernel Dumping/Loading Commands} for more information on
this. 

\item A better support for the VxWorks version (which now supports the \MPA{}
and the \code{kill-mp} programs). \xref{VxWorks} for more information on
this. 

\item Support for multiple languages (French and English for now) and for
iso-latin characters set.

\item The non graphical version of the various \COPRSDE{} programs have been
ported and work under Windows 95 and Windows NT. \xref{Windows95-NT} for more
information on this.

\item No more \code{LISP\_CAR}. \code{LISP\_LIST} now contains Terms, and Terms
can now be Intention, Fact, Goal and Op\_Instance. They are not readable
though. One can remove all the \code{term\_to\_car} and \code{car\_to\_term}
fuctions, these are obsolete and NO-OP now.

\item Introduction of a \code{contrib} directory in the distribution which
contains code contributed by various people but which is not part of the \OPRS{}
distribution. 

\end{itemize}

\subsection{Miscellaneous Changes Between Version 1.3 and Version 1.4}
\node{Miscellaneous Changes Between Version 1.3 and Version 1.4}
\cpindex{Miscellaneous Changes Between Version 1.3 and Version 1.4}

There are a number of small changes which can be of interest to the user. For
more information, consult the \file{NEWS} file in the C-PRS distribution.

\begin{itemize}

\item Added a \code{-x} option to the \MPA{} to specify that new connection with
already registered name should lead to the disconnection of the older client.

\item Node name are printed (in graphic) without the vertical bar. 

\item Replace the \code{LEXPRESSION} term type with the
\code{GEXPRESSION} as they are easier to manipulate...

\item When printing the value of a variable, still print the variable name if
no value was found/bound.

\item Allow id starting with \samp{:} (like keywords). 

\item Added evaluable functions \code{mention} and \code{all-pos}.

\item Added macro to warn the user of the use of obsolete user functions.

\item Added functions to allow the posting of fact from user call function.
Under VxWorks, this even can been done from another process (the posting is
then protected with a semaphore to ensure mutual exclusion). \xref{Fact Posting Functions} for more information on this.

\item oprs-type-pub.h: Due to some stupid type definition in Microsoft Windows,
we had to replace the type FLOAT with TT\_FLOAT and ATOM with TT\_ATOM.

\item Big clean up in the VxWorks version to free all the memory used
when one exit/kill a kernel. This way, one can restart another oprs on
the same board, without loosing any memory allocation. In fact, this mechanism
is used on all kernels, which explain why it takes sometime a long time for a
kernel to exit. However, this mechanism is a good leak detector.

\item Check all conditions after loading a database.

\item Two different commands to print graphics OP and text OPs.

\item The OP printed in the file are now pretty printed.

\item Do not allow id starting with a + or -

\item Just one \code{"\\n.\\n"} should be necessary to reset the \OPRSS{} parser.

\item ASSERT is now equivalent to CONCLUDE, and it prints ASSERT instead of
CONCLUDE.

\item Added an intention failure trace which reports when
an intention failed, with the stack of goals which lead to the
failure.

\item XOprs.adlang: Language dependent ressource file.

\end{itemize}

\chapter{Hardware and Software Dependancies}
\node{Hardware and Software Dependancies}
\cpindex{Hardware and Software Dependancies}

Although \COPRS{} is portable and runs on a large number of platforms, there
exist some differences which are presented in this chapter.



\section{VxWorks}
\node{VxWorks}
\cpindex{VxWorks}

The VxWorks version of \COPRS{} has a lot of particulariries. Although there are
no main programs under VxWorks, we distinguish between task spawnable code and
library code (i.e. which may be used by more than one processes). Files with a
\file{.o} suffix are considered as library and can be used by more than one
processes. Files without any suffix usually contain a \code{main} which can be
called with \code{taskSpawn} or from the shell.

The VxWorks distribution is composed of three programs and two libaries:

\begin{description}

\item [\file{c\_toolkit.o}] This library is used by the \file{vxoprs} and
\file{vx-mp-oprs} components. It can be shared by more than one programs. The
proper variable are taskVar'ed.

\item [\file{vx-mp-lib.o}] This library is used by any application which want
to connect to the message passer. It can be shared by more than one
programs. The proper variable are taskVar'ed. \file{vxoprs} require this
library. 

\item [\file{vxoprs}] This is the default VxWorks \CPK{}. It requires
\file{c\_toolkit.o} and \file{vx-mp-lib.o} to be loaded on the same board. The
entry point is \code{oprs\_main} (see below). 

\item [\file{vxoprs-relocatable}] This is the default VxWorks \CPK{} minus the
entry points to allow the user to link its own code. It requires
\file{c\_toolkit.o} and \file{vx-mp-lib.o} to be loaded on the same board. The
entry point is \code{oprs\_main} (see below). \xref{Linking C Code in the Kernels} for more on this.

\item [\file{vx-mp-oprs}] This is the VxWorks \MPA{} program.  It requires
\file{c\_toolkit.o} to be loaded on the same card. The entry point is
\code{mp\_oprs\_main} (see below).

\item [\file{vx-kill-mp}] This is the VxWorks \code{kill-mp} program.  It
requires \file{vx-mp-lib.o} to be loaded on the same card. The entry point is
\code{kill\_mp\_main} (see below).


\end{description}

In VxWorks, there is no \code{main} in the code. The \COPRS{} main function has
therefore been renamed and is called \code{oprs\_main} and has the following
prototype:

\begin{verbatim}
int oprs_main(char *name_arg, char *server_hostname_arg, 
             int server_port_arg, char *mp_hostname_arg, 
             int mp_port_arg, char *include_filename_arg)
\end{verbatim}

The arguments are explained below:

\begin{description}

\item[\code{name\_arg}] is the name of the \CPK{}.

\item[\code{server\_hostname\_arg}] is the hostname on which the \OPRSS{} is running.

\item[\code{server\_port\_arg}] is the port on which the \OPRSS{} is listenning.
If 0, no connection is made to the \OPRSS{}.

\item[\code{mp\_hostname\_arg}] is the hostname on which the \MPA{} is running.

\item[\code{mp\_port\_arg}] is the port on which the \MPA{} is listenning. If 0,
then no connection is made to the \MPA{}.

\item[\code{include\_filename\_arg}] is an include file name to load upon
starting. If NULL, or empty string no file is loaded.

\end{description}

The \CPK{} cannot be run more than once on the same board (VxWorks processes
share the same symbol space). However, the \file{vx-mp-lib.o} library is
shareable. More than one program can use it on the same board. The proper
variables have been put in VxWorks \code{taskVar}.

Most environment variables are ignored under VxWorks, except for the
\code{OPRS\_DATA\_PATH} variable.

The \MPA{} main function has been renamed and is called \code{mp\_oprs\_main} and
has the following prototype:

\begin{verbatim}
int mp_oprs_main(int mp_port_arg, int verbose_arg, char *mp_log_filename_arg,
int exclude_arg)
\end{verbatim}

The arguments are explained below:

\begin{description}

\item[\code{mp\_port\_arg}] is the port on which the \MPA{} is listenning. 

\item[\code{verbose\_arg}] is the flag which says if yes or no the \MPA{} should
be verbose on the messages exchanged.

\item[\code{mp\_log\_filename\_arg}] is file to which \MPA{} will log the messages
exchanged (if \code{NULL} or an empty string is given, no log are done).

\item[\code{exclude\_arg}] is the exclude flag which says if the \MPA{}
should, upon a newer registration, disconnect the former client with the same
name.

\end{description}

The \code{kill-mp} main function has been renamed and is called
\code{kill\_mp\_main} and has the following prototype:

\begin{verbatim}
int kill_mp_main(int mp_port_arg)
\end{verbatim}

The argument is explained below:

\begin{description}

\item[\code{mp\_port\_arg}] is the port on which the \MPA{} you want
to kill is listenning.

\end{description}

Note that all \CPK{} should be started (using \code{taskSpawn}) with the
\code{VX\_FP\_TASK} set. Some operations on floating point in the kernel require
to have this option set up, as to enable the process context switch to save the
FPU registers.

When properly exiting, the \CPK{} and the other \COPRSDE{} program return the
memory allocated. However, this memory does not show as freed in the standard
\code{memShow}, as \CPK{} uses its own allocation mechanism and will keep its
memory allocated for future use. Use the \code{show memory} command of the
\CPK{} (\pxref{OPRS Kernel Miscellaneous Commands}) to see what it keeps for
its own use. 

% If you really want to free all the memory, you can call the
% \code{free\_morecore} function (no argument), but make sure that no \COPRSDE{}
% programs run... or you will loose big.

\section{C++ Relocatables}
\node{C++ Relocatables}
\cpindex{C++ Relocatables}

The relocatables are given in two format, the standard relocatable, and one
which can be linked to some C++ functions, in which case the \code{main} is not
defined to allow a C++ \code{main} to be used (the C++ main performs some
initialization required by C++ functions). The main is then named:

\begin{verbatim}
int oprs_main(int argc, char **argv, char ** envp);
\end{verbatim}

\section{SparcStation}
\node{SparcStation}
\cpindex{SparcStation}

There exist a multi-thread version of the \OPRSS{} available under Solaris 2.4
which do auto accepting of new \CPK{} clients.

\section{Windows95-NT}
\node{Windows95-NT}
\cpindex{Windows95-NT}
\cpindex{Windows 95}
\cpindex{Windows NT}

Under Windows 95... The \code{Term\_Type} symbol \code{FLOAT} and \code{ATOM}
are already defined. Therefore they are renamed \code{TT\_FLOAT} and
\code{TT\_ATOM}.

The various program are statically linked.

Due to the lack of signal timer under Windows, the conditions echanism is
currently more CPU consuming than under Unix.

Only the non graphical program have been ported.

\chapter{Commands Equivalence between the \CPK{} and the \XPK{}}
\node{Commands Equivalence between the OPRS Kernel and the X-OPRS Kernel}
\cpindex{Commands Equivalence between the \CPK{} and the \XPK{}}

Most commands are available in both kernel (the \XPK{} and the CPK{}). The
following tables will allow the user to find out which command correspond to
which menu and vice verca. Note that some commands are not available in both
interface. This is the case for most declaration commands (\code{declare be},
\code{declare ff}, etc) which one usually put in an include file. In any case, you
can always transmit a command from the \OPRSS{} (providing the kernel has
been properly connected to the \OPRSS{} upon startup).

\begin{table}
\begin{center}
\begin{tabular}{||l|l|l|l||}
\hline
Command Name &  Section &       Menu Item & Section \\
\hline
\code{add goal|fact} & \ref{OPRS Kernel Miscellaneous Commands} & Add Fact or Goal & \ref{Add Fact or Goal} \\
\code{conclude \var{expression}} & \ref{OPRS Kernel Database Commands} & None & \\
\code{consult \var{gexpression}} & \ref{OPRS Kernel Database Commands} & Consult Fact Database & \ref{Consult Fact Database} \\
\code{consult applicable op goal|fact} & \ref{OPRS Kernel OP Library Commands} & Consult Applicable OP & \ref{Consult Applicable OP} \\
\code{consult relevant op goal|fact} & \ref{OPRS Kernel OP Library Commands} & Consult Relevant OP & \ref{Consult Relevant OP} \\
\code{declare be \var{predicate}} & \ref{OPRS Kernel Declaration Commands} & None & \\
\code{declare cwp \var{predicate}} & \ref{OPRS Kernel Declaration Commands} & None & \\
\code{declare ff \var{predicate} \var{integer}} & \ref{OPRS Kernel Declaration Commands} & None & \\
\code{declare function \var{function}} & \ref{OPRS Kernel Declaration Commands} & None & \\
\code{declare id symbol} & \ref{OPRS Kernel Miscellaneous Commands} & None & \\
\code{declare op\_predicate \var{predicate}} & \ref{OPRS Kernel Declaration Commands} & None & \\
\code{declare predicate \var{predicate}} & \ref{OPRS Kernel Declaration Commands} & None & \\
\code{delete \var{expression}} & \ref{OPRS Kernel Database Commands} & None & \\
\code{delete op \var{op\_name}} & \ref{OPRS Kernel OP Library Commands} & Delete a particular OP & \ref{Delete a OP} \\
\code{delete opf \var{file\_name}} & \ref{OPRS Kernel OP Library Commands} & Unload OP File... & \ref{X-OPRS Unload OP File} \\
\code{disconnect} & \ref{OPRS Kernel Miscellaneous Commands} & None & \\
\code{echo (g|gt|gm)expression} & \ref{OPRS Kernel Miscellaneous Commands} & None & \\
\code{empty fact db} & \ref{OPRS Kernel Database Commands} & Empty Fact Database & \ref{Empty Fact Database} \\
\code{empty op db} & \ref{OPRS Kernel Loading Commands} & Empty OP Library & \ref{Empty OP Library} \\
\code{help|h|?} & \ref{OPRS Kernel Miscellaneous Commands} & Help & \ref{X-OPRS Help Menu} \\
\code{include \var{file\_name}} & \ref{OPRS Kernel Loading Commands} & Include... & \ref{Include} \\
\hline
\end{tabular}
\end{center}
\caption{Commands Equivalence Between the Kernels (First Part)}
\label{c-x-commands-table1}
\end{table}


\begin{table}
\begin{center}
\begin{tabular}{||l|l|l|l||}
\hline
Command Name &  Section &       Menu Item & Section \\
\hline
\code{list action} & \ref{OPRS Kernel Listing Commands} & List Action & \ref{List Action} \\
\code{list all} & \ref{OPRS Kernel Listing Commands} & List All & \ref{List All} \\
\code{list be} & \ref{OPRS Kernel Listing Commands} & List Basic Event Predicate & \ref{List Basic Event Predicate} \\
\code{list cwp} & \ref{OPRS Kernel Listing Commands} & List Closed World Predicate & \ref{List Closed World Predicate} \\
\code{list evaluable function} & \ref{OPRS Kernel Listing Commands} & List Evaluable Function & \ref{List Evaluable Function} \\
\code{list evaluable predicate} & \ref{OPRS Kernel Listing Commands} & List Evaluable Predicate & \ref{List Evaluable Predicate} \\
\code{list ff} & \ref{OPRS Kernel Listing Commands} & List Functional Fact Predicate & \ref{List Functional Fact Predicate} \\
\code{list function} & \ref{OPRS Kernel Listing Commands} & List Function & \ref{List Function} \\
\code{list op\_predicate} & \ref{OPRS Kernel Listing Commands} & List OP Predicate & \ref{List OP Predicate} \\
\code{list opfs} & \ref{OPRS Kernel OP Library Commands} & List Loaded OP Files & \ref{List Loaded OP Files} \\
\code{list op} & \ref{OPRS Kernel OP Library Commands} & List Loaded OPs & \ref{List Loaded OPs} \\
\code{list predicate} & \ref{OPRS Kernel Listing Commands} & List Predicate & \ref{List Predicate} \\
\code{load db \file{file\_name}} & \ref{OPRS Kernel Database Commands} & Load Database... & \ref{Load Database} \\
\code{load opf op\_graph|\var{file\_name}} & \ref{OPRS Kernel Loading Commands}
& Load OP File... & \ref{Load OP File} \\
\code{q|quit|exit|EOF} & \ref{OPRS Kernel Miscellaneous Commands} & Quit & \ref{X-OPRS Quit} \\
\code{reload opf \file{file\_name}} & \ref{OPRS Kernel Loading Commands} & Reload OP File... & \ref{Reload OP File} \\
\code{require \var{file\_name}} & \ref{OPRS Kernel Loading Commands} & None & \\
\code{reset kernel} & \ref{OPRS Kernel Miscellaneous Commands} & Reset & \ref{Control Button Menu} \\
\code{save db \file{file\_name}} & \ref{OPRS Kernel Database Commands} & Save Database... & \ref{Save Database} \\
\code{send name message} & \ref{OPRS Kernel Miscellaneous Commands} & None & \\
\code{set action on|off} & \ref{OPRS Kernel Compiler/Parser Option Commands} & Compiler Check Action & \ref{OPRS Compiler/Parser Option} \\
\code{set eval post on|off} & \ref{OPRS Kernel Run Option Commands} & Eval On Post & \ref{OPRS Run Option} \\
\code{set function on|off} & \ref{OPRS Kernel Compiler/Parser Option Commands} & Compiler Check Function & \ref{OPRS Compiler/Parser Option} \\
\code{set meta fact op on|off} & \ref{OPRS Kernel Meta Level Option Commands} & Post Meta Fact: (APPLICABLE-OPS-FACT ...) & \ref{OPRS Meta Level Option} \\
\code{set meta fact on|off} & \ref{OPRS Kernel Meta Level Option Commands} & Post Meta Fact: (FACT-INVOKED-OPS ...) & \ref{OPRS Meta Level Option} \\
\code{set meta goal op on|off} & \ref{OPRS Kernel Meta Level Option Commands} & Post Meta Fact: (APPLICABLE-OPS-GOAL ...) & \ref{OPRS Meta Level Option} \\
\code{set meta goal on|off} & \ref{OPRS Kernel Meta Level Option Commands} & Post Meta Fact: (GOAL-INVOKED-OPS ...) & \ref{OPRS Meta Level Option} \\
\code{set meta on|off} & \ref{OPRS Kernel Meta Level Option Commands} & Meta Level & \ref{OPRS Meta Level Option} \\
\code{set parallel intend on|off} & \ref{OPRS Kernel Run Option Commands} & Parallel Intend & \ref{OPRS Run Option} \\
\code{set parallel intention on|off} & \ref{OPRS Kernel Run Option Commands} & Parallel Intention Execution & \ref{OPRS Run Option} \\
\code{set parallel post on|off} & \ref{OPRS Kernel Run Option Commands} & Parallel Goal Posting & \ref{OPRS Run Option} \\
\code{set predicate on|off} & \ref{OPRS Kernel Compiler/Parser Option Commands} & Compiler Check Predicate & \ref{OPRS Compiler/Parser Option} \\
\code{set soak on|off} & \ref{OPRS Kernel Meta Level Option Commands} & Post Meta Fact: (SOAK ...) & \ref{OPRS Meta Level Option} \\
\code{set symbol on|off} & \ref{OPRS Kernel Compiler/Parser Option Commands} & Compiler Check Symbol & \ref{OPRS Compiler/Parser Option} \\
\code{set time\_stamping on|off} & \ref{OPRS Kernel Run Option Commands} & Time Stamping & \ref{OPRS Run Option} \\
\code{show copyright} & \ref{OPRS Kernel Miscellaneous Commands} & None & \\
\code{show db} & \ref{OPRS Kernel Database Commands} & Show Database & \ref{Show Database} \\
\code{show variable} & \ref{OPRS Kernel Miscellaneous Commands} & Show Global
Variables & \ref{Show Global Variables} \\
\code{show intention} & \ref{OPRS Kernel Miscellaneous Commands} & Show Intentions & \ref{Show Intentions} \\
\code{show memory} & \ref{OPRS Kernel Miscellaneous Commands} & Show Memory
Usage & \ref{Show Memory Usage} \\
\code{show op \var{op\_name}} & \ref{OPRS Kernel OP Library Commands} & Display a Particular OP & \ref{Display a Particular OP} \\
\code{show version} & \ref{OPRS Kernel Miscellaneous Commands} & None & \\
\code{stat all} & \ref{OPRS Kernel Miscellaneous Commands} & Stat All Hashtables & \ref{Stat All Hashtables} \\
\code{stat db} & \ref{OPRS Kernel Miscellaneous Commands} & Stat Database Hashtables & \ref{Stat Database Hashtables} \\
\code{stat id} & \ref{OPRS Kernel Miscellaneous Commands} & Stat Symbol Hashtable & \ref{Stat Symbol Hashtable} \\
\hline
\end{tabular}
\end{center}
\caption{Commands Equivalence Between the Kernels (Second Part)}
\label{c-x-commands-table2}
\end{table}


\begin{table}
\begin{center}
\begin{tabular}{||l|l|l|l||}
\hline
Command Name &  Section &       Menu Item & Section \\
\hline
\code{trace all on|off} & \ref{OPRS Kernel Trace Commands} & None & \\
\code{trace applicable op on|off} & \ref{OPRS Kernel Trace Commands} & Soak
& \ref{OPRS Trace} \\
\code{trace db frame on|off} & \ref{OPRS Kernel Trace Commands} & Database
Frames & \ref{OPRS Trace} \\
\code{trace db on|off} & \ref{OPRS Kernel Trace Commands} & Database
operations & \ref{OPRS Trace} \\
\code{trace fact on|off} & \ref{OPRS Kernel Trace Commands} & Fact Posting
& \ref{OPRS Trace} \\
\code{trace feature on|off} & \ref{OPRS Kernel Trace Commands} & Trace &
\ref{OPRS Trace} \\
\code{trace goal on|off} & \ref{OPRS Kernel Trace Commands} & Goal Posting
& \ref{OPRS Trace} \\
\code{trace graphic \var{on|off}} & \ref{OPRS Kernel Trace Commands} & OP
Graphic & \ref{OPRS Trace} \\
\code{trace graphic op \var{op\_name} \var{on|off}} & \ref{OPRS Kernel OP Library Commands} & OP Trace/Step... & \ref{OP Trace/Step} \\
\code{trace graphic opf \var{file\_name} \var{on|off}} & \ref{OPRS Kernel OP Library Commands} & None & \\
\code{trace intend on|off} & \ref{OPRS Kernel Trace Commands} & Intention &
\ref{OPRS Trace} \\
\code{trace load op on|off} & \ref{OPRS Kernel Trace Commands} & OP Compiler
& \ref{OPRS Trace} \\
\code{trace receive on|off} & \ref{OPRS Kernel Trace Commands} & Message
Reception & \ref{OPRS Trace} \\
\code{trace relevant op on|off} & \ref{OPRS Kernel Trace Commands} & Relevant
OP & \ref{OPRS Trace} \\
\code{trace send on|off} & \ref{OPRS Kernel Trace Commands} & Message Sent
& \ref{OPRS Trace} \\
\code{trace suc\_fail on|off} & \ref{OPRS Kernel Trace Commands} & OP Success
Failure & \ref{OPRS Trace} \\
\code{trace intention failure on|off} & \ref{OPRS Kernel Trace Commands} &
Intention Failure & \ref{OPRS Trace} \\
\code{trace text \var{on|off}} & \ref{OPRS Kernel Trace Commands} & OP Text
& \ref{OPRS Trace} \\
\code{trace text op \var{op\_name} \var{on|off}} & \ref{OPRS Kernel OP Library Commands} & OP Text Trace... & \ref{OPRS Trace} \\
\code{trace text opf \var{file\_name} \var{on|off}} & \ref{OPRS Kernel OP Library Commands} & None & \\
\code{trace thread on|off} & \ref{OPRS Kernel Trace Commands} & Thread
Forking/Joining & \ref{OPRS Trace} \\
\code{undeclare be \var{predicate}} & \ref{OPRS Kernel Declaration Commands} & None & \\
\code{unify expression expression} & \ref{OPRS Kernel Miscellaneous Commands} & None & \\
\hline
\end{tabular}
\end{center}
\caption{Commands Equivalence Between the Kernels (Third Part)}
\label{c-x-commands-table3}
\end{table}

\chapter{Default OPs}
\node{Default OPs}
\cpindex{Default OPs}

Some default OPs are provided as is in the standard \COPRS{} distribution.
However, unlike SRI PRS, these default OPs are not loaded by default in all
\CPK{}s. Most of these files comes with a \file{.sym} and \file{.inc}
companion. If this is the case, always load the coresponding \file{.inc} file.



\section{\protect\file{new-default.opf}}
\node{new-default.opf}
\cpindex{new-default.opf}

Here is the list of default procedures provided with the current revision
of \COPRSDE{}. They can be found in the file \file{new-default.opf}. Final user
are encourage to make a copy of this file and select the OP they need and they
want to keep for their application.

\input{new-default}

\section{\protect\file{meta-intended-goal.opf}}
\node{meta-intended-goal.opf}
\cpindex{meta-intended-goal.opf}

Here is the list of meta OP to intend a goal directly distributed with the
current revision of \COPRSDE{}. They can be found in the file
\file{meta-intended-goal.opf}.

\input{meta-intended-goal}

\section{\protect\file{new-meta-ops.opf}}
\node{new-meta-ops.opf}
\cpindex{new-meta-ops.opf}

Here is the list of meta procedures provided with the current revision
of \COPRSDE{}. They can be found in the file \file{new-meta-ops.opf}.

\strong{This file contains more than on meta level OP, do not load all of
them at the same time in one application, you would get a very weird
behavior. Just pick up the one which seems to be appropriate to your
application.}

\input{new-meta-ops}

\section{\protect\file{semaphore.opf}}
\node{semaphore.opf}
\cpindex{semaphore.opf}

Here is the list of procedures, provided with the current revision of
\COPRSDE{}, which implement semaphores. They can be found in the file
\file{semaphore.opf}. Make sure you do not load the \file{.opf} directly but
include the \file{semaphore.inc} instead. Indeed, some important declaration
are made in the  \file{semaphore.sym} file which is loaded by the
\file{semaphore.inc} file.

This OP library provides two type of semaphores: \code{SEM-BASIC} semaphores and
\code{SEM-FIFO} semaphores. For both types, there is a give, a take and a take
with timeout OPs. See the OP documentation for more details.

\input{semaphore}

\chapter{Library and Kernel Functions}
\node{Library and Kernel Functions}
\cpindex{Library and Kernel Functions}

Few libraries (currently two) come with \COPRSDE{}. All of them are needed to
write module to connect to the \MPA{}. However, to write your own kernel, i.e.
to extend relocatable, you may need access to functions defined in these
relocatable (most likely to write evaluable functions, evaluable predicates and
actions). In this chapter, we shall describe the libraries and the kernel
functions which can or must be used by the user.



\section{Kernel Functions}
\node{Kernel Functions}
\cpindex{Kernel Functions}

The \code{oprs-relocatable} and the \code{xoprs-relocatable} files contains
already a number of functions the user can use. Considering that these
functions are already present in the relocatable, there is no special action to
use them (no new file to link).



\subsection{Data Structures and Types Used}
\node{Data Structures and Types Used}
\cpindex{Data Structures and Types Used}

Most data structures are hidden behind opaque pointers. Access functions
are provided when required. However, some structure definitions are given
to the end user.

Here is the definition of the \code{Oprs\_Date} structure:
\begin{verbatim}
#include <sys/time.h>

typedef struct timeval PDate;
\end{verbatim}

Here is the definition of the \code{Term} structure:
\begin{verbatim}
/* Definition of the types used in a Term */
/* Under Windows 95... FLOAT and ATOM are already defined. Therefore
   they are renamed TT_FLOAT and TT_ATOM */
typedef enum {INTEGER, FLOAT, STRING, ATOM, TERM_COMP, VARIABLE, GTEXPRESSION,
      LEXPRESSION, LENV, LISP_LIST, INT_ARRAY, FLOAT_ARRAY, U_POINTER} Term_Type;

/* Definition of a composed Term. For example in (P a (f g h) c), (f g h) is a
 * composed term. */
typedef struct term_comp {
 Function function;
 int  n_arg;
 TermList  terms;
} Term_Comp;

/* Definition of a term (a typed union) */
typedef struct term Term;

struct term {
 Term_Type type;
 union {
   int  intval;
   double doubleval;
   char *string;
   char *id;
   Term_Comp *term;
   Gtexpression *gtexpr;
   Lexpression *lexpr;
   VarList var_list;
   Envar *var;
   L_Car l_car;
   L_List l_list;
   Int_array *int_array;
   Float_array *float_array;
   void *u_pointer;
 }u;
};
\end{verbatim}

\subsection{Important Variables}
\node{Important Variables}
\cpindex{Important Variables}

The following symbols can be found in \file{oprs-type-pub.h}. (\pxref{Special Symbols})

\begin{typevr}{Kernel Variable}{extern Symbol}{nil\_sym} is the \code{nil}
symbol. This is the symbol you should return (in a Term) when an action
fails.
\end{typevr}

\begin{typevr}{Kernel Variable}{extern Symbol}{lisp\_t\_sym} is the \code{T}
symbol.
\end{typevr}

\begin{typevr}{Special Symbols}{extern Symbol}{wait\_sym}
is the \code{:WAIT} symbol as returned by actions when they have not completed
their computation.
\end{typevr}

\begin{typevr}{Kernel Variable}{extern Oprs *}{current\_oprs} is the global
variable which points at the current OPRS kernel. It is used whenever you
want to access some specific modules of the kernel, like the intention
graph.
\end{typevr}

\begin{typevr}{Kernel Variable}{extern Thread\_Intention\_Block *}{current\_tib} is the global
variable which points at the current tib.
\end{typevr}

\begin{typevr}{Kernel Function}{extern Intention*}{current\_intention} is the
current intention of the intention graph \var{ig} (\code{NULL} if not applicable).
\end{typevr}

\begin{typevr}{Kernel Variable}{extern long}{main\_loop\_pool\_sec} is the global
variable which points at the number of seconds the \CPK{} will wait before
checking the sleeping conditions of the sleeping intentions. It is used in
conjunction with \code{main\_loop\_pool\_usec} presented below. It is defined
in \file{default-user-external.h}.
\end{typevr}

\begin{typevr}{Kernel Variable}{extern long}{main\_loop\_pool\_usec} is the
global variable which points at the number of micro seconds the \CPK{}
will wait
before checking the sleeping conditions of the sleeping intentions. It is
used
in conjunction with \code{main\_loop\_pool\_sec} presented above. It is
defined in
\file{default-user-external.h}.
\end{typevr}

Example of use (withdrawn from \file{default-user-external.c}):

\begin{verbatim}
void start_kernel_user_hook()
{
 intention_list_sort_predicate = &my_intention_list_sort;
 main_loop_pool_sec = 0L;
 main_loop_pool_usec = 10000L; /* 10 milliseconds */

}
\end{verbatim}

\begin{typevr}{Kernel Variable}{extern Widget}{x\_oprs\_top\_level\_widget} is the
variable which points at the \XOPRS{} top level widget. It can be used by the
user to hook its own Xt widget in the widget tree, for its own interface.
It is defined in \file{xp-main-pub.h}.
\end{typevr}


\subsection{Important Constants}
\node{Important Constants}
\cpindex{Important Constants}

The 2 following constants can be found in \file{constant-pub.h}.
\begin{verbatim}
#define TRUE 1
#define FALSE 0
\end{verbatim}

\subsection{Oprs Manipulation Functions}
\node{Oprs Manipulation Functions}
\cpindex{Oprs Manipulation Functions}

These function prototypes can be found in the file \file{oprs\_f-pub.h}.

\begin{typefn}{Kernel Function}{Intention\_Graph *}{oprs\_intention\_graph}{(Oprs
*\var{oprs})}
returns a pointer to the intention graph of the OPRS passed as an argument
(most
likely \var{current\_oprs}).
\end{typefn}

\subsection{Array Manipulation Functions}
\node{Array Manipulation Functions}
\cpindex{Array Manipulation Functions}

These function prototypes can be found in the file \file{oprs-array\_f-pub.h}.

\begin{typefn}{Kernel Function}{Term *}{make\_float\_array\_from\_array}{(int \var{size},
double *\var{array})}
returns a \code{Term *} of type \code{FLOAT\_ARRAY} containing the \var{array} of size
\var{size} passed in argument.
\end{typefn}

\begin{typefn}{Kernel Function}{Term *}{make\_int\_array\_from\_array}{(int \var{size}, int
*\var{array})}
returns a \code{Term *} of type \code{INT\_ARRAY} containing the \var{array} of size
\var{size} passed in argument.
\end{typefn}

\begin{typefn}{Kernel Function}{double *}{get\_array\_from\_float\_array}{(Term *\var{t})}
returns a double * pointer to the array of the \code{FLOAT\_ARRAY} contained in the
term \var{t}.
\end{typefn}

\begin{typefn}{Kernel Function}{int *}{get\_array\_from\_int\_array}{(Term *\var{t})}
returns a int * pointer to the array of the \code{INT\_ARRAY} contained in the
term \var{t}.
\end{typefn}

\begin{typefn}{Kernel Function}{int }{get\_float\_array\_size}{(Term *\var{t})}
returns an int which is the size of the \code{FLOAT\_ARRAY} contained in the
term \var{t}.
\end{typefn}

\begin{typefn}{Kernel Function}{int }{get\_int\_array\_size}{(Term *\var{t})}
returns an int which is the size of the \code{INT\_ARRAY} contained in the
term \var{t}.
\end{typefn}

\subsection{Fact and Goal Manipulation Functions}
\node{Fact and Goal Manipulation Functions}
\cpindex{Fact and Goal Manipulation Functions}

These function prototypes can be found in the file
\file{fact-goal\_f-pub.h}.

\begin{typefn}{Kernel Function}{void}{fprint\_goal}{(FILE *\var{file}, Goal*
\var{goal})}
prints the \var{goal} in the file \var{file}.
\end{typefn}

\begin{typefn}{Kernel Function}{void}{print\_goal}{(Goal *\var{goal})}
prints the \var{goal} on stdout.
\end{typefn}

\begin{typefn}{Kernel Function}{void}{fprint\_fact}{(FILE *\var{file}, Fact
*\var{fact})}
prints the \var{fact} in the file \var{file}.
\end{typefn}

\begin{typefn}{Kernel Function}{void}{print\_fact}{(Fact *\var{fact})}
prints the \var{fact} on stdout.
\end{typefn}

\begin{typefn}{Kernel Function}{Oprs\_Date}{fact\_soak}{(Fact *\var{fact})}
returns the date at which this \var{fact} was taken into account by
the soak mechanism. This is only meaningful if Time Stamping has been allowed
(\pxref{OPRS Kernel Run Option Commands}, \code{set time\_stamping on|off}). 
\end{typefn}

\begin{typefn}{Kernel Function}{PString}{fact\_sender}{(Fact *\var{fact})}
returns the name of the sender if this fact was a message, NULL if it is an
internally generated fact. This function can in fact be used to check if
the
fact is an external message or a fact.
\end{typefn}

\begin{typefn}{Kernel Function}{Oprs\_Date}{fact\_soak}{(Fact *\var{fact})}
returns the date at which this \var{fact} was taken into account by
the soak mechanism. This is only meaningful if Time Stamping has been allowed
(\pxref{OPRS Kernel Run Option Commands}, \code{set time\_stamping on|off}). 
\end{typefn}

\begin{typefn}{Kernel Function}{Oprs\_Date}{fact\_creation}{(Fact *\var{fact})}
returns the date at which this \var{fact} was created.This is only meaningful
if Time Stamping has been allowed 
(\pxref{OPRS Kernel Run Option Commands}, \code{set time\_stamping on|off}). 
\end{typefn}

\begin{typefn}{Kernel Function}{Oprs\_Date}{fact\_reception}{(Fact *\var{fact})}
returns the date at which this \var{fact} was received by the kernel.This is
only meaningful if Time Stamping has been allowed 
(\pxref{OPRS Kernel Run Option Commands}, \code{set time\_stamping on|off}). 
\end{typefn}

\begin{typefn}{Kernel Function}{Oprs\_Date}{fact\_response}{(Fact *\var{fact})}
returns the date at which all the applicable OPs have completed because
of this \var{fact} (if no OP was applicable, it returns a zero date). This is
only meaningful if Time Stamping has been allowed (\pxref{OPRS Kernel Run Option Commands}, \code{set time\_stamping on|off}).
\end{typefn}

\begin{typefn}{Kernel Function}{Oprs\_Date}{goal\_soak}{(Goal *\var{goal})}
returns the date at which this \var{goal} was taken into account by
the soak mechanism. This is only meaningful if Time Stamping has been allowed
(\pxref{OPRS Kernel Run Option Commands}, \code{set time\_stamping on|off}).
\end{typefn}

\begin{typefn}{Kernel Function}{Oprs\_Date}{goal\_creation}{(Goal *\var{goal})}
returns the date at which this \var{goal} was created. This is
only meaningful if Time Stamping has been allowed (\pxref{OPRS Kernel Run Option Commands}, \code{set time\_stamping on|off}). 
\end{typefn}

\begin{typefn}{Kernel Function}{Oprs\_Date}{goal\_reception}{(Goal *\var{goal})}
returns the date at which this \var{goal} was received by the kernel. This is
only meaningful if Time Stamping has been allowed (\pxref{OPRS Kernel Run Option  Commands}, \code{set time\_stamping on|off}).
\end{typefn}

\begin{typefn}{Kernel Function}{Oprs\_Date}{goal\_response}{(Goal *\var{goal})}
returns the date at which all the applicable OPs have completed because
of this \var{goal} (if no OP was applicable, it returns a zero date). This is
only meaningful if Time Stamping has been allowed (\pxref{OPRS Kernel Run Option Commands}, \code{set time\_stamping on|off}).
\end{typefn}

\subsection{Fact Posting Functions}
\node{Fact Posting Functions}
\cpindex{Fact Posting Functions}

The kernel provide a number of functions to allow the user to add a new fact
from its own code. This mechanism can be very useful on a VxWorks
implementation as it will allow the user to post/add a fact from a different
process (one the same board though). Under traditional Unix system, these
functions work too but can be more simply emulated using the
\code{send\_command\_to\_parser} function (\pxref{Miscellaneous Kernel Functions}).

\strong{One should make sure that a \CPK{} is running and is alive before
calling these functions. If no \CPK{} is running, these functions will most
likely crash the calling process.}

These function prototypes can be found in the file
\file{oprs\_f-pub.h}.

\begin{typefn}{Kernel Function}{void }{add\_external\_fact}{(char *\var{predicat}, TermList
\var{param\_list})} This function will add the fact built with \var{predicat}
and \var{param\_list} in the \code{current\_oprs} kernel. The \var{param\_list}
should be used only once. The \var{predicat} argument can be freed after the
call. Under VXWOrks, this function can be called from another process. Proper
variable modification is protected by mutex semaphores.
\end{typefn}

\begin{typefn}{Kernel Function}{TermList}{make\_external\_term\_list}{(int \var{nb\_arg},
...)}  Build and return a "use once only" TermList containing \var{nb\_arg}
elements built with the subsequent paire \code{TermType}, \code {Term *} in the
argument list. See the example below. All the element are duplicated,
i.e. strings, symbols are appropriately copied, except for those produce with
\code{make\_external\_term\_comp} and \code{make\_external\_lisp\_list}. You should
therefore appropriately free the objects passed to this function.
\end{typefn}

\begin{typefn}{Kernel Function}{Term\_Comp *}{make\_external\_term\_comp}{(char
*\var{function}, TermList \var{param\_list})} This function will build a "use
once only" TermComp built with \var{function} and \var{param\_list}. The
\var{param\_list} should be used only once. The \var{function} argument can be
freed after the call. This function is usually used inside
\code{make\_external\_term\_list}, in which case its result should not be freed.
\end{typefn}

\begin{typefn}{Kernel Function}{L\_List}{make\_external\_lisp\_list}{(TermList
\var{param\_list})}This function will build a "use once only" L\_List built with
\var{param\_list}. The \var{param\_list} should be used only once. This function
is usually used inside \code{make\_external\_term\_list}, in which case its
result should not be freed.
\end{typefn}

Here is a complete example to illustrate the use of the functions presented
above: 
\begin{verbatim}
add_external_fact(
     "boo", 
     make_external_term_list(
          6, 
          FLOAT, 3.1415, 
          INTEGER, 5,
          U_POINTER, 0x123456,
          ATOM, "foobar",
          STRING, "This is a string in the fact",
          LISP_LIST, make_external_lisp_list(
               make_external_term_list(
                    2, 
                    FLOAT, 0.0, 
                    ATOM, "second")),
          TERM_COMP, make_external_term_comp(
               "bar",
               make_external_term_list(
                    2,
                    INTEGER, -1,
                    ATOM, "atom"))));
\end{verbatim}

a call to the function above will post the fact:\*
\code{(BOO 3.1415 5 0x123456 FOOBAR "This is a string in the fact" (. 0.0
SECOND. ) (BAR -1 ATOM))}


\subsection{Intention Manipulation Functions}
\node{Intention Manipulation Functions}
\cpindex{Intention Manipulation Functions}

These function prototypes can be found in the file
\file{intention\_f-pub.h}.

\begin{typefn}{Kernel Function}{void}{fprint\_intention}{(FILE *\var{file},
Intention *\var{intention})}
prints the \var{intention} in \var{file}.
\end{typefn}

\begin{typefn}{Kernel Function}{int}{intention\_priority}{(Intention
*\var{intention})}
returns the priority of \var{intention}.
\end{typefn}

\begin{typefn}{Kernel Function}{Fact *}{intention\_fact}{(Intention
*\var{intention})}
returns the fact which caused this \var{intention} to arise
(\code{NULL} if not applicable).
\end{typefn}

\begin{typefn}{Kernel Function}{PBoolean}{action\_first\_call}{()}
when executed in a C action code, returns \code{TRUE} if this is the first time
this action is called, \code{FALSE} otherwise. This is used for action slicing
to distinguish the first call from the subsequent call.
\end{typefn}

\begin{typefn}{Kernel Function}{int}{action\_number\_called}{()}
when executed in a C action code, returns the number of time this action has
been called in this context (it is equal to 0 in the first call) . This is used
for action slicing to distinguish subsequent calls.
\end{typefn}

\begin{typefn}{Kernel Function}{Goal *}{intention\_goal}{(Intention
*\var{intention})} 
returns the goal which caused this \var{intention} to arise
(\code{NULL} if not applicable).
\end{typefn}

\begin{typefn}{Kernel Function}{Op\_Instance *} {intention\_bottom\_op\_instance}
{(Intention *\var{intention})}
returns the bottom OP instance of this intention stack,
\code{NULL} if the stack is empty.
\end{typefn}

\subsection{OP Instance Manipulation Functions}
\node{OP Instance Manipulation Functions}
\cpindex{OP Instance Manipulation Functions}

These function prototypes can be found in the file
\file{op-instance\_f-pub.h}.

\begin{typefn}{Kernel Function}{void}{fprint\_op\_instance}{(FILE *\var{f},
Op\_Instance *\var{opi})}
prints the \var{opi} in \var{file}.
\end{typefn}

\begin{typefn}{Kernel Function}{Op\_Structure *}{op\_instance\_op}{(Op\_Instance
*\var{opi})}
returns the Op\_Structure pointed by this \var{opi}, i.e. the OP from
which this OP\_instance is an instance...
\end{typefn}

\begin{typefn}{Kernel Function}{Goal *}{op\_instance\_goal}{(Op\_Instance
*\var{opi})}
returns the goal which caused this \var{opi} to arise
(\code{NULL} if not applicable).
\end{typefn}

\begin{typefn}{Kernel Function}{Fact *}{op\_instance\_fact}{(Op\_Instance
*\var{opi})}
returns the fact which caused this \var{opi} to arise
(\code{NULL} if not applicable).
\end{typefn}

\subsection{OP Manipulation Functions}
\node{OP Manipulation Functions}
\cpindex{OP Manipulation Functions}
\cpindex{OP Structure Manipulation Functions}

OPs and OP structures are the same type of object. When we say OP, we mean
OP structure, this is explicited to avoid any confusion with OP
instances (which are instances of applicable OP structures).

These function prototypes can be found in the file
\file{op-structure\_f-pub.h}.

\begin{typefn}{Kernel Function}{PString}{op\_name}{(Op\_Structure *\var{op})}
returns the name of the \var{op}. This function can be used to build an
evaluable predicate which indicates (in a Meta OP) if a OP instance is an
instance of itself...
\end{typefn}

\begin{typefn}{Kernel Function}{PString}{op\_file\_name}{(Op\_Structure
*\var{op})}
returns the file name of the \var{op}.
\end{typefn}

\subsection{Intention Graph Manipulation Functions}
\node{Intention Graph Manipulation Functions}
\cpindex{Intention Graph Manipulation Functions}

These function prototypes can be found in the file
\file{int-graph\_f-pub.h}.

\begin{typefn}{Kernel Function}{Intention\_List}{intention\_graph\_roots}{(Intention\_Graph
*\var{ig})}
returns the current roots of the intention graph \var{ig}.
\end{typefn}

\subsection{Allocation Functions}
\node{Allocation Functions}
\cpindex{Allocation Functions}

\strong{Memory Allocation Rules} The standard allocation functions are of
course available for the user to program its evaluable functions, predicates
and actions (as well as all the linked code to the kernels). However, all the
objects handled by the kernel itself (mostly terms and their components) must
be allocated/freed using the following macros. Objects pointed by USER\_POINTER
can be allocated/freed as the user want. In any situation, never mix the
different allocation mechanism (in particular on systems where the standard
memory allocator is used: VxWorks, Purified version, etc). Mixing them will
lead to allocation/free error.

\begin{typefn}{Kernel Macro}{void *}{OPRS\_MALLOC} {(size\_t \var{nBytes})}
This macro allocate \var{nBytes} of memory and return a \code{void *}pointer to
it. The allocated memory must be freed with \code{OPRS\_FREE}.  \strong{Caution:}
All objects (and their components) returned to the kernel must be allocated
using the
\code{OPRS\_MALLOC} macro or the following macro \code{MAKE\_OBJECT}.
\begin{verbatim}
Term *toto_eval_func(TermList terms)
{
     Term *res;

     res = MAKE_OBJECT(Term);

     res->type = STRING;
     res->u.string = (char *)OPRS_MALLOC(20);

     ....

     return res;
}
\end{verbatim}
\end{typefn}

\begin{typefn}{Kernel Macro}{type *}{MAKE\_OBJECT} {(type)}
This macro allocates memory to store an object of type \code{type}, and
returns a pointer to this memory block.

The parameter is a type specification. The macro uses it to allocate
the right amount of memory and to generate the proper recasting
instructions to make both the \i{C}-compiler and \i{lint(1)} happy.

\strong{Caution:} Do not allocate using your own memory allocator
(providing it is linked in the \CPK). Always use the \code{MAKE\_OBJECT}
macro for your memory use.

\begin{verbatim}
Term *toto_eval_func(TermList terms)
{
 Term *t1, *res;

 res = MAKE_OBJECT(Term);
....
 return res;
}
\end{verbatim}
\end{typefn}

\begin{typefn}{Kernel Macro}{void}{OPRS\_FREE} {(void *\var{object})}
This macro frees the memory allocated with \code{MAKE\_OBJECT}.
\strong{Caution:} Do not attempt to free memory you did not allocate... You
should only free temporary variables you created for your own use.
\begin{verbatim}
Term *toto_eval_func(TermList terms)
{
 Term *temporary, *res;

 res = MAKE_OBJECT(Term);
 temporary = MAKE_OBJECT(Term);
....
 OPRS_FREE(temporary);
 return res;
}
\end{verbatim}
\end{typefn}

The prototypes of the following functions are defined in
\file{oprs-type\_f-pub.h}.

\begin{typefn}{Kernel Function}{char *}{make\_atom}{(char *atom)}
return an atom which can then be stored in a built Term.
\end{typefn}

\begin{typefn}{Kernel Function}{char *}{find\_atom}{(char *atom)}
it does exactly like \code{make\_atom} but warn you if the symbol has not been
declared previously.
\end{typefn}

\begin{typefn}{Kernel Function}{char *}{declare\_atom}{(char *atom)}
it does exactly like \code{make\_atom} but does not warn you if the symbol had
not been declared previously.
\end{typefn}

\begin{typefn}{Kernel Function}{Term *}{build\_string}{(char *string)} build a Term
containing a copy of the (\code{STRING}) string. It returns the pointer to this
Term. 
\end{typefn}

\begin{typefn}{Kernel Function}{Term *}{build\_integer}{(int i)} build a Term
containing the integer (\code{INTEGER}) i. It returns the pointer to this Term.
\end{typefn}

\begin{typefn}{Kernel Function}{Term *}{build\_float}{(double i)} build a Term
containing the double (\code{FLOAT}) i. It returns the pointer to this Term.
\end{typefn}

\begin{typefn}{Kernel Function}{Term *}{build\_l\_list}{(L\_List l)} build a Term
containing the L\_List (\code{LISP\_LIST}) i. It returns the pointer to this Term.
\end{typefn}

\begin{typefn}{Kernel Function}{Term *}{build\_c\_list}{(OPRS\_LIST l)} build a Term
containing the OPRS\_LIST (\code{OPRS\_LIST}) l. It returns the pointer to this Term.
\end{typefn}

\begin{typefn}{Kernel Function}{Term *}{build\_qstring}{(char *i)} build a Term
containing the string (\code{STRING}) i. It returns the pointer to this Term.
\end{typefn}

\begin{typefn}{Kernel Function}{Term *}{build\_id}{(char *id)} build a Term
containing the id (\code{ATOM}) id. It returns the pointer to this Term.
\end{typefn}

\begin{typefn}{Kernel Function}{Term *}{build\_t}{(void)} build a Term
containing the (\code{T}) id. It returns the pointer to this Term.
\end{typefn}

\begin{typefn}{Kernel Function}{Term *}{build\_nil}{(void)} build a Term
containing the (\code{NIL}) id. It returns the pointer to this Term.
\end{typefn}

\begin{typefn}{Kernel Function}{void }{free\_term}{(Term *term)} Free a Term.
\end{typefn}

\subsection{LISP\_LIST Manipulation Functions}
\node{LISPLIST Manipulation Functions}
\cpindex{LISP\_LIST Manipulation Functions}

These function prototypes can be found in the file
\file{lisp-list\_f-pub.h}.

\begin{typefn}{Kernel Function}{L\_Car}{l\_car}{(L\_List \var{l})}
returns the car of an L\_List.
\end{typefn}

\begin{typefn}{Kernel Function}{L\_List}{l\_cdr}{(L\_List \var{l})}
returns the cdr of an L\_List.
\end{typefn}

\begin{typefn}{Kernel Function}{L\_List}{l\_cons}{(L\_Car \var{car}, L\_List
\var{cdr})}
returns a new L\_List, created by consing the car with the cons. The car is
duplicated, but not the cdr.
\end{typefn}

\begin{typefn}{Kernel Function}{L\_List}{l\_add\_to\_tail}{(L\_List \var{list},
L\_Car \var{car})}
adds a car at the end of an L\_List. The car is duplicated, not the list.
\end{typefn}

\begin{typefn}{Kernel Function}{int}{l\_length}{(L\_List \var{l})}
returns and int, the length of the L\_List.
\end{typefn}

\begin{typefn}{Kernel Function}{L\_Car}{l\_nth}{(L\_List \var{l}, int \var{i})}
returns the nth L\_Car element of the L\_List.
\end{typefn}

\begin{typefn}{Kernel Function}{L\_Car}{make\_l\_car\_from\_term}{(Term *\var{t})}
returns a L\_Car containing the copy of the term t.
\end{typefn}

\begin{typefn}{Kernel Function}{Term *}{get\_term\_from\_l\_car}{(L\_Car \var{l})}
returns a pointer to the term contained in the L\_Car.
\end{typefn}

\begin{typefn}{Kernel Function}{L\_List}{copy\_l\_list}{(L\_List \var{l})}
returns a copy of the L\_List.
\end{typefn}

\subsection{Miscellaneous Kernel Functions}
\node{Miscellaneous Kernel Functions}
\cpindex{Miscellaneous Kernel Functions}

\begin{typefn}{Kernel Function}{void}{send\_command\_to\_parser}{(PString \var{command})}
is used to get the kernel to execute a particular command. This can be used to
conclude something in the database, or adding a fact, or any command parsable
by the parser. You should make sure that the command has the proper syntax, or
you may hang the kernel for ever... It is strongly advised to terminate the
command with a new line.
\end{typefn}


\section{Registration and Communication Functions, \protect\file{libmp.a}}
\node{Registration and Communication Functions libmp.a}
\cpindex{Registration and Communication Functions, \file{libmp.a}}

\xref{How to Connect from an External Module}, for a description of the
functions available to connect to the \MPA{}. \xref{Messages Format}, for a
description of the functions available to exchange messages with the \MPA{}.

\section{\protect\file{liblist.a} library}
\node{liblist.a library}
\cpindex{liblist.a library}

The \file{liblist.a} library is a very general library which contains the
functions dealing with lists of objects, among other things. It is
heavily used by the various parts of the \COPRSDE{} and is needed if you
plan to write your own evaluable predicates or functions (to parse the
argument list for example which is a \code{TermList}).

We present here a subset of this library. We do advise the user to
stick with the functions we present below, and not to use
undocumented functions from this library.

The \i{listPack} library provides an abstract data type called a
\code{OPRS\_LIST}, and a complete set of operations to manipulate objects of that
type. In its simplest form, a \code{OPRS\_LIST} is an ordered collection of
objects which the user creates. \i{listPack} functions allow the user to
add objects at any point in a \code{OPRS\_LIST}, to retrieve objects from
\code{OPRS\_LIST}s, to delete objects from \code{OPRS\_LIST}s, and to apply user
functions to
\code{OPRS\_LIST}s.  Furthermore, the notion of current object exists, and
each \code{OPRS\_LIST} maintains the required information a step forward and
backward
from the current object.

\i{listPack} provides the single data type \code{OPRS\_LIST}. No operation on
variables of type \code{OPRS\_LIST} is allowed, except for assigning them the
values returned by \i{listPack} functions and for passing them as
parameters to \i{listPack} functions. However, it is permitted (and safe)
to
assign variables of type \code{OPRS\_LIST} to other variables of type
\code{OPRS\_LIST}.



\subsection{Creating Lists}
\node{Creating Lists}

Before the user may place objects in a \code{OPRS\_LIST}, he must create it with
the function - \code{make\_list()} - which returns a pointer to the newly
created
 list.

\begin{typefn}{listPack Function}{OPRS\_LIST}{make\_list}{()}
simply creates a new list and returns a pointer to it. The value returned
by
\code{make\_list()}
\i{must} be saved by the caller in order to use that \code{OPRS\_LIST} with
subsequent \i{listPack} functions.
\end{typefn}

\subsection{Destroying Lists}
\node{Destroying Lists}

When the user no longer needs a \code{OPRS\_LIST}, he may destroy it.

\begin{typefn}{listPack Macro}{void}{FREE\_OPRS\_LIST}{(OPRS\_LIST \var{list})}
deallocates the storage needed for a \code{OPRS\_LIST}. After a
\code{OPRS\_LIST} has been passed to \code{FREE\_OPRS\_LIST} it may not be used in any
subsequent \i{listPack} function. \var{list} is the list which is to be
freed.
\end{typefn}

\subsection{Placing Elements in a List}
\node{Placing Elements in a List}

\i{listPack} provides many different ways of inserting elements into
\code{OPRS\_LIST}s. The one to use depends on where in the list the user wishes
to place the element.

Conventionally, every list has two ends called the \var{head} and the
\var{tail}. An element may be added at either end and thus becomes
the new head or tail.

Alternatively, each list may be seen as an array of elements. The head
is the first element, the tail is the \code{list\_length(list)-th}.
Elements may be added in the middle of a list by specifying the position
before which they are to be inserted. The position is then simply the
distance from the head of the list.

\begin{typefn}{listPack Function}{OPRS\_NODE} {add\_to\_head} {(OPRS\_LIST \var{list}, OPRS\_NODE
\var{element})}
The element is added before the head of the specified list, thus becoming the
new head. For convenience, \code{add\_to\_head()} returns (\code{OPRS\_NODE}
\var{element}). This allows the user to perform operations like
\begin{verbatim}
save = add_to_head( list_1, read_next_element() );
\end{verbatim}
or anything else.  \i{listPack} tries to impose
no hidden restriction on the use of lists.
\end{typefn}

\begin{typefn}{listPack Function}{OPRS\_NODE} {add\_to\_tail} {(OPRS\_LIST \var{list}, OPRS\_NODE
\var{element})}
The element is added after the tail of the specified list and thus becomes the
new tail. For convenience, \code{add\_to\_tail()} returns (\code{OPRS\_NODE}
\var{element}).
\end{typefn}

\begin{typefn}{listPack Function}{OPRS\_NODE} {insert\_list\_pos} {( OPRS\_LIST \var{list},
OPRS\_NODE \var{element}, int \var{pos})}
\code{(OPRS\_NODE \var{element})} is inserted into \code{(OPRS\_LIST \var{list})}
before an element number \code{int \var{pos}}. The positions of all elements
subsequent to the inserted element are thus incremented.

If \var{pos} is less than or equal to one, it is inserted before the head of
\var{list}. If \var{pos} is greater than \code{list\_length( list )}, it is
added after the tail.

For convenience, \code{insert\_list\_pos()} returns \code{(OPRS\_NODE
\var{element})}.
\end{typefn}

\begin{typefn}{listPack Function}{int} {replace\_list} {( OPRS\_LIST \var{list}, OPRS\_NODE
\var{old}, OPRS\_NODE \var{new})}
The first appearance of \code{OPRS\_NODE \var{old}} in the specified list is replaced
with \code{(OPRS\_NODE \var{new})}. No other part of \code{(OPRS\_LIST
\var{list})} is affected. \code{replace\_list()} returns a boolean
\code{TRUE} if the replacement succeeded and a boolean
\code{FALSE} otherwise. Thus, replacing all occurrences of an element in a
list may be done via
\begin{verbatim}
while ( replace_list( list, old, new ) )    /* null body */ ;
\end{verbatim}
\end{typefn}

\begin{typefn}{listPack Function} {OPRS\_LIST} {append\_list} {( OPRS\_LIST \var{first},
OPRS\_LIST \var{second})}
\code{append\_list()} adds all the elements of \code{(OPRS\_LIST \var{second})}
after the tail of \code{(OPRS\_LIST \var{first})}.

The algorithm used to append lists \emph{destroys the second list}.
Therefore, \code{(OPRS\_LIST \var{second})} may not be used in subsequent
\i{listPack} functions. It may be thus desirable to use
\code{append\_list()} as:
\begin{verbatim}
append_list( first, copy_list(second, flag) );   /* append a copy of list 2 */
\end{verbatim}
\code{append\_list()} returns \code{(OPRS\_LIST \var{first})} to the caller.
This allows the following convenient operation:
\begin{verbatim}
new_list = append_list( copy_list(first, flag), l2 );    /* l2 is destroyed! */
\end{verbatim}
Of course, these are also valid uses of \code{append\_list()}:
\begin{verbatim}
new_list = append_list( copy_list(first, flag), copy_list(second, flag) );
append_list( first, second );
append_list( first, append_list( second, third ) );
\end{verbatim}
\end{typefn}

\subsection{Examining the Elements of a List}
\node{Examining the Elements of a List}

Once elements have been placed in a \code{OPRS\_LIST}, they are accessible to the
user through any pointer pointing to them that the user has maintained. The
user does not often want to maintain his own pointers, and relies on
\i{listPack} functions to return pointers to the elements in a list.

\begin{typefn}{listPack Function} {OPRS\_NODE} {get\_list\_head} {(OPRS\_LIST \var{list})}

\code{get\_list\_head()} returns a pointer to the first element in the
specified list.  If the \code{OPRS\_LIST} is empty, \code{NULL} is returned.
\end{typefn}

\begin{typefn}{listPack Function} {OPRS\_NODE} {get\_list\_tail} {(OPRS\_LIST \var{list})}

\code{get\_list\_tail()} returns a pointer to the last element in the
specified list.  If the \code{OPRS\_LIST} is empty, \code{NULL} is returned.
\end{typefn}

\begin{typefn}{listPack Function} {OPRS\_NODE} {get\_list\_pos} {(OPRS\_LIST \var{list}, int
\var{pos})}\code{get\_list\_pos()} returns a pointer to the \code{pos-th}
elements in the specified list. If the \code{OPRS\_LIST} is empty, \var{pos} is
less than one, or \var{pos} is greater than the length of the list,
\code{NULL} is returned.
\end{typefn}

\subsection{Removing Elements from Lists }
\node{Removing Elements from Lists}

Elements may be removed from \code{OPRS\_LIST}s with the appropriate \i{listPack}
functions. All these functions return a pointer to the element removed
from the list, so the user does not need to save the pointer before
removing the
element.

\begin{typefn}{listPack Function} {OPRS\_NODE} {get\_from\_head} {(OPRS\_LIST \var{list})}
\code{get\_from\_head()} returns a pointer to the first element in the
specified list. The element is removed from the list, and the next
element becomes the new head. If the \code{OPRS\_LIST} is empty, \code{NULL}
is returned.
\end{typefn}

\begin{typefn}{listPack Function} {OPRS\_NODE} {get\_from\_tail} {(OPRS\_LIST \var{list})}
\code{get\_from\_tail()} returns a pointer to the last element in the
specified list. The element is removed from the list, and the previous
element becomes the new tail. If the \code{OPRS\_LIST} is empty, \code{NULL}
is returned.
\end{typefn}

\begin{typefn}{listPack Function} {OPRS\_NODE} {delete\_list\_pos} {(OPRS\_LIST \var{list},
int \var{pos})}
\code{delete\_list\_pos()} returns a pointer to the \code{pos-th} element in
the specified list. The element is removed from the list, and the position of
subsequent elements is decremented. If the \code{OPRS\_LIST} is empty, \var{pos} is
less than one, or \var{pos} is greater than the length of the list, then
\code{NULL} is returned.
\end{typefn}

\begin{typefn}{listPack Function} {OPRS\_NODE} {delete\_list\_node} {(OPRS\_LIST \var{list},
OPRS\_NODE \var{element})}
The first appearance of \code{(OPRS\_NODE} \var{element}) in the specified
\code{OPRS\_LIST} is deleted. For convenience, \code{(OPRS\_NODE} \var{element}) is
returned to the caller, but if the list is empty or \code{(OPRS\_NODE
\var{element})} is not present, then \code{NULL} is returned.
\end{typefn}

\subsection{Examining the Lists}
\node{Examining the Lists}

\i{listPack} provides several functions which return boolean values
based on inquiries being done upon \code{OPRS\_LIST}s.

\begin{typefn}{listPack Function}{int} {list\_length} {(OPRS\_LIST \var{list})}
This function returns the number of elements in the specified list.
\end{typefn}

\begin{typefn}{listPack Function}{int} {in\_list} {( OPRS\_LIST \var{list}, OPRS\_NODE
\var{element})}
This function returns \code{0} (boolean \code{FALSE}) if \code{(OPRS\_NODE
\var{element})} does not appear in the specified \code{(OPRS\_LIST \var{list})}.
Otherwise, it returns the position of the element in the list (in the range
1..\code{list\_length(list)}).
\end{typefn}

\begin{typefn}{listPack Function}{int} {list\_empty} {(OPRS\_LIST \var{list})}
This function returns boolean \code{TRUE} if the specified \code{OPRS\_LIST} has no
element, it returns boolean \code{FALSE} otherwise.
\end{typefn}

\begin{typefn}{listPack Function}{int} {first\_in\_list} {(OPRS\_LIST \var{list}, OPRS\_NODE
\var{element})}
It returns \code{TRUE} if the head of the \code{OPRS\_LIST} is the specified
\code{(OPRS\_NODE \var{element})}.
\end{typefn}

\begin{typefn}{listPack Function}{int} {last\_in\_list} {(OPRS\_LIST \var{list}, OPRS\_NODE
\var{element})}
It returns \code{TRUE} if the tail of the \code{OPRS\_LIST} is the specified
\code{(OPRS\_NODE \var{element})}.
\end{typefn}

\subsection{Applying Functions to Lists}
\node{Applying Functions to Lists}

The \i{C} language provides the capability to pass functions as parameters to
other functions. This is utilized by \i{listPack} to allow users to write list
processing functions. Instead of having the user walk through each
\code{OPRS\_LIST} (i.e. with get\_list\_pos()) and pass each returned element to
some function, the function is passed to \i{listPack} and calls the user
function.

List processing always begins at the head of the \code{OPRS\_LIST}, and proceeds
towards the tail.

\strong{Warning:} It is extremely dangerous to add or remove elements from
a \code{OPRS\_LIST} \i{while a function is being applied to it}. \i{listPack} can
handle most of the common cases that occur when this happens, however such code
should be rigorously exercised.

\begin{typefn}{listPack Function} {int} {for\_all\_list} {( OPRS\_LIST \var{list},
(OPRS\_NODE) ptr, (PFI) func)}
The user may have some functions like \code{(int) func( (OPRS\_NODE) ptr, OPRS\_NODE
\var{element})} he wishes to call with every element in the \code{(OPRS\_LIST
\var{list})}. \code{for\_all\_list()} invokes \code{func()} for the user (as
often as necessary), adds up the returned values, and returns the total value
to the user. \code{(OPRS\_NODE \var{ptr})} is a pointer to whatever the user wishes
to pass to the function (i.e. it is a free pointer with which the user may
play). There are two related forms of \code{for\_all\_list}:

\begin{verbatim}
(int) for_all_2list( (OPRS_LIST) list, (OPRS_NODE) ptr, ptr2, (PFI) func )
(int) for_all_3list( (OPRS_LIST) list, (OPRS_NODE) ptr, ptr2, ptr3, (PFI) func )
\end{verbatim}

They provide one or two additional pointers for the user convenience.
In all forms, the pointer is passed to \code{func()} in order, followed
by a \code{(OPRS\_NODE} \var{element}).
\end{typefn}

\begin{typefn}{listPack Function} {OPRS\_NODE} {search\_list} {(OPRS\_LIST \var{list},
(OPRS\_NODE) ptr, (PFI) func)}
\code{search\_list()} is useful for finding an element in a \code{OPRS\_LIST}
which matches some criteria. The user writes a function \code{(int) func(OPRS\_NODE
\var{ptr}, OPRS\_NODE \var{element})} which returns boolean \code{TRUE} when it
matches the criteria on an element in the specified list. List processing
stops after the first
\code{TRUE} returns, and the element on which \code{func()} succeeded is
returned. \code{NULL} is returned if \code{func()} never succeeds.
\end{typefn}

\begin{typefn}{listPack Function} {OPRS\_LIST} {list\_equal} {(OPRS\_LIST \var{list1}, OPRS\_LIST
\var{list2}, PFI \var{func})}
This function compares two \code{OPRS\_LIST}s, and returns a boolean quantity
indicating whether they are equivalent or not.  \code{(PFI \var{func})} is a
function which passes one element from each list, and returns the boolean
quantity describing their equality.  The elements in each \code{OPRS\_LIST} are
passed (in order) to \code{func}, until \code{list\_equal()} determines the
lists are not equal.  If \code{func} is \code{NULL}, \code{list\_equal()} simply
tests that the elements are the same.

\begin{verbatim}
int test_records( rec1, rec2 )  /* are two records equal? */
struct record *rec1, *rec2; /* both are rec pointers */
    {
    /* We define them to be equivalent iff height and weight are the same */
    return (rec1->height == rec2->height) && (rec1->weight == rec2->weight);
    }

...
    /* Print the records in the first list, and if the second list is
   not the same, print the records in that list */
    for_all_list( rec_list1, stdout, print_record );
    if (! (list_equal( rec_list1, rec_list2, test_records )))
    for_all_list( rec_list1, stdout, print_record );
\end{verbatim}
\end{typefn}

\subsection{Changing the Order of the Elements}
\node{Changing the Order of the Elements}

Since one of the fundamental tasks of programming entails sorting,
\i{listPack}
provides primitives to reorder a list of elements to the user
specifications.

\begin{typefn}{listPack Function} {OPRS\_LIST} {reverse\_list} {(OPRS\_LIST \var{list})}
The order of elements in the \code{OPRS\_LIST} is reversed. For example, the tail
becomes the first element, and the head becomes the last one. \code{(OPRS\_LIST
\var{list})} is returned for the convenience of the caller.
\end{typefn}

\begin{typefn}{listPack Function} {OPRS\_LIST} {sort\_list} {(OPRS\_LIST \var{list})}
The elements of the specified \code{list} are sorted in the ascending order of
their tag values. The \var{list} is returned so the user can do:
\begin{verbatim}
new = reverse_list( sort_list( eval_list( copy_list(my_list, flag),
my_func)) );
\end{verbatim}
\end{typefn}

\begin{typefn}{listPack Function} {OPRS\_LIST} {sort\_list\_func} {(OPRS\_LIST \var{list},
PFI \var{func})} This function sorts a \code{OPRS\_LIST} with more complicated
criteria than can easily be specified by an integer tag variable. The user
specified functions,
\code{func( element\_a, element\_b )}, return boolean
\code{TRUE} if they are ordered properly (i.e. element\_a precedes
element\_b) and \code{FALSE} otherwise.
\code{sort\_list\_func()} returns \code{(OPRS\_LIST \var{list})}
for the user convenience.
\emph{Note:} For efficiency reasons, \code{func()} should return
\code{TRUE} if
the elements are equivalent.
\end{typefn}

\begin{typefn}{listPack Function} {OPRS\_LIST} {merge\_sort\_list\_func} {(OPRS\_LIST
\var{list}, PFI \var{func})}
Identical to \code{sort\_list\_func()}, but uses merge sort rather than bubble
sort.
\end{typefn}

\subsection{Marking Current Position in a OPRS\_LIST}
\node{Marking Current Position in a OPRSLIST}

A spare pointer in the \code{OPRS\_LIST} header record can be used
to point at a particular place in a \code{OPRS\_LIST}.  This feature may be used
to walk forwards or backwards along a \code{OPRS\_LIST}, doing incremental
searches or loop iteration.

Note that there is only one such pointer in each \code{OPRS\_LIST}.  Thus the use of
any function which changes that pointer will affect any other software that
uses it.  In particular, it means that the
\code{loop\_through\_list} macro may not be used recursively on the
same \code{list}.  If this is necessary, use the \code{for\_all\_list()} function
or \code{for\_list\_loop()} macro.

\begin{typefn}{listPack Function} {OPRS\_NODE} {get\_list\_next} {(OPRS\_LIST \var{list},
OPRS\_NODE \var{current})}
This function advances the current \code{OPRS\_LIST} element to be the one
immediately following \code{(OPRS\_NODE \var{current})}, and that element is
returned.

If \var{current} is not in \code{(OPRS\_LIST \var{list})}, \code{NULL} is
returned.
If \var{current} is \code{NULL}, the head of the list becomes the
current element, and is returned.

This function is particularly useful for iterating on two lists
simultaneously:
\begin{verbatim}
ptr1 = get_list_next( list1, NULL );    /* get first element of list1 */
ptr2 = get_list_next( list2, NULL );    /* get first element of list2 */

while ( ptr1 != NULL && ptr2 != NULL){    /* loop on each element */

    printf( "Nodes %x and %x\n", ptr1, ptr2 );    /* code to deal with elements */

    ptr1 = get_list_next( list1, ptr1 );    /* now get next from list1 */
    ptr2 = get_list_next( list2, ptr2 );    /* now get next from list2 */
    }
\end{verbatim}
\end{typefn}

\begin{typefn}{listPack Function} {OPRS\_NODE} {get\_list\_prev} {( OPRS\_LIST \var{list},
OPRS\_NODE
\var{current})}
This function retreats the current \code{OPRS\_LIST} element to be the one
immediately preceding \code{(OPRS\_NODE \var{current})}, and that element is
returned.

If \var{current} is not in \code{(OPRS\_LIST \var{list})}, \code{NULL} is returned.
If \var{current} is \code{NULL}, the tail of the list becomes the current
element, and is returned.
\end{typefn}

\begin{typefn}{listPack Function} {OPRS\_NODE} {get\_list\_next\_func} {( OPRS\_LIST
\var{list},
OPRS\_NODE \var{current}, OPRS\_NODE \var{ptr}, PFI \var{func})} This function advances the
current \code{OPRS\_LIST} element to be the first one in
\code{(OPRS\_LIST \var{list})} after \code{(OPRS\_NODE \var{current})} upon which the
function
\var{func} returns boolean \code{TRUE}. That element is returned.

If \var{current} is not in \code{(OPRS\_LIST \var{list})}, \code{NULL} is returned.
If \var{current} is \code{NULL}, processing starts with the first element.

\code{(OPRS\_NODE \var{ptr})} is passed to the function, \var{func} as a free
parameter. Hence the signature for \var{func} should look like:
\begin{verbatim}
int my_function( parameter, list_element )
OPRS_NODE parameter, list_element;    /* or any other pointer type */
    {
    }
\end{verbatim}

This function is useful for processing selected elements in a \code{OPRS\_LIST}:
\begin{verbatim}
ptr = NULL;
while ((ptr = get_list_next_func( list, ptr, NULL, my_function )) != NULL)
{

    ...    /* process each element that my_function found */

    }
\end{verbatim}
\end{typefn}

\begin{typefn}{listPack Macro} {void} {loop\_through\_list} {(OPRS\_LIST \var{list},
type
\var{pointer}, type)}
This macro is provided by \i{listPack} to serve as a
\code{for} loop in normal programming. A \i{C}-language \code{for}
statement is
generated and iterates \code{pointer} through the elements in
\code{(OPRS\_LIST \var{list})}.

Note that only one \code{loop\_through\_list()} macro can be active on a
given list at any time. If recursion is desired, use the
\code{for\_list\_loop()} macro.

The last parameter is a type specification, and is used by the macro to
generate the proper recasting instructions to make both the \i{C}-compiler
and \i{lint(1)} happy.
\begin{verbatim}
struct foo {
    int x, y;
} * temp;        /*temp is a pointer */
OPRS_LIST my_list;

...

        /* for each pointer in the list */
loop_through_list( my_list, temp, struct foo *){
            /* process the element */
    printf( "%d, %d\n", temp->x, temp->y );
    }
\end{verbatim}
\end{typefn}

\begin{typefn}{listPack Macro} {void} {for\_list\_loop} {(OPRS\_LIST \var{list}, OPRS\_LIST
\var{temp}, type \var{pointer}, type)}
This macro is provided by \i{listPack} to serve as a
\code{for} loop in normal programming. A \i{C}-language \code{for}
statement is generated and iterates \code{pointer} through the elements in
\code{(OPRS\_LIST \var{list})}.

The last parameter is a type specification, and is used by the macro to
generate the proper recasting instructions to make both the \i{C}-compiler
and \i{lint(1)} happy.

It differs from \code{loop\_through\_list()} in that it allows recursive
access to the list, and thus requires a local variable.  The above example
would read.
\begin{verbatim}
OPRS_LIST loop_temp;
        /* for each pointer in the list */
for_list_loop( my_list, loop_temp, temp, struct foo *){
            /* process the element */
    printf( "%d, %d\n", temp->x, temp->y );
    }
\end{verbatim}
\end{typefn}

\chapter{Lisp and Lisp-like Functions}
\node{Lisp and Lisp-like Functions}
\cpindex{Lisp and Lisp-like Functions}
\cpindex{Lisp Functions}
\cpindex{Lisp'isms}

There is no a priori reason for the \COPRSDE{} to provide Lisp like functions,
or a Lisp environment. However, for upward compatibility reasons, and to allow
the user familiar with Lisp lists manipulation functions to use lisp functions
in the \CPK{}, we added various functions and data structures in the \COPRSDE{}.

In this chapter, we describe the different functions and the mechanisms which
are provided to use lispisms in the \COPRSDE{}.



\section{LISP\_LIST}
\node{LISPLIST}
\cpindex{LISP\_LIST}

% ****** un petit dessin vaux mieux que tout...

First of all, it is important to note that the overall syntax used in the
\COPRSDE{} is ``Lisp like'', and looks like Lisp. In other words, the
various expressions are currently given in a syntax which looks a lot like the
Lisp one (parenthesis are certainly here to remind you of this). But it does
not mean there is a lisp interpreter behind the reader. In fact, whatever is
read from the keyboard, or from a file is fed in a parser (or a command
interpreter) and transformed in internal structures (the details of which are
of no interest to the reader). Most of these structures are accessed by the
user or the kernel in a transparent way. For example, when you conclude the
fact \code{(foo (bar 3 4))} in the database, you cannot do a \code{car} on the
inner expression \code{(bar 3 4)} to conclude let say \code{(foo bar)}. The
term \code{(bar 3 4)} is not a Lisp list, it is a composed term.

However, one can define Lisp like structures. They are subject to a special
reader syntax. In Lisp, when the reader encounters a left parenthesis \samp{(},
he builds a list until he finds the matching closing parenthesis \samp{)}. In
\COPRS{}, the left and right parentheses do not create any Lisp list object in
any situation. To create a \code{LISP\_LIST} (note the Term type) (on which you
will be able to apply Lisp functions such as \code{car}, \code{cdr},
\code{cons}), you need to use \samp{(.} and a matching \samp{.)}. As a
mnemonic, you can remember that the dot \samp{.}  is the cons operator in
Lisp. The \COPRS{} writer uses the same syntax to write \code{LISP\_LIST}. So, if
you conclude \code{(foo (. bar 3 4 .))} in the database, you have a
\code{(. bar 3 4 .)} term which is a \code{LISP\_LIST}, not a composed term, and
you can apply to it any function requiring a \code{LISP\_LIST} as an operand.

One interesting thing in Lisp is the lack of type. The various elements of a
list are typed internally by the Lisp kernel, but you can cons anything with
anything, and do an eval on it if you wish to. In \COPRS{}, it is the same, most
objects are typed when read and the syntax defines exactly what can go where.
For example, the expression \code{(foo (1 2 3 4))} is not legal because
\code{1} is recognized as an integer and not as a symbol. However, one may want
to manipulate integer lists, or lists of objects unknown to the reader a priori
(probably internal structures such as OP-instances, intentions, goals, etc). In
this case, one uses \code{LISP\_LIST}s: because of their lack of type
requirement, they can basically handle whatever terms.  When you build a list
like in the expression: \code{(foo (. 1 2 3 4 .))} you actually build a
\code{LISP\_LIST} of terms which are all integers. In fact, this is how you can
build \code{LISP\_LIST} of \code{LISP\_LIST}s... The expression \code{(foo
(. (. 1 2 3 4 .) 2 3 4 .) )} is a predicate with one term, a \code{LISP\_LIST}
which contains terms, the first of which is a \code{LISP\_LIST} itself. Even if
it is not obvious at a first glance, a list of terms is the only list you can
build from the reader.  All other lists and the objects they contain are built
by internal operations.  For example, the \code{soak} meta fact contains a
\code{LISP\_LIST} of OP-instances.

Keep in mind that the elements of a \code{LISP\_LIST} are \code{Terms}.

The empty LISP OPRS\_LIST is defined as follow in the file \file{oprs-type-pub.h}.
\begin{verbatim}
extern const L_List l_nil;  /* The lisp nil constant. */
\end{verbatim}

\section{Standard Lisp Functions}
\node{Standard Lisp Functions}
\cpindex{Standard Lisp Functions}

Standard lisp functions are predefined evaluable functions
(\pxref{Predefined Evaluable Functions}) used to manipulate \code{LISP\_LIST}.

\begin{typeefa}{Evaluable Function}{INTEGER}{length}{(LISP\_LIST)}
\xref{Predefined Evaluable Functions}.
\end{typeefa}

\begin{typeefa}{Evaluable Function}{Any Term}{select-randomly}{(LISP\_LIST)}
\xref{Predefined Evaluable Functions}.
\end{typeefa}

\begin{typeefa}{Evaluable Function}{LISP\_LIST}{cons}{(Any Term, LISP\_LIST)}
\xref{Predefined Evaluable Functions}.
\end{typeefa}

\begin{typeefa}{Evaluable Function}{Any Term}{first}{(LISP\_LIST)}
\xref{Predefined Evaluable Functions}.
\end{typeefa}

\begin{typeefa}{Evaluable Function}{Any Term}{car}{(LISP\_LIST)}
\xref{Predefined Evaluable Functions}.
\end{typeefa}

\begin{typeefa}{Evaluable Function}{LISP\_LIST}{cdr}{(LISP\_LIST)}
\xref{Predefined Evaluable Functions}.
\end{typeefa}

\begin{typeefa}{Evaluable Function}{Any Term}{caar}{(LISP\_LIST)}
\xref{Predefined Evaluable Functions}.
\end{typeefa}

\begin{typeefa}{Evaluable Function}{Any Term}{cadr}{(LISP\_LIST)}
\xref{Predefined Evaluable Functions}.
\end{typeefa}

\begin{typeefa}{Evaluable Function}{LISP\_LIST}{cdar}{(LISP\_LIST)}
\xref{Predefined Evaluable Functions}.
\end{typeefa}

\begin{typeefa}{Evaluable Function}{LISP\_LIST}{cddr}{(LISP\_LIST)}
\xref{Predefined Evaluable Functions}.
\end{typeefa}

\begin{typeefa}{Evaluable Function}{Any Term}{caaar}{(LISP\_LIST)}
\xref{Predefined Evaluable Functions}.
\end{typeefa}

\begin{typeefa}{Evaluable Function}{Any Term}{cadar}{(LISP\_LIST)}
\xref{Predefined Evaluable Functions}.
\end{typeefa}

\begin{typeefa}{Evaluable Function}{LISP\_LIST}{cdaar}{(LISP\_LIST)}
\xref{Predefined Evaluable Functions}.
\end{typeefa}

\begin{typeefa}{Evaluable Function}{LISP\_LIST}{cddar}{(LISP\_LIST)}
\xref{Predefined Evaluable Functions}.
\end{typeefa}

\begin{typeefa}{Evaluable Function}{Any Term}{caadr}{(LISP\_LIST)}
\xref{Predefined Evaluable Functions}.
\end{typeefa}

\begin{typeefa}{Evaluable Function}{Any Term}{caddr}{(LISP\_LIST)}
\xref{Predefined Evaluable Functions}.
\end{typeefa}

\begin{typeefa}{Evaluable Function}{LISP\_LIST}{cdadr}{(LISP\_LIST)}
\xref{Predefined Evaluable Functions}.
\end{typeefa}

\begin{typeefa}{Evaluable Function}{LISP\_LIST}{cdddr}{(LISP\_LIST)}
\xref{Predefined Evaluable Functions}.
\end{typeefa}

\begin{typeefa}{Evaluable Function}{Any Term}{second}{(LISP\_LIST)}
is the second function. It is defined for 1 term \code{LISP\_LIST}. It returns
the second (or cadr) \code{Any Term} of the \code{LISP\_LIST}.
\end{typeefa}

\begin{typeefa}{Evaluable Function}{LISP\_LIST}{reverse}{(LISP\_LIST)}
is the reverse function. It is defined for 1 term \code{LISP\_LIST}. It returns the
reverse of the \code{LISP\_LIST}.
\end{typeefa}

\begin{typeefa}{Evaluable Function}{Any Term}{second}{(LISP\_LIST)}
\xref{Predefined Evaluable Functions}.
\end{typeefa}


\chapter{Examples}
\node{Examples}
\cpindex{Examples}

Several examples are included in the \COPRS{} distribution.  We shall describe
them in this chapter in an increasing complexity order. All examples are
available in the \file{data} sub directory of the \COPRSDE{} distribution. Feel
free to look at these examples and to play with the various files contained in
this directory (make a copy if you need to modify them).



\section{Message Example}
\node{Message Example}
\cpindex{Message Example}

This example presents one important feature of \COPRS: the message
passing mechanism. To run it, you create three \COPRS{}, or \XPK{}s named
\code{foo}, \code{bar} and \code{boo}, in which you load
\file{data/foo.inc}, \file{data/bar.inc} and \file{data/boo.inc}
respectively.



\subsection{Message Example OPs}
\node{Message Example OPs}
\cpindex{Message Example OPs}

Here are the procedures from one of the OP files provided for this demo.  It
can be found in the file \file{bar.opf}.

\input{bar}

The other OP files are similar.

\section{Test Examples}
\node{Test Examples}
\cpindex{Test Examples}

Other OPs files are provided in the standard distribution as examples. Most of
these examples are located in the \file{data/test} directory.



\subsection{Wait OPs}
\node{Wait OPs}
\cpindex{Wait OPs}

These OPS can be found in the file \file{test/wait.opf}.

\input{wait}

\subsection{LISP\_LIST manipulation OPs}
\node{LISPLIST manipulation OPs}
\cpindex{LISP\_LIST manipulation OPs}

These OPS can be found in the file \file{example-cons.opf}.

\input{example-cons}

\subsection{Fibonacci OPs}
\node{Fibonacci OPs}
\cpindex{Fibonacci OPs}

These OPS can be found in the file \file{fib.opf}.

\input{fib}

\subsection{Parallel Fibonacci OPs}
\node{Parallel Fibonacci OPs}
\cpindex{Parallel Fibonacci OPs}

These OPS can be found in the file \file{fib-par.opf}.

\input{fib-par}

\chapter{How to Install the \COPRSDE{}}
\node{How to Install the OPRS Development Environment}
\cpindex{How to Install the \COPRSDE{}}

\textbf{This appendix is completely out of date and need to be updated since I moved
the code in open source.}

The easiest way to install \COPRSDE{} is to either to use git:

\noindent
\code{git clone http://softs.laas.fr/git/robots/openprs.git\\
cd openprs\\
./configure\\
make}

or to use robotpkg \url{http://homepages.laas.fr/mallet/robotpkg}. If you
use robotpkg and have it installed already, just do:

\noindent\code{cd architecture/openprs}\\
\code{make update}

\vspace{5mm}

The rest of this appendix is left for ``information'' it is mostly out of
date and useless.

The installation procedure is different (although very similar), according to
the license you have. One general comment is that most Makefile are written for
GNU Make \cite{GNU-make-manual}, and may not work with other \code{make}
program. Similarly, a number of tools (such as Perl \cite{perl-manual}) may be
required to build the system (this is particularly true for the source
distribution).



\section{Unpacking the Distribution}
\node{Unpacking the Distribution}
\cpindex{Unpacking the Distribution}

\begin{enumerate}

\item Choose a directory where you want to install the distribution. \*
\code{\# cd /usr/local/oprs (or your preferred directory)}

\item Unpack the distribution with: \*
\code{\# tar xvf <tape-dev>} \*
or, if you got the distribution as a compressed tar file: \*
\code{\# gnutar zxvf <dist-file> }

\end{enumerate}

\section{Description of the Distribution}
\node{Description of the Distribution}
\cpindex{Description of the Distribution}

Assuming \COPRSDE{} is installed in the directory \file{/usr/local/oprs/} you
will find in it the directories: \file{bin}, \file{lib}, \file{include},
\file{pub\_src}, \file{data}, \file{doc}, \file{util}, \file{app-defaults},
\file{site.make}, \file{contrib} and \file{demo} (\file{src} if you have a
source license).  If your distribution contains more than one architecture
distribution, the \file{bin} and \file{lib} directory as well as the
\file{site.make} file (which contain machine dependant stuff) will be located
in a \code{\$TARGET} directory (as defined by tcsh, i.e sun4 for SunOS 4.1.3,
sparc for Solaris 2.x, VxWorks, etc).

According to your license, some of these directories may be present or not (if
you do not have a source license, the \file{src} directory is not present).

Here is an explanation of what you will find in each of these directories
(after installation if you have a source license):

\begin{description}

\item [\file{bin}] contains all the binaries. If your license is valid for more
than one CPU type, the various binaries (in \file{bin} and \file{lib} should be
in their \code{\$TARGET} directory. 

If you plan to use \COPRSDE{}, your \code{PATH} environment variable should
contain the appropriate bin directory.  Consequently, you should add the
following code to your \file{.login} or \file{.cshrc} file.

\begin{verbatim}
setenv OPRSDIR /usr/local/oprs
setenv PATH "${PATH}:${OPRSDIR}/${TARGET}/bin"
\end{verbatim}

Here is a list of the files contained in this directory:

\begin{description}

\item [\file{oprs}] is the \CPK{} program.

\item [\file{oprs-server}] is the \OPRSS{} program.

\item [\file{xoprs}] is the \XPK{} program.

\item [\file{mp-oprs}] is the \MPA{} program.

\item [\file{oprs-cat}] is a program which is used by the \XPK{}.

\item [\file{op-editor}] is the \OPE{} program.

\end{description}

\item [\file{lib}] contains a number of libraries and relocatable files which
you may need to build your application and to allow it to connect to the \MPA{}.

If your license is valid for more than one CPU type, the various
binaries/libraries should be in the \code{\$TARGET} directory.

The relocatable are given in two format, one which can be linked to some C++
functions, in which case the \code{main} is not defined to allow a C++
\code{main} to be used (the C++ main performs some initialization required by
C++ functions).

\begin{description}

\item [\file{libmp.a}] is a library you can use if you want your own program to
connect to the \MPA{}.

\item [\file{libopaque.a}] is a library which can be necessary if the following
symbols are missing: \*
\code{user\_time, user\_time\_note, user\_call\_from\_parser}.

\item [\file{oprs-relocatable}] is a relocatable \CPK{},  you can use
to build your own version of the \CPK{} (i.e. containing your own evaluable
functions and predicates).

\item [\file{xoprs-relocatable}] is a relocatable \XPK{},  you can use
to build your own version of the \XPK{} (i.e. containing your own evaluable
functions and predicates).

\item [\file{c++-oprs-relocatable}] is a relocatable \CPK{},  you can use
to build your own version of the \CPK{} (i.e. containing your own evaluable
functions and predicates) with C++ code. In this relocatable the oprs
\code{main} is called \code{oprs\_main} and takes the same argument as the real
\code{main}: \code{argc}, \code{argv} and \code{envp}. \code{oprs\_main} does not
return.

\item [\file{ c++-xoprs-relocatable}] is a relocatable \XPK{}, you can use
to build your own version of the \XPK{} (i.e. containing your own evaluable
functions and predicates) with C++ code. In this relocatable the oprs
\code{main} is called \code{oprs\_main} and takes the same argument as the real
\code{main}: \code{argc}, \code{argv} and \code{envp}. \code{oprs\_main} does not
return. 

\end{description}

\item [\file{include}] contains include files needed to build your own
version of the \XPK{} and the \CPK{}.

\item [\file{pub\_src}] contains some source example files needed to build your own
version of the \XPK{} and the \CPK{}.

\begin{description}

\item [\file{user-action.c}] contains examples of how to defined your own
actions and an example of a call to \code{declare\_user\_action},

\item [\file{user-action.h}] contains definitions needed for \file{user-actions.c}.

\item [\file{user-ev-function.c}] contains examples of how to defined your
evaluable functions and example of a call to \code{declare\_user\_eval\_funct}.

\item [\file{user-ev-function.h}] contains definitions needed for \file{user-ev-function.c}.

\item [\file{user-ev-predicate.c}] contains examples of how to defined your own
evaluable  and an example of a call to \code{declare\_user\_eval\_pred}.

\item [\file{user-ev-predicate.h}] contains definitions needed for \file{user-ev-predicate.c}.

\item [\file{user-external.c}] contains examples of how to call
\code{start\_kernel\_user\_hook} and \code{end\_kernel\_user\_hook}.

\item [\file{user-external.h}] contains definitions needed for \file{user-external.c}.

\item [\file{user-external\_f.h}] contains function definitions needed for
\file{user-external.c}.

\end{description}

\item [\file{doc}] contains the documentation of the \COPRSDE{}. Here is a
list of the files contained in this directory:

\begin{description}

\item [\file{dir}] Info top level dir for the documentation.

\item [\file{oprs}] is the oprs master info file.

\item [\file{oprs-[0-9][0-9]}] are the oprs info files.

\item [\file{oprs.dvi}] is the oprs.dvi file (dvi format).

\item [\file{oprs.ps}] is the postscript version of the documentation.

\item [\file{fig}] is a directory containing the postscript figures of the
manual.

\end{description}

\item [\file{contrib}] contains some code (source) contributed by other people.
For now, one can find a library which ease the extraction of Terms parameters
from TermLists (as passed to evaluable functions, predicates and actions), but
ans the building of Terms to return value from these functions.

\item [\file{site.make}] contains a the Makefile variable used to produce the
\COPRSDE{} you are using. This can be usefull if you have to produce your own
\PK{}s.

\item [\file{util}] contains a number of utilities, shells, perl scripts and C
programs to build \COPRS{}.

\begin{description}

\item [\file{Makefile}] a makefile...

\item [\file{ad2c}] is a shell script to transform a \file{.ad} resource file
in a string suitable for inclusion in a \file{.c} file (to be defined as a
fall back resources string).

\item [\file{update-inc-file}] is a shell script which upgrade your \file{.inc}
file from the pre 1.1.1 version format to the newest format.

\item [\file{mkdep}] is a perl script which can be used to create dependencies
files.

\item [\file{vw-script}] is an exemple of a VxWorks script to load and run a
\CPK{}, a \MPA{} or a \MPK{}.

\item [\file{lex-includer.l}] is a lex program which is used during the \COPRS{}
compilation to build the various lex and yacc grammar files. This file is only
present in the source distribution.

\end{description}

\item [\file{data}] contains a number of data files examples (\file{.opf}
files, \file{.inc} files, \file{.db} files, \file{.sym} files, etc)

\item [\file{data/test}] contains a number of test files (\file{.opf}
files, \file{.inc} files, \file{.db} files, \file{.sym} files, etc)

\item [\file{app-defaults}] contains some application default files for the
\XPK{} application (\file{XOprs}) and the \OPE{} application
(\file{Op-editor}). Both applications are Xt based, and can be customized
using the resources file \cite{Xt-manual,Xlib-manual,Motif-manual}. \xref{OP Editor Motif Widgets Hierarchy and Resources} and \nxref{X-OPRS Motif Widgets Hierarchy and Resources} for a list of the widget and resources available.
In any case, you should either put this directory in your
\code{XFILESEARCHPATH} environment variable as in:

\begin{verbatim}
setenv OPRSDIR /usr/local/oprs
if ( $?XFILESEARCHPATH ) then
     setenv XFILESEARCHPATH "${XFILESEARCHPATH}:${OPRSDIR}/%T/%N%C"
else
     setenv XFILESEARCHPATH "${OPRSDIR}/%T/%N%C"
endif
\end{verbatim}

The \code{\%C} suffix is very important if you want to take advantage of the
multi language support of \COPRS{} (\pxref{Xt/Motif Widgets Hierarchy and Resources}).

or copy these two files in your own app default directory pointed by the
environment variable \code{XAPPLRESDIR} (in this case, remove the \file{.ad}
suffix).

\begin{verbatim}
setenv  XAPPLRESDIR ${HOME}/X/app-defaults/
\end{verbatim}

\item [\file{demo}] contains some demonstration applications. In
particular it contains the truck-demo.

\item [\file{src}] contains all the source of the \COPRSDE{}. This directory is
only present when you have a source license.

\end{description}

\section{Installation for Demonstration License}
\node{Installation for Demonstration License}
\cpindex{Installation for Demonstration License}

The demonstration license is basically provided to ``play'' with the
\COPRSDE{}. Therefore, some critical features are disabled to
prevent the user from making a real use of the kernel. For example, the garbage
collector is modified in such a way that the kernel will grow indefinitely
after a period of time, so do not be surprised if you run out of swap space or
if your application is becoming slower and slower, this is due to the swapping
activity.

Moreover, you may not get the relocatable files... As a result, you will
not be able to include your evaluable functions or predicates in the
kernels.

\section{Installation for Binary License}
\node{Installation for Binary License}
\cpindex{Installation for Binary License}

Binary license still comes with some sources. Some include files are
provided, and some examples of C code are also given to help the user
define his own evaluable predicates or functions. In fact, when you get a
binary license, you get relocatable executables which must be linked to
other object files to be really executable.

\section{Installation for Source License}
\node{Installation for Source License}
\cpindex{Installation for Source License}

If you have a source license, you have the source of all the \COPRSDE{}.
The distribution contains the \file{src} directory.
You may need to compile and install the \COPRSDE{}.

For this go in the oprs directory and create a \file{build} directory at the
same level than the  \file{src} directory. Cd in the \file{build} directory and
then call the \file{util/mkoprsbintree} utility. Then check the site.make file
to customize it (check the installation directory for example) and then do a:

\begin{verbatim}
make depend
make
make install
\end{verbatim}

The \code{make depend} command will create the dependencies between the
sources files... This step is really require if and only if you plan to
modify part of the sources, and you want then to recompile only the
appropriate files. Nevertheless, it is recommended to do it.

The \code{make} command will build the executables, libraries, etc.

The \code{make install} command will install the various programs,
libraries, defaults files, data, include in the appropriate directories.

Note that you may have some problem while generating the lex grammar C
files. For example, on some Sun OS version (4.1.3 for example), or under Ultrix
(4.2a), the lex program fails to handle such a large lex grammar. If this is
the case, you should either request a revised version of lex from th OS vendor
or generate the C files on another machine. For this, you may defined something
like:

\begin{verbatim}
LEX = rsh <other-host> lex
\end{verbatim}

where <other-host> is a machine which is able to parse the \COPRS{} lex grammar.
Do not use flex as a lex replacement. This will not work (unless you hack a
fair amount of the grammar and the macros).

\chapter{Grammar Used in the \COPRSDE{}}
\node{Grammar Used in the OPRS Development Environment}
\cpindex{Grammar Used in the \COPRSDE{}}



\section{Syntaxic Grammar Used in the \COPRSDE{}}
\node{Syntaxic Grammar Used in the OPRS Development Environment}
\cpindex{Syntaxic Grammar Used in the \COPRSDE{}}

Here is the yacc-style grammar used in the different parsers of the
\COPRSDE{}.

\smallverbatimfile{oprs-yacc.tex}

\section{Lexical Grammar Used in the \COPRSDE{}}
\node{Lexical Grammar Used in the OPRS Development Environment}
\cpindex{Lexical Grammar Used in the \COPRSDE{}}

Here is the lex-style token grammar used in the different parsers of the
\COPRSDE{}.

\smallverbatimfile{oprs-lex.tex}

\chapter{Xt/Motif Widgets Hierarchy and Resources}
\node{Xt/Motif Widgets Hierarchy and Resources}
\cpindex{Xt/Motif Widgets Hierarchy and Resources}

Motif and Xt allow the user to set most widget resource values using the
resource file (see the X11 \cite{Xlib-manual} and Xt \cite{Xt-manual}
documentation for more on this topic).  \XOPRS{} and the \OPE{} takes advantage
of this mechanism as much as possible. In fact, as few as possible resource
values are ``hard wired'' in the code, and most of them are set in the
\file{XOprs} and \file{Op-editor} resources files. To take full advantage
of this mechanism, the user needs to know the widget hierarchy of the
application. The versions of the \XOPRS{} and the \OPE{} which are based on
X11R5/Xt/Motif 1.2 can even take advantage of the new resource protocol which
allows the user to interactively set resource values. The user can thus make a
lot of customization using resource setting.

Since version 1.3.1, \COPRS{} offers support for different language (French and
English). Thus the proper selection of the application default file is very
important. A priori, if no application file is found, the default language
resources value are stored in the kernel. 

For each X application (\XOPRS{} and the \OPE{}) , the \file{app-defaults}
directory contains 3 files.  For example, for \XOPRS{}, it contains
\file{XOprs-fr}, \file{XOprs-en} and \file{XOprs} (which is a link on one of the
two first file). A priori, you need to tell X where are the application
defaults file are located. This can be done using the variable
\code{XFILESEARCHPATH}.

\begin{verbatim}
setenv OPRSDIR /usr/local/oprs
if ( $?XFILESEARCHPATH ) then
     setenv XFILESEARCHPATH "${XFILESEARCHPATH}:${OPRSDIR}/%T/%N%C"
else
     setenv XFILESEARCHPATH "${OPRSDIR}/%T/%N%C"
endif
\end{verbatim}

The \%C (for customization) is supported in X11 since release 5. Basically, if
you have, for an application \code{app\_name} with \code{app\_name\_class} as
application class, a \code{app\_name\_class.customization: res-value} resource
defined in your \file{.Xresources} or \file{.Xdefault} files (i.e. already
defined in the resource manager of your X server), then when you launch
application \code{app-name}, it will look for an application default file with
the name \%N\%C, i.e. \code{app\_name\_class} concatenated with \code{res-value}.

So if you define \code{XOprs.customization: -en}, it will load \file{XOprs-en},
while if you define \code{XOprs.customization: -fr}, it will load
\file{XOprs-fr}. If none are defined, it will load \file{XOprs}. This mechanism
is the same for the \code{Op-editor}. This mechanism will allow you to
automatically select the language of the interface.

Note that each application file contains a \code{*.language} resource which is
used by the application to then select the proper string when these are not
defined a X resources.



\section{Xt Command Line Arguments}
\node{Xt Command Line Arguments}
\cpindex{Xt Command Line Arguments}

\begin{table}
\begin{center}
\begin{tabular}{||l|l|l|l||}
\hline
Option String & Resource Name & Argument Kind & Resource Value \\
\hline
-background & *background   & SepArg    & next argument \\
-bd & *borderColor  & SepArg    & next argument \\
-bg & *background   & SepArg    & next argument \\
-borderwidth    & .borderWidth  & SepArg    & next argument \\
-bordercolor    & *borderColor  & SepArg    & next argument \\
-bw & .borderWidth  & SepArg    & next argument \\
-display    & .display  & SepArg    & next argument \\
-fg & *foreground   & SepArg    & next argument \\
-fn & *font     & SepArg    & next argument \\
-font   & *font     & SepArg    & next argument \\
-foreground & *foreground   & SepArg    & next argument \\
-geometry   & .geometry & SepArg    & next argument \\
-iconic & .iconic   & NoArg     & ``true'' \\
-name   & .name     & SepArg    & next argument \\
-reverse    & .reverseVideo     & NoArg     & ``on'' \\
-rv & .reverseVideo     & NoArg     & ``on'' \\
+rv & .reverseVideo     & NoArg     & ``off'' \\
-selectionTimeout   & .selectionTimeout & SepArg    & next argument \\
-synchronous    & .synchronous  & NoArg     & ``on'' \\
+synchronous    & .synchronous  & NoArg     & ``off'' \\
-title  & .title    & SepArg    & next argument \\
-xnllanguage    & .xnlLanguage  & SepArg    & next argument \\
-xrm    & next argument  & ResArg   &next argument \\
\hline
\end{tabular}
\end{center}
\caption{Xt Application Default Line Arguments and Resources}
\label{xt-args-table}
\end{table}

There are a number of arguments and resources which are defined by default for
any Xt based application. These are indeed available for the \XOPRS{}
application and the \OPE{} application. Table \ref{xt-args-table} presents
these arguments. See \cite{Xt-manual} for more information on this subject.

Example:
\begin{verbatim}
op-editor -iconic
xoprs -bg blue
\end{verbatim}

\section{\XOPRS{} Motif Widgets Hierarchy and Resources}
\node{X-OPRS Motif Widgets Hierarchy and Resources}
\cpindex{X-OPRS Motif Widgets Hierarchy and Resources}
\cpindex{Xt Resources}
\cpindex{Resources}



\subsection{How to Connect your Own Widget in \XOPRS{}}
\node{How to Connect your Own Widget in X-OPRS}
\cpindex{How to Connect your Own Widget in X-OPRS}

You can if you want connect you own widget or widget tree to the \XOPRS{} one.
They will be treated by the \XPK{} as any other widget. You can then manage
them or unmanage them from your OPs. However, you should not make active wait
(by starting recursively a Xt Main Loop...) or you will most likely block the
\XPK{}.

\begin{typevr}{Kernel Variable}{extern Widget}{x\_oprs\_top\_level\_widget}
\pxref{Important Variables}
\end{typevr}

\begin{verbatim}
void start_kernel_user_hook()
{
 intention_scheduler = &intention_scheduler_time_sharing;
 create_my_widget_tree(x_oprs_top_level_widget);
}
\end{verbatim}

\subsection{\XOPRS{} Resources}
\node{X-OPRS Resources}
\cpindex{X-OPRS Resources}

Specific \XOPRS{} resources can be set by the user:

\begin{verbatim}
XOprs.fontList: variable=variable_cs,\
fixed=default_cs,\
variable=text_title_cs,\
-adobe-helvetica-bold-r-normal-*-24-*-*-*-*-*-*-*=title_cs,\
fixed=text_cs,\
-adobe-helvetica-bold-r-normal-*-*-100-*-*-*-*-*-*=node_cs,\
6x12=edge_cs
XOprs.defaultFontList: variable=variable_cs,\
fixed=fixed_cs
XOprs.ipX:               10
XOprs.ipY:               50
XOprs.ipWidth:           80
XOprs.ctxtX:             10
XOprs.ctxtY:             150
XOprs.ctxtWidth:         60
XOprs.setX:              10
XOprs.setY:              250
XOprs.setWidth:          60
XOprs.effX:              10
XOprs.effY:              350
XOprs.effWidth:          60
XOprs.propX:             10
XOprs.propY:             400
XOprs.propWidth:         60
XOprs.docX:              10
XOprs.docY:              450
XOprs.docWidth:          60
XOprs.actX:              310
XOprs.actY:              50
XOprs.actWidth:          60
XOprs.bdX:               310
XOprs.bdY:               50
XOprs.bdWidth:           120
XOprs.edgeWidth:         40
\end{verbatim}

The \code{XOprs.fontList} resource defines the fonts which are used
in the following character set:

\begin{description}

\item[title\_cs] is the character set used to name the OP which appear
in the top left corner.

\item[text\_title\_cs] is the character set used to name text
fields like \code{INVOCATION, CONTEXT, etc}.

\item[text\_cs] is the character set used for the text contained in the
text field.

\item[node\_cs] is the character set used for the name of the node.

\item[edge\_cs] is the character set used for the edge text.

\end{description}

\subsection{\XOPRS{} Motif Widgets Hierarchy}
\node{X-OPRS Motif Widgets Hierarchy}
\cpindex{X-OPRS Motif Widgets Hierarchy}

The widget hierarchy for the \XOPRS{} program can now be obtained by issuing
the command:

\begin{verbatim}
xoprs -pwt
\end{verbatim}

Each widget is given with its path from the root and with its class name in
parenthesis.

\section{\OPE{} Motif Widgets Hierarchy and Resources}
\node{OP Editor Motif Widgets Hierarchy and Resources}
\cpindex{OP Editor Motif Widgets Hierarchy and Resources}
\cpindex{Xt Resources}
\cpindex{Resources}



\subsection{\OPE{} Resources}
\node{OP Editor Resources}
\cpindex{OP Editor Resources}

The \OPE{} has a number of resource which can be set by the user:

\begin{verbatim}
Op-editor.fontList: variable=variable_cs,\
fixed=default_cs,\
-adobe-helvetica-bold-r-normal-*-24-*-*-*-*-*-*-*=title_cs,\
variable=text_title_cs,\
fixed=text_cs,\
-adobe-helvetica-bold-r-normal-*-*-100-*-*-*-*-*-*=node_cs,\
6x12=edge_cs
Op-editor.defaultFontList: variable=variable_cs,\
fixed=fixed_cs
!
! Default value for the size of the drawing area 
!
Op-editor.workWidth:    2000
Op-editor.workHeight:   1500
!
! Default value for the x, y and with
! of the various OP fields. These values
! are the one used for the Text OP.
!
Op-editor.ipX:          10
Op-editor.ipY:          50
Op-editor.ipWidth:      80
Op-editor.ctxtX:        10
Op-editor.ctxtY:        150
Op-editor.ctxtWidth:    60
Op-editor.setX:         10
Op-editor.setY:         250
Op-editor.setWidth:     60
Op-editor.effX:         10
Op-editor.effY:         350
Op-editor.effWidth:     60
Op-editor.propX:        10
Op-editor.propY:        400
Op-editor.propWidth:    60
Op-editor.docX:         10
Op-editor.docY:         450
Op-editor.docWidth:     60
Op-editor.actX:         310
Op-editor.actY:         50
Op-editor.actWidth:     60
Op-editor.bdX:          310
Op-editor.bdY:          50
Op-editor.bdWidth:      120
Op-editor.edgeWidth:    40
!
! Default print command
!
! This one is for A4 if you have the most recent pbm pacopge.
Op-editor.printCommand: xwdtopnm < \%s |  pnmtops -r -w 8 -h 11.25 | lpr
! For us letter
!Op-editor.printCommand:        xwdtopnm < %s |  pnmtops -r | lpr
\end{verbatim}

The \code{Op-editor.fontList} resource defines the fonts which are used
in the following character set:

\begin{description}

\item[title\_cs] is the character set used for the name of the OP which
appear
in the top left corner.

\item[text\_title\_cs] is the character set used for the name of text
fields like \code{INVOCATION, CONTEXT, etc}.

\item[text\_cs] is the character set used for the text contained in the
text field.

\item[node\_cs] is the character set used for the name of the node.

\item[edge\_cs] is the character set used for the edge text.

\end{description}

The \code{Op-editor.printCommand} resource defines the shell command used
by the various print commands of the \OPE{}.

The \code{Op-editor.workWidth} and \code{Op-editor.workHeight} resource
defines the width and the height of the drawing canvas. These values can
be changed with the ``Change Drawing Size'' command, \xref{Change Drawing Size}.

\subsection{\OPE{} Motif Widgets Hierarchy}
\node{OP Editor Motif Widgets Hierarchy}
\cpindex{OP Editor Motif Widgets Hierarchy}


The widget hierarchy for the \OPE{} program can now be obtained by issuing
the command:

\begin{verbatim}
op-editor -pwt
\end{verbatim}

Each widget is given with its path from the root and with its class name in
parenthesis.

\chapter{Known Problems and Things to Avoid}
\node{Known Problems and Things to Avoid}
\cpindex{Known Problems and Things to Avoid}

Although our goal is to constantly improve the \COPRSDE, few problems
have been reported by users or found by our development team and, remain
unsolved or unresolved. Most of them originate in mechanisms, tools or
systems which are not directly under our control (such as X11, Motif or
Unix).

In any case, bugs and problems should be reported to the following electronic
mail address:\*
\code{oprs-bug@ingenia.fr}.\*
If your license agreement include software support, your problem will be taken
care of as specified in the License Agreement.



\section{Known Problems}
\node{Known Problems}
\cpindex{Known Problems}
\cpindex{Known Bugs}
\cpindex{Bugs}

\begin{enumerate}

\item There is a terrible memory leak in X11R4 which may make the \XPK{} grows
indefinitely. Here is the purify trace of this leak:

\begin{verbatim}
      16 bytes (3874 times).     Last memory leak at 0x779040 
Purify (mlk): 61984 total bytes lost, allocated from:
        malloc       [p6.o, pc=0x46d4]
        XtMalloc     [Alloc.o, pc=0x1c8aec]
        XtAppAddWorkProc [NextEvent.o, pc=0x1e8898]
        register_main_loop_from_other_events [line 133, xp-main.c, pc=0x2e280]
        register_main_loop [line 140, xp-main.c, pc=0x2e2b8]
        DoOtherSources [NextEvent.o, pc=0x1e97a0]
\end{verbatim}

\item All the modules with a graphic interface are written under
X11/Motif. Both X11R5/Motif1.2 and X11R4/Motif1.1.X versions are
available. But because of some weirdness with Sun OpenWindows 3.0 X
server, the X11R5 version does not work. So you should not try to use it.
(A patch to Sun OpenWindows 3.0 exists which fixes this problem).

\item Due to some weird interaction between OpenLook Window Manager (olwm)
and Motif application, a number of things may not work as expected. It
is strongly recommended to use mwm (Motif Window Manager) as window
manager, or if not possible twm (the window manager distributed with X11).

\item If Motif is not properly installed on your host, the application may
not works has expected. A very common problem is the lack of file
\file{/usr/lib/X11/XKeysymDB}. The symptom are very easy to identify.
Whenever you start a Motif application, you get a number of message
stating that some Key Sym are undefined. This problem is fixed
in Motif 1.2 (there is now a variable XKEYSYMDB for this purpose).

\item Using the \OPE{} under twm, the Information Dialog which pops up when OP
files are loaded may not properly disappear after the load has completed.

\end{enumerate}

\section{Things to Avoid}
\node{Things to Avoid}
\cpindex{Things to Avoid}

\begin{enumerate}

\item Remember that the word \code{test} is reserved (as are achieve, preserve,
maintain, wait, conclude and retract), and cannot be used as a predicate name,
nor a symbol or a function.

\item Recursive data structures are currently forbidden for many
different reasons (printing problems in particular). So do not build structures
which point on themselves or you will most likely break the kernel. For
example, if \code{foo} is not an evaluable function, the following code will
badly break the kernel (or make it loop...) \code{(! (= @x (foo @x)))}, while
\code{(! (= @x (foo (val @x))))} will work, as it will use the value of
\code{@x} instead of \code{@x}.

\end{enumerate}

\chapter{Glossary}
\node{Glossary}
\cpindex{Glossary}

Here is an alphabetical list of some words and concepts used in this
manual and which may have a particular meaning in \COPRS{}.

\begin{description}

\item[Action:] Actions are the basic operations in \OPRS{}. They are the means
by which a \CPK{} or a \XPK{} modifies or acts on the external world. Actions
can be as simple as printing objects, or sending messages, or complex as RPC or
call to linked code. The user can define his own actions.

\item[Command mode:]. A \CPK{} is usually in Run Mode. When the \OPRSS{}
connects it, it switches to Command Mode (see also Run Mode).

\item[Conditions:]. A condition is a particular statement the \CPK{} is
  periodically checking. It can be a waiting condition (corresponding to a wait
  goal in a OP) , or a preserve condition (corresponding to a passive or active
  preserve in a OP).

\item[\COPRSDE{}:] The \COPRSDE{} is the set of programs provided to
develop an application using procedural reasoning.

\item[\COPRSAE{}:] The \COPRSAE{} is the set of programs provided to
run an application using procedural reasoning.

\item[\CPK{}:] See Kernel.

\item[Database:] The database is where \OPRS{} stores all the static information
representing the state of the world.

\item[Binding Environment:] Binding environments are \OPRS{} structures which
hold the binding of a set of variables in a particular context.

\item[Edge:] The edges of the OPs are the links between the nodes of a OP. They
are labeled with goals (one goal per edge) which must be satisfied to allow
the transition between the two nodes linked by the edge.

\item[Evaluable Function:] Evaluable functions are functions which correspond
to machine executable code and are evaluated when needed. The main difference
between actions and evaluable functions is that evaluable functions appear as
arguments of predicates, and can be evaluated a number of times (depending of
the Current/Quote setting). As a result, they should not have any side effect
which repetition could lead to unexpected results. The user can define his own
evaluable functions.

\item[Evaluable Predicate:] Evaluable predicates are predicates which are not
stored explicitly in the database but are evaluated executing some machine
executable code. The user can define his own evaluable predicates.

\item[Fact:] A fact is an expression which represents a true statement in the
current world state.

\item[Frame:] A Frame is a synonym for binding environment (see also Binding
Environment).

\item[Goal:] A goal is a statement and a temporal operator specifying how this
statement must be treated.

\item[Include File:] Include files are files containing commands to be executed
by a \CPK{} or a \XPK{}.

\item[Intention:] An intention or a task is a stack of procedures (calling each
other) and working towards the achievement of a goal or responding to the
occurrence of an event. This goal or this event are responsible for the
execution of the first procedure in the stack.

\item[Join Node:] Join nodes are node corresponding to a synchronization point
in multi threads execution. The execution can proceed from a join node, only
when a number of threads equal to the number of incoming edges, reached the
join node.

\item[OP:] OP is more of an historical name. It means \dfn{Knowledge
Area}, and was used at the beginning of \OPRS{} research probably to
remind people of other \dfn{Knowledge} constructions such as \dfn{Knowledge
Source} used in Black Board systems. See also \dfn{Procedure}.

\item[\OPE{}:] The \OPE{} is the program which allows the user to build
his own procedures, which can then be executed by a \XPK{} or \CPK{}.

\item[OP File:] A OP file is a file containing a set of OPs. There are no
particular reasons to put some OPs and other together in the same file.
However, it is usually a good practice to break down the set of OPs used in a
particular application into various files.

\item[OP Instance:] A OP instance is a OP which is ``applicable'', i.e. a OP
for which there is a binding environment currently satisfied. The same OP can
have more than one instances applicable at the same time (in which case, the
binding environments are different).

\item[OP Library:] The OP library is the set of OP loaded in a particular \CPK{}
or \XPK{}.

\item[OP Predicate:] A OP predicate is a predicate which can only be satisfied
by calling OPs. It is usually a good idea to declare OP predicates as it tends
to speed up the kernels (OP predicate are not checked in the database as the
system knows they can only be satisfied calling OPs).

\item[Kernel:] The \CPK{} or the \XPK{} are the kernels of the \COPRSDE{}.
However, the kernel part of both programs are identical. This is the reason why
we refer sometimes to them as the Kernel. The Kernel is in fact the ``brain''
of the \COPRSDE{}. In the \CPK{} environment, it displays information and
results in a standard tty interface. In the \XPK{} environment, it displays
graphical information and results in a X11/Motif environment. Besides, this
X11/Motif environment enables the user to interact with the kernel in an easier
way.

\item[Message:] A message is a fact sent to or receives by a \PK{} (in fact by
any module of a \COPRS{} application). Messages are identical to facts, except
that they come from outside, i.e. they are not concluded by the \PK{} itself
but are received from outside.

\item[Motif:] Motif is a commercial product sold by Open Software
Foundation. It is a popular widget set (see Xt) which can be used to
build high level user interfaces with a consistent look and a uniform
interaction mechanism \cite{Motif-manual}.

\item[Node:] The nodes are the building blocks of OPs. They are meant to
present states. When you have achieved a particular goal labeling an edge, you
end up on a new node representing the new state you have reached. 

\item[Plan:] In this documentation, we use interchangeably the
word \dfn{OP}, \dfn{procedure} and even sometimes \dfn{plan}. See OP.

\item[Procedure:] In this documentation, we use interchangeably the
word \dfn{OP}, \dfn{procedure} and even sometimes \dfn{plan}. See OP.

\item[\OPRSS{}:] The \OPRSS{} is used to start/kill \CPK{}s and
\XPK{}s and to interact with the \CPK{}s.

\item[Relevant OP:] Relevant OPs are OPs which ``mention'' a particular fact or
goal in their invocation part. 

\item[Relocatable:] A relocatable is a file containing linked code which can be
relocated by the linker. The \CPK{} and the \XPK{} are distributed as pure
executable but also as relocatable in which the final user can link his own
actions, evaluable functions and predicates and set up a number of other
important variables. The \COPRSDE{} comes with two kinds of relocatable, one to
link C code, and the other to link C++ functions (in the later the main is
called oprs\_main).

\item[Run mode:] A \CPK{} is in run mode when its main loop is executing. See
Command mode. 

\item[Split Node:] A split node is the forking point in multi threads
execution.  When execution reaches a split node, it forks as many threads as
there are outgoing edges from this node.

\item[Thread:] Thread corresponds to an execution sequence in an intention. An
intention has always at least one thread. However, due to parallel split
operations in OPs, more than one thread can be active in an intention.

\item[Unix:] Unix is an operating system family with many members and branches.

\item[Variable:] A variable is a specific symbol which can be bound or not.
When it is, it is bound to a particular value which depends of the current
binding environment.

\item[X Toolkit Intrinsics:] The X Intrinsics is a library which provides high
level mechanism to build user interface. However, the X Intrinsics library does
not preclude any particular look and feel, or any particular manipulated
object. Moreover, the X Intrinsics by itself is not enough to build a user
interface, one needs to associate a widget set (such as the Athena widget set,
the Motif widget, or the Olit widget set, etc) \cite{Xt-manual}.

\item[X11:] X11 is a popular window system available on many platforms
\cite{Xlib-manual}.

\item[\XPK{}:] See Kernel.

\item[XInfo:] XInfo is an \dfn{info} format displaying widget. It is used in
the X application of the \COPRSDE{} to display the manuals and the documentation
of the \COPRSDE{}.

\item[Xt:] See X Toolkit Intrinsics.

\end{description}

\chapter*{General Index}
\addcontentsline{toc}{chapter}{General Index}
\node{General Index}
\cpindex{General Index}

\printindex[cp]

\chapter*{Command Index}
\addcontentsline{toc}{chapter}{Command Index}
\node{Command Index}
\cpindex{Command Index}

\printindex[pg]

\chapter*{Evaluable Function and Action Index}
\addcontentsline{toc}{chapter}{Evaluable Function and Action Index}
\node{Evaluable Function and Action Index}
\cpindex{Evaluable Function and Action Index}

\printindex[ef]

\chapter*{Evaluable Predicate Index}
\addcontentsline{toc}{chapter}{Evaluable Predicate Index}
\node{Evaluable Predicate Index}
\cpindex{Evaluable Predicate Index}

\printindex[ep]

\chapter*{Kernel Function Index}
\addcontentsline{toc}{chapter}{Kernel Function Index}
\node{Kernel and Library Function Index}
\cpindex{Kernel Function Index}

\printindex[fn]

\chapter*{Variable Index}
\addcontentsline{toc}{chapter}{Variable Index}
\node{Variable Index}
\cpindex{Variable Index}

\printindex[vr]


%%% Local Variables: 
%%% mode: latex
%%% TeX-master: "oprs"
%%% End: 
